\documentclass[11pt]{article}
\usepackage[utf8]{inputenc} % si usas pdfLaTeX
\usepackage[T1]{fontenc}    % si usas pdfLaTeX
\usepackage{listings}
\usepackage{xcolor}
\usepackage{listingsutf8}
\usepackage{comment}
\lstset{inputencoding=utf8}
%------------------------------------------------------------------
%------------------------------------------------------------------
%------------------------------------------------------------------
% Informació de l'informe

\newcommand{\titol}{
	 Trabajo 2 - Línea de producción de vacunas y su embalaje
	 }

\newcommand{\titolcap}{Memoria Línea de producción de vacunas y su embalaje}

\newcommand{\AlumnoA}{Carlos Mira López}
\newcommand{\AlumnoB}{Nicolàs Miró Mira}
\newcommand{\AlumnoC}{Vittorio Alessandro Esposito Ceballos}

\newcommand{\AlumnosPie}{\AlumnoA\ -- \AlumnoB\ -- \AlumnoC}
\newcommand{\Asignatura}{Automatización Industrial}
\newcommand{\CursoTitulacion}{4$.\!^\circ$ curso - Grado en Informática Industrial y Robótica }
%https://www.rae.es/dpd/ordinales

\newcommand{\Data}{Diciembre del 2025}

%------------------------------------------------------------------
% Configuració de formats i bibliografia


\usepackage[spanish,es-tabla,es-nosectiondot,es-sloppy,es-noshorthands]{babel} 

\usepackage[utf8]{inputenc}
\usepackage{mathtools}
\usepackage{graphicx,xcolor}
\usepackage{tabularx,booktabs}

\usepackage{float}

\floatstyle{plaintop}
%\floatstyle{ruled}
\newfloat{listado}{htbp}{lol}
\floatname{listado}{Listado}

\usepackage[tableposition=top]{caption}
\renewcommand{\captionlabelfont}{\bfseries\small}
\renewcommand{\captionfont}{\small\itshape}
%\captionsetup[table]{position=above}

\usepackage{enumitem}
\setlist{leftmargin=1.25cm}

\renewcommand{\labelenumi}{\arabic{enumi})}
\renewcommand{\labelenumii}{\alph{enumii})}

% ---------------------------------------------------------------------
% Document electrònic

\usepackage[ 
    colorlinks,
    linkcolor = blue,
    urlcolor = blue,
    citecolor = blue,
    ]{hyperref}

% ---------------------------------------------------------------------
% Configuració de pàgina

\usepackage{geometry}
\geometry{
    a4paper, 
    twoside = false,
    hmargin = {2.5cm,2.5cm},
    vmargin = {1cm,1cm},
    headsep = 1.0cm,
    footskip = 1.5cm,
    includehead, includefoot
    }
        
% ---------------------------------------------------------------------
% Unitats del sistema internacional

\usepackage{siunitx}
\sisetup{
    output-decimal-marker = {,},
    % per-mode = symbol,
    }
    
% ---------------------------------------------------------------------
% Capçaleres i peus de pàgina

\usepackage{fancyhdr}
\pagestyle{fancy}

\usepackage{lastpage}

\fancyhead{} 
% \fancyhead[L]{\footnotesize \sffamily \titolcap}
% \fancyhead[R]{\footnotesize \sffamily \nouppercase \leftmark}
\fancyfoot{} 
% \fancyfoot[R]{\sffamily\footnotesize\thepage/\pageref*{LastPage}}
\ifundef{\AlumnosPie}
	{
		\fancyfoot[L]{\sffamily\footnotesize Se debe configurar el nombre de los alumnos que aparece en el pie}
	}
	{
		\fancyfoot[L]{\sffamily\footnotesize\AlumnosPie}
	}
\renewcommand{\headrulewidth}{0.4pt}
\renewcommand{\footrulewidth}{0.4pt}

% ---------------------------------------------------------------------
% Confiuració de la bibliografia

\usepackage[
	url = false,
	style = apa,
	%style = ieee,
	hyperref = true,
	backref = true,
	]{biblatex}

\usepackage{csquotes}

% ---------------------------------------------------------------------
% On estan les figures? 

\graphicspath{
    {./figuras/}
    {./logos/}
    }

% ---------------------------------------------------------------------
% Informació de portada

\usepackage{xifthen}

\newboolean{LogoUPV}
\newboolean{LogoAlcoi}

\ifundef{\AlumnoB}{\newcommand{\AlumnoB}{}}{}
\ifundef{\AlumnoC}{\newcommand{\AlumnoC}{}}{}

\title{\Huge\bfseries%
	\vspace*{-1cm}
    \ifLogoAlcoi
	   \includegraphics[width=5cm]{UPV_Campus_Alcoi}\\
    \else
        \includegraphics[width=5cm]{UPV_horitzontal_color}\\
    \fi
    \vspace*{6cm}
    \titol
    \vspace*{3.5cm}
    ~
    }
    
\author{
    \AlumnoA\\[1ex]
    \AlumnoB\\[1ex]
	\AlumnoC\\[3.5cm]
  	\textbf{\Asignatura}\\[2ex]
  	\textbf{\CursoTitulacion}
    }
    
\date{\Data}

% ---------------------------------------------------------------------
% Format de paràgraf

\parindent = 0cm
\parskip = 2ex
\partopsep = -1ex

% ---------------------------------------------------------------------
% Control de línies orfes i vídues

\clubpenalty = 10000
\widowpenalty = 10000
\displaywidowpenalty = 10000

% ---------------------------------------------------------------------
% Comandaments personalitzats

\usepackage{xspace}

\newcommand{\matlab}{{\textsc{Matlab}}\xspace}
\newcommand{\simulink}{\textit{Simulink}\xspace}


% ---------------------------------------------------------------------
% Fi
\input{./configuraciones/preambulo_listings}
\input{./configuraciones/preambul_GRAFCET}
\usepackage{listings}

\lstset{
	language=Pascal,
	basicstyle=\scriptsize\ttfamily,         
	commentstyle=\color[rgb]{.2,.5,.2},
	tabsize=4,
	classoffset=0,
	deletekeywords=[1]{reset},	
	morekeywords={
		END_IF, ELSIF, 
		},
	keywordstyle=\color{blue},
	morecomment=[l]{//}
%	numbers=left,                    
%	numbersep=4pt,                   
%	numberstyle=\sffamily\tiny\color[rgb]{.4,.4,.4}, 		
	}

% Si vols utilitzar un tipus de lletra semblant a Arial, descomenta les dos línies següents:
% \usepackage{cmbright}
% \usepackage[OT1]{fontenc}

\bibliography{./configuraciones/referencias}

%------------------------------------------------------------------
% Logo:

\setboolean{LogoUPV}{false}
\setboolean{LogoAlcoi}{true}


\parindent = 0cm
\parskip = 2ex

%------------------------------------------------------------------
%------------------------------------------------------------------
%------------------------------------------------------------------

\begin{document}

% -------------------------------------
% -------------------------------------


\renewcommand{\itemautorefname}{punto}
\renewcommand{\sectionautorefname}{sección}
\renewcommand{\subsectionautorefname}{subsección}
\renewcommand{\subsubsectionautorefname}{subsección}
\renewcommand{\appendixautorefname}{apéndice}
\renewcommand{\figureautorefname}{figura}
\renewcommand{\tableautorefname}{tabla}

\renewcommand{\indexname}{Índice alfabético}
\renewcommand{\bibname}{Bibliograf\'{\i}a}
\renewcommand{\contentsname}{Índice general}
\renewcommand{\abstractname}{Resumen}	

\def\listadoautorefname{listado}

% Per a les ecuacions en una línia
\abovedisplayshortskip = -1.0ex plus 0ex minus 0.25ex
\belowdisplayshortskip = 2.0ex plus 1ex minus 0.0ex

% Per a les equacions en varies línies
\abovedisplayskip = -1.0ex plus 0ex minus 0.25ex
\belowdisplayskip = 2.0ex plus 1ex minus 0.0ex

\begin{titlepage}
    \maketitle
    \thispagestyle{empty}
\end{titlepage}

\renewcommand{\thepage}{\Roman{page}}
\fancyfoot[R]{\sffamily\footnotesize\thepage}
\fancyhead{} 

\renewcommand{\headrulewidth}{0.0pt}

\setcounter{page}{1}

{
\footnotesize
\parskip=1ex

\tableofcontents

\clearpage

\listoffigures

\bigskip

\listoftables

}

\clearpage

\renewcommand{\thepage}{\arabic{page}}
\fancyfoot[R]{\sffamily\footnotesize\thepage/\pageref*{LastPage}}
\setcounter{page}{1}

\fancyhead[L]{\footnotesize \sffamily \titolcap}
\fancyhead[R]{\footnotesize \sffamily \nouppercase \leftmark}

\renewcommand{\headrulewidth}{0.4pt} % No eliminar!!!

% -------------------------------------
% -------------------------------------


%------------------------------------------------------------------
%------------------------------------------------------------------
% Resumen

\phantomsection
\addcontentsline{toc}{section}{Resumen}
%\section*{Resumen}

%Este documento es un pequeño manual para la realización de informes científico-técnicos y trabajos académicos. Se describe la estructura que debe tener un documento de este tipo y se describen las directrices estándar mínimas parar la redacción del contenido. También se incluyen recomendaciones para incluir figuras, tablas y ecuaciones, así como sobre la forma de hacer referencia a estos elementos en el documento. Por último, se incluyen ejemplos de figuras, tablas, ecuaciones y referencias bibliográficas.

%Este documento también se puede utilizar como plantilla ya que los formatos son los adecuados para un documento científico-técnico.

%\textbf{Palabras clave}: informe científico-técnico; plantilla; \LaTeX, Overleaf.

%------------------------------------------------------------------
%------------------------------------------------------------------

%---- VAMOS A INTENTAR QUE SEAN CAPITULOS SEPARADOS

%\section*{Resumen}

En este proyecto se detalla el proceso completo desarrollado para resolver el problema propuesto de automatización 
y gestión de alarmas en una línea indexada. La solución se ha basado en el diseño de Grafcets que definen la secuencia 
de operaciones del sistema y su implementación en un PLC Siemens mediante el lenguaje SCL en el entorno TIA Portal. 
Además, el sistema integra una botonera externa cableada al PLC, desde la cual se generan las señales de control 
correspondientes a las órdenes de ejecución, parada y rearme del proceso.

Palabras clave: GRAFCET, SCL, TIA Portal, PLC.




%La segunda página de la memoria será un resumen del TFG/TFM tanto en la lengua de redacción del trabajo como en inglés. El resumen deberá reflejar el contenido de trabajo de forma precisa y descriptiva (entre 50 y 200 palabras).

%En esa misma página se incluirán, como máximo, 5 palabras clave. Las palabras clave asignadas al trabajo deberán reflejar la materia, método, lugar o cualquier otro aspecto relevante para la recuperación del trabajo en búsquedas bibliográficas en una base de datos. 
\section{Introducción}
\label{sec:introduccion}
\begin{figure}[H]
	\centering
	\scalebox{0.75}{%\nodeDist = 2.5cm
%\retornoDist = 0.7\nodeDist % Distància vertical dels retorns
%\bifDistX = 10ex
%\bifDistY = 1.1\nodeDist
%\sincDistYabove = 0.85\nodeDist
%\sincDistYbelow = 0.6\sincDistYabove
%\sincDistYBlockbelow = 0.5\nodeDist


\begin{tikzpicture}[auto]
	
	\ttfamily
	
	% --------------------------------------------
	\etapaInicial{0}
	\primeraAccion{0}{X0-A1}{$\mathtt{C:=0}$}
	\sobreActivacion{X0-A1}{X0-A2}{}
	
	
	\bifurcacion[100]{0}{1}{Marcha $\cdot$ \NOT{MAN}}{4}{MAN}

	\retornoInicio[-6em]{4}{0}{\NOT{MAN}}

	% --------------------------------------------
  	{
   	\renewcommand{\transitionPos}{0.35}
   	%\etapa[1.25\nodeDist]{1}{0}{MARCHA};
	\primeraAccion{1}{X1-A1}{Act\_Mantenimiento}
	%\otraAccion{X1-A1}{X1-A2}{QCINALIM}
	}
	
	% --------------------------------------------
   	\etapa[1.5\nodeDist]{2}{1}{Mantenimiento\_Fin};
	\primeraAccion{2}{X2-A1}{Proceso\_Activo}
	%\condicionada{X2-A2}{CONDICION}
	%\comentari[1]{X2-A1}{Máquina fresadora en marcha}

	\etapa[1.25\nodeDist]{3}{2}{Pieza\_Empaquetada};
	\primeraAccion{3}{X3-A1}{Proceso\_Activo}
	\otraAccion{X3-A1}{X3-A2}{$\mathtt{C:=C+1}$}
	\sobreActivacion{X3-A2}{X3-A3}{}

	\bifurcacionRetorno[15ex]{3}{6}{7}
	\retornoInicio[6em]{6}{0}{Paro\_Ciclo}
	{
	\setlength{\enlaceRetornoDist}{0.30\nodeDist}
	\retorno[18em]{7}{2}{\NOT{Paro\_Ciclo}}
	}
	%\retornoInicio[7em]{4}{0}{\NOT{INICIO}}
	%\retorno[18em]{4}{3}{INICIO}


            
\end{tikzpicture}

}
	\medskip
	\label{graf:fresadora}
	\caption{Grafcet de la fresadora}
\end{figure}

\begin{figure}[H]
	\centering
	\scalebox{0.75}{%\nodeDist = 2.5cm
%\retornoDist = 0.7\nodeDist % Distància vertical dels retorns
%\bifDistX = 10ex
%\bifDistY = 1.1\nodeDist
%\sincDistYabove = 0.85\nodeDist
%\sincDistYbelow = 0.6\sincDistYabove
%\sincDistYBlockbelow = 0.5\nodeDist


\begin{tikzpicture}[auto]
	
	\ttfamily
	
	% --------------------------------------------
	\etapaInicial{0P}
		

	% --------------------------------------------
  	{
   	\renewcommand{\transitionPos}{0.35}
   	\etapa[1.25\nodeDist]{1}{0}{PARO};
	\primeraAccion{1}{X1-A1}{Paro\_de\_Ciclo}
	%\otraAccion{X1-A1}{X1-A2}{QCINALIM}
	}
	\retornoInicio[6em]{1}{0}{X0}
	%\retornoInicio[7em]{4}{0}{\NOT{INICIO}}
	%\retorno[18em]{4}{3}{INICIO}


            
\end{tikzpicture}

}
	\medskip
	\label{graf:fresadora}
	\caption{Grafcet de la fresadora}
\end{figure}

\begin{figure}[H]
	\centering
	\scalebox{0.75}{%\nodeDist = 2.5cm
%\retornoDist = 0.7\nodeDist % Distància vertical dels retorns
%\bifDistX = 10ex
%\bifDistY = 1.1\nodeDist
%\sincDistYabove = 0.85\nodeDist
%\sincDistYbelow = 0.6\sincDistYabove
%\sincDistYBlockbelow = 0.5\nodeDist


\begin{tikzpicture}[auto]
	
	\ttfamily
	
	% --------------------------------------------
	\etapaInicial{10}
	\primeraAccion{10}{X10-A1}{CINTA\_LIBRE}	

	%-----
	\renewcommand{\transitionPos}{0.35}
   	\etapa[1.25\nodeDist]{11}{10}{Proceso\_Activo};
	\primeraAccion{11}{X11-A1}{Emisor}
	\otraAccion{X11-A1}{X11-A2}{QC1\_A}
	\otraAccion{X11-A2}{X11-A3}{QC1\_B}
	

	%---
	\etapa[1.25\nodeDist]{11b}{11}{SPress\_X};
	\primeraAccion{11b}{X11-A1}{QC1\_B}

	\etapa[1.25\nodeDist]{12}{11b}{\NOT{SPress\_X}};
	\primeraAccion{12}{X12-A1}{QC1\_A}
	\otraAccion{X12-A1}{X12-A2}{QC1\_B}
	


	%---
	%\etapa[1.25\nodeDist]{13}{12}{S\_VAC\_A $\cdot$ S\_PRESENCIA};
	

	%---
	\etapa[1.25\nodeDist]{13}{12}{S1};
	

	%---
	%\etapa[1.25\nodeDist]{14}{12}{\NOT(S\_VAC\_A) $\cdot$ S\_PRESENCIA};

	%----
	\etapa[1.25\nodeDist]{14}{13}{Descartador\_Libre};
	\primeraAccion{14}{X14-A1}{QC1\_B}
	\otraAccion{X14-A1}{X14-A2}{Act\_Descartador\_B}

	\etapa[1.25\nodeDist]{15}{14}{\NOT{S1}};


	%----
	\retornoInicio[6em]{15}{0}{Pieza\_Rec}
	


	

            
\end{tikzpicture}}
	\medskip
	\label{graf:fresadora}
	\caption{Grafcet de la fresadora}
\end{figure}

\begin{figure}[H]
	\centering
	\scalebox{0.75}{%\nodeDist = 2.5cm
%\retornoDist = 0.7\nodeDist % Distància vertical dels retorns
%\bifDistX = 10ex
%\bifDistY = 1.1\nodeDist
%\sincDistYabove = 0.85\nodeDist
%\sincDistYbelow = 0.6\sincDistYabove
%\sincDistYBlockbelow = 0.5\nodeDist


\begin{tikzpicture}[auto]
	
	\ttfamily
	
	% --------------------------------------------
	\etapaInicial{15}
	\primeraAccion{15}{X15-A1}{Descartador\_Libre}

	\bifurcacion[20ex]{15}{16}{SOPORTE\_A}{17}{SOPORTE\_B}
	%-----
	\primeraAccion{16}{X16-A1}{QC2\_VAC}
	\primeraAccion{17}{X17-A1}{QC2\_VAC}
	

	%---
	\etapa[1.25\nodeDist]{16b}{16}{S1GIR\_VAC};
	\comentari[3]{X16b}{Esperamos máquina de calidad libre}

	%---
	\etapa[1.25\nodeDist]{16c}{16b}{MAQ\_CAL\_VAC\_LIBRE};
	\primeraAccion{16c}{X16c-A1}{QC2\_VAC}
	\otraAccion{X16c-A1}{X16c-A2}{ACT\_MAQ\_CAL\_VAC}

	%---
	
	\etapa[1.25\nodeDist]{17b}{17}{S1GIR\_VAC};
	\primeraAccion{17b}{X17b-A1}{QC2\_VAC}

	



	\retornoInicio[6em]{16c}{0}{\NOT {S1GIR\_VAC}}
	\retornoInicio[-9em]{17b}{0}{\NOT {S1GIR\_VAC}}




		




	

            
\end{tikzpicture}}
	\medskip
	\label{graf:fresadora}
	\caption{Grafcet de la fresadora}
\end{figure}



\begin{figure}[H]
	\centering
	\scalebox{0.75}{%\nodeDist = 2.5cm
%\retornoDist = 0.7\nodeDist % Distància vertical dels retorns
%\bifDistX = 10ex
%\bifDistY = 1.1\nodeDist
%\sincDistYabove = 0.85\nodeDist
%\sincDistYbelow = 0.6\sincDistYabove
%\sincDistYBlockbelow = 0.5\nodeDist


\begin{tikzpicture}[auto]
	
	\ttfamily
	
% --------------------------------------------
	\etapaInicial{30}
	\primeraAccion{30}{X30-A1}{MAQ\_Libre}
	%---

	\bifurcacion[100]{30}{31}{Act\_Maq\_A\_Buena}{32}{Act\_Maq\_A\_Mala}

	\primeraAccion{31}{X31-A1}{QC3}
	\primeraAccion{32}{X32-A1}{QC3}
	
	%---
	
	%---
	\etapa[1.25\nodeDist]{31b}{31}{S3};

	\etapa[1.25\nodeDist]{31c}{31b}{MAQ\_SIG\_LIBRE};
	\primeraAccion{31c}{X31c-A1}{QC3}
	\otraAccion{X31c-A1}{X31c-A2}{SigMaqBuena}


	\etapa[1.25\nodeDist]{32b}{32}{S3};

	\etapa[1.25\nodeDist]{32c}{32b}{MAQ\_SIG\_LIBRE};
	\primeraAccion{32c}{X32c-A1}{QC3}
	\otraAccion{X32c-A1}{X32c-A2}{SigMaqMala}
	
	\retornoInicio[6em]{31c}{0}{\NOT{S3}}
	\retornoInicio[-16em]{32c}{0}{\NOT{S3}}

	








	




	

            
\end{tikzpicture}}
	\medskip
	\label{graf:fresadora}
	\caption{Grafcet de la fresadora}
\end{figure}

\begin{figure}[H]
	\centering
	\scalebox{0.75}{%\nodeDist = 2.5cm
%\retornoDist = 0.7\nodeDist % Distància vertical dels retorns
%\bifDistX = 10ex
%\bifDistY = 1.1\nodeDist
%\sincDistYabove = 0.85\nodeDist
%\sincDistYbelow = 0.6\sincDistYabove
%\sincDistYBlockbelow = 0.5\nodeDist


\begin{tikzpicture}[auto]
	
	\ttfamily
	
	\etapaInicial{40}
	\primeraAccion{40}{X40-A1}{Descartador\_Libre}
	%---

	\bifurcacion[100]{40}{41}{Act\_Descartador}{42}{SigMaqMala}

	\primeraAccion{41}{X41-A1}{QC4}
	\primeraAccion{42}{X42-A1}{QC4}
	
	%---
	
	%---
	\etapa[1.25\nodeDist]{41b}{41}{S4};
	\comentari[3]{X31b}{Esperamos máquina de calidad libre}

	\etapa[1.25\nodeDist]{41c}{41b}{SIG\_MAQ\_LIBRE};
	\primeraAccion{41c}{X41c-A1}{QC4}
	\otraAccion{X41c-A1}{X41c-A2}{ACT\_SIG\_MAQ}


	\etapa[1.25\nodeDist]{42b}{42}{S4};
	\primeraAccion{42b}{X42b-A1}{QTRANS}

	
	
	\retornoInicio[6em]{41c}{0}{\NOT{S4}}
	\retornoInicio[-16em]{42b}{0}{\NOT{S4}}











	




	

            
\end{tikzpicture}}
	\medskip
	\label{graf:fresadora}
	\caption{Grafcet de la fresadora}
\end{figure}

\begin{figure}[H]
	\centering
	\scalebox{0.75}{%\nodeDist = 2.5cm
%\retornoDist = 0.7\nodeDist % Distància vertical dels retorns
%\bifDistX = 10ex
%\bifDistY = 1.1\nodeDist
%\sincDistYabove = 0.85\nodeDist
%\sincDistYbelow = 0.6\sincDistYabove
%\sincDistYBlockbelow = 0.5\nodeDist


\begin{tikzpicture}[auto]
	
	\ttfamily
	
	\etapaInicial{50}
	\primeraAccion{50}{X50-A1}{MAQ\_CAL\_VAC\_LIBRE}

	\etapa[1\nodeDist]{51}{50}{Act\_Maq};
	\primeraAccion{51}{X51-A1}{QC5\_VAC}
	\otraAccion{X51-A1}{X51-A2}{QC6\_VAC}
	\otraAccion{X51-A2}{X51-A3}{QC7\_VAC}

	%---

	%---
	\etapa[1\nodeDist]{52}{51}{S5};
	\primeraAccion{52}{X52-A1}{QC5\_VAC}
	\otraAccion{X52-A1}{X52-A2}{QC6\_VAC}
	\otraAccion{X52-A2}{X52-A3}{QC7\_VAC}
	

	%---
	\etapa[1\nodeDist]{53}{52}{1s/X52};
	\primeraAccion{53}{X53-A1}{QCLAMP}

	\etapa[1\nodeDist]{54}{53}{1s/X53};
	\primeraAccion{54}{X54-A1}{HAYVacuna}
	\otraAccion{X54-A1}{X54-A2}{QCLAMP}

	\etapa[1\nodeDist]{55}{54}{Hay\_Soporte};
	\primeraAccion{55}{X55-A1}{QBAJ}
	\otraAccion{X55-A1}{X55-A2}{QCLAMP}

	\etapa[1\nodeDist]{56}{55}{2s/X55};
	\primeraAccion{56}{X56-A1}{QBAJ}
	\otraAccion{X56-A1}{X56-A2}{GRIPPER}
	\otraAccion{X56-A2}{X56-A3}{QCLAMP}

	\etapa[1\nodeDist]{57}{56}{2s/X56};
	\primeraAccion{57}{X57-A1}{GRIPPER}
	\otraAccion{X57-A1}{X57-A2}{QCLAMP}

	\etapa[1\nodeDist]{58}{57}{2s/X57};
	\primeraAccion{58}{X58-A1}{GRIPPER}
	\otraAccion{X58-A1}{X58-A2}{QHOR}

	\etapa[1\nodeDist]{59}{58}{2s/X58};
	\primeraAccion{59}{X59-A1}{GRIPPER}
	\otraAccion{X59-A1}{X59-A2}{QBAJ\_MAQ}
	\otraAccion{X59-A2}{X56-A3}{QHOR}

	\etapa[1\nodeDist]{60}{59}{2s/X59};
	\primeraAccion{60}{X60-A1}{QHOR}
	\otraAccion{X60-A1}{X60-A2}{QBAJ\_MAQ}

	\etapa[1\nodeDist]{61}{60}{2s/X60};
	\primeraAccion{61}{X61-A1}{QHOR}

	\etapa[1\nodeDist]{62}{61}{2s/X61};
	\primeraAccion{62}{X62-A1}{Pieza\_Empaquetada}

	
	

	\retornoInicio[13em]{62}{0}{\underline{1}}
	













	




	

            
\end{tikzpicture}}
	\medskip
	\label{graf:fresadora}
	\caption{Grafcet de la fresadora}
\end{figure}

\begin{figure}[H]
	\centering
	\scalebox{0.75}{%\nodeDist = 2.5cm
%\retornoDist = 0.7\nodeDist % Distància vertical dels retorns
%\bifDistX = 10ex
%\bifDistY = 1.1\nodeDist
%\sincDistYabove = 0.85\nodeDist
%\sincDistYbelow = 0.6\sincDistYabove
%\sincDistYBlockbelow = 0.5\nodeDist


\begin{tikzpicture}[auto]
	
	\ttfamily
	
	\etapaInicial{70}
	\primeraAccion{70}{X70-A1}{MAQ\_CAL\_VAC\_LIBRE}
	%---
	\etapa[1.25\nodeDist]{71}{70}{Act\_Maq};
	\primeraAccion{71}{X71-A1}{QC5\_VAC}
	\otraAccion{X71-A1}{X71-A2}{QC6\_VAC}
	\otraAccion{X71-A2}{X71-A3}{QC7\_VAC}
	
	%--
	\etapa[1.25\nodeDist]{72}{71}{S5};
	\primeraAccion{72}{X72-A1}{QC5\_VAC}
	\otraAccion{X72-A1}{X72-A2}{QC6\_VAC}
	\otraAccion{X72-A2}{X72-A3}{QC7\_VAC}
	%---
	\etapa[1.25\nodeDist]{73}{72}{1s/X72};
	\primeraAccion{73}{X73-A1}{QCLAMP}

	\etapa[1.25\nodeDist]{74}{73}{1s/X73};
	\primeraAccion{74}{X74-A1}{HAYSoporte}

	\etapa[1.25\nodeDist]{75}{74}{Pieza\_Empaquetada};

	\etapa[1.25\nodeDist]{76}{75}{SIG\_MAQ\_LIBRE};
	\primeraAccion{76}{X76-A1}{QCLAMP\_SUB}
	\otraAccion{X76-A1}{X76-A2}{QC7\_VAC}
	\otraAccion{X76-A2}{X76-A3}{Act\_SIG\_MAQ}

	\retornoInicio[6em]{76}{0}{\underline{1}}

\end{tikzpicture}}
	\medskip
	\label{graf:fresadora}
	\caption{Grafcet de la fresadora}
\end{figure}

\begin{figure}[H]
	\centering
	\scalebox{0.75}{%\nodeDist = 2.5cm
%\retornoDist = 0.7\nodeDist % Distància vertical dels retorns
%\bifDistX = 10ex
%\bifDistY = 1.1\nodeDist
%\sincDistYabove = 0.85\nodeDist
%\sincDistYbelow = 0.6\sincDistYabove
%\sincDistYBlockbelow = 0.5\nodeDist


\begin{tikzpicture}[auto]
	
	\ttfamily
	
	\etapaInicial{80}
	\primeraAccion{80}{X80-A1}{Descartador\_Libre}
	%---
	\etapa[1.25\nodeDist]{81}{80}{Act\_Maq\_A\_Buena};
	\primeraAccion{81}{X81-A1}{QC5\_VAC}
	
	%--
	\etapa[1.25\nodeDist]{82}{81}{S5};
	\comentari[3]{X82}{Esperamos máquina de calidad libre}

	%---
	\etapa[1.25\nodeDist]{83}{82}{SIG\_MAQ\_LIBRE};
	\primeraAccion{83}{X83-A1}{QC3}
	\otraAccion{X83-A1}{X83-A2}{Act\_SIG\_MAQ}
	




	\retornoInicio[6em]{83}{0}{\NOT{S5}}

\end{tikzpicture}}
	\medskip
	\label{graf:fresadora}
	\caption{Grafcet de la fresadora}
\end{figure}

\begin{figure}[H]
	\centering
	\scalebox{0.75}{%\nodeDist = 2.5cm
%\retornoDist = 0.7\nodeDist % Distància vertical dels retorns
%\bifDistX = 10ex
%\bifDistY = 1.1\nodeDist
%\sincDistYabove = 0.85\nodeDist
%\sincDistYbelow = 0.6\sincDistYabove
%\sincDistYBlockbelow = 0.5\nodeDist


\begin{tikzpicture}[auto]
	
	\ttfamily
	
% --------------------------------------------
	\ttfamily
	
	\etapaInicial{90}
	\primeraAccion{90}{X90-A1}{MAQ\_Libre}
	
	
	%---
	\etapa[1.25\nodeDist]{91}{90}{Act\_MAQ};
	\primeraAccion{91}{X91-A1}{QCINTA}
	
	%--
	\etapa[1.25\nodeDist]{92}{91}{SENSOR};

	%---
	\etapa[1.25\nodeDist]{93}{92}{SIG\_MAQ\_LIBRE};
	\primeraAccion{93}{X93-A1}{QTRANS}

	\etapa[1.25\nodeDist]{94}{93}{\NOT{SENSOR}};
	\primeraAccion{94}{X94-A1}{Act\_SIG\_MAQ}
	
	




	\retornoInicio[6em]{94}{0}{\underline{1}}
	
       
\end{tikzpicture}}
	\medskip
	\label{graf:fresadora}
	\caption{Grafcet de la fresadora}
\end{figure}

\begin{figure}[H]
	\centering
	\scalebox{0.75}{%\nodeDist = 2.5cm
%\retornoDist = 0.7\nodeDist % Distància vertical dels retorns
%\bifDistX = 10ex
%\bifDistY = 1.1\nodeDist
%\sincDistYabove = 0.85\nodeDist
%\sincDistYbelow = 0.6\sincDistYabove
%\sincDistYBlockbelow = 0.5\nodeDist


\begin{tikzpicture}[auto]
	
	\ttfamily
	
% --------------------------------------------
	\ttfamily
	
	\etapaInicial{100}
	\primeraAccion{100}{X100-A1}{MAQ\_LIBRE}
    \otraAccion{X100-A1}{X100-A2}{$\mathtt{C:=0}$}
  \sobreActivacion{X100-A2}{X100-A3}{}
	%---

	\etapa[1.25\nodeDist]{101}{100}{Act\_Maq};
	\primeraAccion{101}{X101-A1}{QC7}
	
	%--
	\etapa[1.25\nodeDist]{102}{101}{S5};
	\primeraAccion{102}{X102-A1}{QC7}

	%---
	\etapa[1.25\nodeDist]{103}{102}{1s/X102};
	\primeraAccion{103}{X103-A1}{QCLAMP}

	\etapa[1.25\nodeDist]{104}{103}{1s/X103};
	\primeraAccion{104}{X104-A1}{HAYSPieza}

  \etapa[1.25\nodeDist]{105}{104}{Pieza\_Empaquetada};
  \primeraAccion{105}{X105-A1}{$\mathtt{C:=C+1}$}
  \sobreActivacion{X105-A1}{X105-A2}{}

  %\retornoInicio[-10em]{105}{104}{$\mathtt{C<2}$}
  %\etapa[1.25\nodeDist]{104}{105}{$\mathtt{C<2}$}
  \saltoHorizontal[5.5cm]{105}{104}{$\mathtt{C<2}$}
  
  \etapa[1.4\nodeDist]{106}{105}{$\mathtt{C==1}$};
	\primeraAccion{106}{X106-A1}{QCLAMP\_SUB}
	\otraAccion{X106-A1}{X106-A2}{QC7}

  \etapa[1.25\nodeDist]{107}{106}{3s/X106};
	\primeraAccion{107}{X107-A1}{Pieza\_Entregada}
	
	

	\retornoInicio[6em]{107}{0}{\underline{1}}
	
       
\end{tikzpicture}}
	\medskip
	\label{graf:fresadora}
	\caption{Grafcet de la fresadora}
\end{figure}
\section{Metodolgía}\label{sec:metodologia}

\begin{itemize}
    \item Análisis del problema
    \item Crear el entorno virtual
    \item Comunicar el PLC con el entorno de la máquina en 3D
    \item Generar los grafcets para el proceso normal
    \item Implementar y evaluar la secuencia normal en el PLC
    \item Creación control con Marcha-Paro
    \item Implementar y evaluar el Marcha-Paro
    \item Generar el grafcet de mantenimiento
    \item Implementar y evaluar el manteniendo
    \item Generar los grafcets de Alarmas
    \item Implementar y evaluar las Alarmas
    \item Generar el grafcet de Emergencia
    \item Implementar y evaluar la Emergencia
    \item Generar el grafcet de vaciado
    \item Implementar y evaluar el vaciado
    \item Crear el HMI
    \item Implementar y evaluar el HMI
\end{itemize}


\section{Estructura TIA Portal}\label{sec:estructura_codigos_scl}
Respecto a la estructura del bloque main del proyecto en TIA Portal, se ha seguido una metodología de 
dividir el bloque main en varias ramas, donde cada rama corresponde a una línea o función del sistema.


\subsection{Línea 1}
En la línea 0 se encuentra el grafcet principal del sistema, que se encarga de coordinar el funcionamiento de todas las máquinas y líneas del sistema.
\begin{figure}[H]
	\centering
	\includegraphics[width=0.8\linewidth]{./Figuras/linea1.png}
	\caption{Estructura de la Línea 1}
	\label{fig:linea1}
\end{figure}



\subsection{Línea 2}
En la línea 2 vemos el bloque que corresponde con el modo de mantenimiento.

\begin{figure}[H]
	\centering
	\includegraphics[width=0.8\linewidth]{./Figuras/linea2.png}
	\caption{Estructura de la Línea 2}
	\label{fig:linea2}
\end{figure}


\subsection{Línea 3}
En esta línea podemos encontrar el modo manual del sistema.

\begin{figure}[H]
	\centering
	\includegraphics[width=0.8\linewidth]{./Figuras/linea3.png}
	\caption{Estructura de la Línea 3}
	\label{fig:linea3}
\end{figure}


\subsection{Línea 4}
En la siguiente línea encontramos el bloque que gestiona el vaciado general del sistema,
y más ramas donde cada una controla el vaciado de cada línea de producción.

\begin{figure}[H]
	\centering
	\includegraphics[width=0.6\linewidth]{./Figuras/linea4.png}
	\caption{Estructura de la Línea 4}
	\label{fig:linea4}
\end{figure}

\begin{figure}[H]
	\centering
	\includegraphics[width=0.8\linewidth]{./Figuras/linea4b.png}
	\caption{Estructura de la Línea 4}
	\label{fig:linea4b}
\end{figure}


\subsection{Línea 5}
Aquí encontramos todos los bloques que controlan la línea de las vacunas.
\begin{figure}[H]
	\centering
	\includegraphics[width=0.8\linewidth]{./Figuras/linea5.png}
	\caption{Estructura de la Línea 5}
	\label{fig:linea5}
\end{figure}


\subsection{Línea 6}
Aquí encontramos todos los bloques que controlan la línea de los soportes.
\begin{figure}[H]
	\centering
	\includegraphics[width=0.8\linewidth]{./Figuras/linea6.png}
	\caption{Estructura de la Línea 6}
	\label{fig:linea6}
\end{figure}


\subsection{Línea 7}
Aquí encontramos todos los bloques que controlan la línea de embalaje.
\begin{figure}[H]
	\centering
	\includegraphics[width=0.8\linewidth]{./Figuras/linea7.png}
	\caption{Estructura de la Línea 7}
	\label{fig:linea7}
\end{figure}





%Para realizar este proyecto, se ha seguido la siguiente metodología.

\section{Grafcets}
\label{sec:introduccion}

\subsection{Grafcet Principal}
\begin{figure}[H]
	\centering
	\scalebox{0.75}{%\nodeDist = 2.5cm
%\retornoDist = 0.7\nodeDist % Distància vertical dels retorns
%\bifDistX = 10ex
%\bifDistY = 1.1\nodeDist
%\sincDistYabove = 0.85\nodeDist
%\sincDistYbelow = 0.6\sincDistYabove
%\sincDistYBlockbelow = 0.5\nodeDist


\begin{tikzpicture}[auto]
	
	\ttfamily
	
	% --------------------------------------------
	\etapaInicial{0}
	\primeraAccion{0}{X0-A1}{$\mathtt{C:=0}$}
	\sobreActivacion{X0-A1}{X0-A2}{}
	
	
	\bifurcacion[100]{0}{1}{Marcha $\cdot$ \NOT{MAN}}{4}{MAN}

	\retornoInicio[-6em]{4}{0}{\NOT{MAN}}

	% --------------------------------------------
  	{
   	\renewcommand{\transitionPos}{0.35}
   	%\etapa[1.25\nodeDist]{1}{0}{MARCHA};
	\primeraAccion{1}{X1-A1}{Act\_Mantenimiento}
	%\otraAccion{X1-A1}{X1-A2}{QCINALIM}
	}
	
	% --------------------------------------------
   	\etapa[1.5\nodeDist]{2}{1}{Mantenimiento\_Fin};
	\primeraAccion{2}{X2-A1}{Proceso\_Activo}
	%\condicionada{X2-A2}{CONDICION}
	%\comentari[1]{X2-A1}{Máquina fresadora en marcha}

	\etapa[1.25\nodeDist]{3}{2}{Pieza\_Empaquetada};
	\primeraAccion{3}{X3-A1}{Proceso\_Activo}
	\otraAccion{X3-A1}{X3-A2}{$\mathtt{C:=C+1}$}
	\sobreActivacion{X3-A2}{X3-A3}{}

	\bifurcacionRetorno[15ex]{3}{6}{7}
	\retornoInicio[6em]{6}{0}{Paro\_Ciclo}
	{
	\setlength{\enlaceRetornoDist}{0.30\nodeDist}
	\retorno[18em]{7}{2}{\NOT{Paro\_Ciclo}}
	}
	%\retornoInicio[7em]{4}{0}{\NOT{INICIO}}
	%\retorno[18em]{4}{3}{INICIO}


            
\end{tikzpicture}

}
	\medskip
	\label{graf:principal}
	\caption{Grafcet Principal}
\end{figure}

\subsection{Grafcet Paro}
\begin{figure}[H]
	\centering
	\scalebox{0.75}{%\nodeDist = 2.5cm
%\retornoDist = 0.7\nodeDist % Distància vertical dels retorns
%\bifDistX = 10ex
%\bifDistY = 1.1\nodeDist
%\sincDistYabove = 0.85\nodeDist
%\sincDistYbelow = 0.6\sincDistYabove
%\sincDistYBlockbelow = 0.5\nodeDist


\begin{tikzpicture}[auto]
	
	\ttfamily
	
	% --------------------------------------------
	\etapaInicial{0P}
		

	% --------------------------------------------
  	{
   	\renewcommand{\transitionPos}{0.35}
   	\etapa[1.25\nodeDist]{1}{0}{PARO};
	\primeraAccion{1}{X1-A1}{Paro\_de\_Ciclo}
	%\otraAccion{X1-A1}{X1-A2}{QCINALIM}
	}
	\retornoInicio[6em]{1}{0}{X0}
	%\retornoInicio[7em]{4}{0}{\NOT{INICIO}}
	%\retorno[18em]{4}{3}{INICIO}


            
\end{tikzpicture}

}
	\medskip
	\label{graf:paro}
	\caption{Grafcet del botón de Paro}
\end{figure}




\subsection{Grafcet Máquina 1}



Como vemos en la siguiente figura (\hyperref[fig:maq1]{Máquina 1}), esta máquina está compuesta por:
\begin{itemize}
	\item 1 Cinta para Vacunas y Soportes, 2 Cintas para los embalajes.
	\item 1 Sensor de presencia al principio de la cinta, o al final de la segunda en los embalajes.
	\item 1 Sensor final de la cinta
	\item 1 Emisor de objetos.
\end{itemize}

\begin{figure}[H]
	\centering
	\includegraphics[width=0.8\linewidth]{./Figuras/maq1.png}
	\medskip
	\label{fig:maq1}
	\caption{Máquina 1 de la planta}
\end{figure}

Con estos elementos conformamos el siguiente grafcet (\hyperref[graf:cinta_1]{Grafcet Máquina 1}), el cual su funcionamiento es 
generar un objeto enviarlo al final de la cinta, y si la siguiente máquina esta libre
entregárselo.



Las alarmas que tenemos en está máquina que la detendrían son:
\begin{itemize}
	\item Si cuando el objeto está de camino al sensor del final de la cinta, se pone otro objeto al principio de la cinta.
	\item Si el objeto desde que se ha generado hasta el final de la cinta tarda más de 8 segundos.
\end{itemize}


\begin{figure}[H]
	\centering
	\scalebox{0.75}{%\nodeDist = 2.5cm
%\retornoDist = 0.7\nodeDist % Distància vertical dels retorns
%\bifDistX = 10ex
%\bifDistY = 1.1\nodeDist
%\sincDistYabove = 0.85\nodeDist
%\sincDistYbelow = 0.6\sincDistYabove
%\sincDistYBlockbelow = 0.5\nodeDist


\begin{tikzpicture}[auto]
	
	\ttfamily
	
	% --------------------------------------------
	\etapaInicial{10}
	\primeraAccion{10}{X10-A1}{CINTA\_LIBRE}	

	%-----
	\renewcommand{\transitionPos}{0.35}
   	\etapa[1.25\nodeDist]{11}{10}{Proceso\_Activo};
	\primeraAccion{11}{X11-A1}{Emisor}
	\otraAccion{X11-A1}{X11-A2}{QC1\_A}
	\otraAccion{X11-A2}{X11-A3}{QC1\_B}
	

	%---
	\etapa[1.25\nodeDist]{11b}{11}{SPress\_X};
	\primeraAccion{11b}{X11-A1}{QC1\_B}

	\etapa[1.25\nodeDist]{12}{11b}{\NOT{SPress\_X}};
	\primeraAccion{12}{X12-A1}{QC1\_A}
	\otraAccion{X12-A1}{X12-A2}{QC1\_B}
	


	%---
	%\etapa[1.25\nodeDist]{13}{12}{S\_VAC\_A $\cdot$ S\_PRESENCIA};
	

	%---
	\etapa[1.25\nodeDist]{13}{12}{S1};
	

	%---
	%\etapa[1.25\nodeDist]{14}{12}{\NOT(S\_VAC\_A) $\cdot$ S\_PRESENCIA};

	%----
	\etapa[1.25\nodeDist]{14}{13}{Descartador\_Libre};
	\primeraAccion{14}{X14-A1}{QC1\_B}
	\otraAccion{X14-A1}{X14-A2}{Act\_Descartador\_B}

	\etapa[1.25\nodeDist]{15}{14}{\NOT{S1}};


	%----
	\retornoInicio[6em]{15}{0}{Pieza\_Rec}
	


	

            
\end{tikzpicture}}
	\medskip
	\label{graf:cinta_1}
	\caption{Grafcet de la Máquina 1}
\end{figure}



\subsection{Grafcet Máquina 2}
Como vemos en la siguiente figura (\hyperref[fig:maq2]{Máquina 2}), esta máquina esta compuesta por:
\begin{itemize}
	\item 1 Tranfer, que puede desplazar hacia el lado, o ir recto.
	\item 1 Sensor de visión, o 3 sensores en la parte del embalaje.
\end{itemize}


\begin{figure}[H]
	\centering
	\includegraphics[width=0.3\linewidth]{./Figuras/maq2.png}
	\medskip
	\label{fig:maq2}
	\caption{Máquina 2 de la planta}
\end{figure}

Como vemos en el grafcet (\hyperref[graf:Identificador_Tipo]{Grafcet Máquina 2}), en esta máquina cuando nos entrega la pieza la máquina anterior, determinamos
3 caminos:
\begin{itemize}
	\item Determinamos si la pieza es del tipo A bueno, que nos esperamos a que 
	la siguiente máquina esté disponible.
	\item Determinamos si la pieza es del tipo A malo, que seguimos a la espera de
	que la siguiente máquina siga disponible, pero le avisaremos de que la pieza 
	que está de camino es mala .
	\item Determinamos si es del tipo B, que lo que hacemos es descartarla con
	un movimiento lateral del transfer.
\end{itemize}

Respecto a los 3 sensores en la máquina de embalaje, está hecho por la siguiente razón:
\begin{itemize}
	\item Si se detectan los 3 sensores al final del transfer, es el tipo A bueno.
	\item Si al final de la cinta sólo se detectan los dos sensores de abajo 
	es el tipo A malo.
	\item Si solo se detectan los dos sensores finales, que están uno encima del otro
	determinamos que es tipo B y se debe de descartar.
\end{itemize}

En esta máquina sólo se trata una alarma, que es que tardemos más de 5 segundos en recibir la pieza de la
máquina anterior.

\begin{figure}[H]
	\centering
	\includegraphics[width=0.8\linewidth]{./Figuras/grafcet_20.png}
	\medskip
	\label{graf:Identificador_Tipo}
	\caption{Grafcet de la Máquina 2}
\end{figure}







\subsection{Grafcet máquina 3}

Como vemos en la siguiente figura (\ref{}), esta máquina esta compuesta por:
\begin{itemize}
	\item 1 cinta.
	\item 1 Sensor al final de la cinta.
\end{itemize}

\begin{figure}[H]
	\centering
	\includegraphics[width=0.8\linewidth]{./Figuras/maq3.png}
	\medskip
	\label{fig:maq2}
	\caption{Maquina 2 de la planta}
\end{figure}

Como vemos en el grafcet (REF), en esta máquina cuando nos entrega la pieza la máquina anterior,esta
solo será del tipo A, pero nos idicará si es buena o mala, para poder indicarle
a la máquina siguietne si la tiene que descartar o no.

El funcionamiento es recibir la pieza indicando si es buena o mala, 
llevarla hasta el final donde estará el sensor y si la siguiente máquina
esta libre entregarla.

En está máquina solo se trata una alarma, que es que tardemos más de 10 segundos
en recorrer toda la cinta

\begin{figure}[H]
	\centering
	\scalebox{0.75}{%\nodeDist = 2.5cm
%\retornoDist = 0.7\nodeDist % Distància vertical dels retorns
%\bifDistX = 10ex
%\bifDistY = 1.1\nodeDist
%\sincDistYabove = 0.85\nodeDist
%\sincDistYbelow = 0.6\sincDistYabove
%\sincDistYBlockbelow = 0.5\nodeDist


\begin{tikzpicture}[auto]
	
	\ttfamily
	
% --------------------------------------------
	\etapaInicial{30}
	\primeraAccion{30}{X30-A1}{MAQ\_Libre}
	%---

	\bifurcacion[100]{30}{31}{Act\_Maq\_A\_Buena}{32}{Act\_Maq\_A\_Mala}

	\primeraAccion{31}{X31-A1}{QC3}
	\primeraAccion{32}{X32-A1}{QC3}
	
	%---
	
	%---
	\etapa[1.25\nodeDist]{31b}{31}{S3};

	\etapa[1.25\nodeDist]{31c}{31b}{MAQ\_SIG\_LIBRE};
	\primeraAccion{31c}{X31c-A1}{QC3}
	\otraAccion{X31c-A1}{X31c-A2}{SigMaqBuena}


	\etapa[1.25\nodeDist]{32b}{32}{S3};

	\etapa[1.25\nodeDist]{32c}{32b}{MAQ\_SIG\_LIBRE};
	\primeraAccion{32c}{X32c-A1}{QC3}
	\otraAccion{X32c-A1}{X32c-A2}{SigMaqMala}
	
	\retornoInicio[6em]{31c}{0}{\NOT{S3}}
	\retornoInicio[-16em]{32c}{0}{\NOT{S3}}

	








	




	

            
\end{tikzpicture}}
	\medskip
	\label{graf:fresadora}
	\caption{Grafcet de la máquina 3}
\end{figure}


\subsection{Grafcet máquina 4}
Como vemos en la siguiente figura (\ref{}), esta máquina esta compuesta por:
\begin{itemize}
	\item 1 transfer, que puede mover hacia un lado o ir en linea recta
	\item 1 Sensor al final de la cinta.
\end{itemize}

\begin{figure}[H]
	\centering
	\includegraphics[width=0.8\linewidth]{./Figuras/maq4.png}
	\medskip
	\label{fig:maq2}
	\caption{Maquina 2 de la planta}
\end{figure}

Como vemos en el grafcet (REF), en esta máquina cuando nos entrega la pieza la máquina anterior,esta
vendrá indicada si es buena o mala, por lo que al llegar al sensor si es buena se esperará
a que la siguiente máquina este libre, y si es mala, la descartaremos


En está máquina solo se trata una alarma, que es que tardemos más de 5 segundos
en llegar al sensor desde que nos indican que nos activemos


\begin{figure}[H]
	\centering
	\scalebox{0.75}{%\nodeDist = 2.5cm
%\retornoDist = 0.7\nodeDist % Distància vertical dels retorns
%\bifDistX = 10ex
%\bifDistY = 1.1\nodeDist
%\sincDistYabove = 0.85\nodeDist
%\sincDistYbelow = 0.6\sincDistYabove
%\sincDistYBlockbelow = 0.5\nodeDist


\begin{tikzpicture}[auto]
	
	\ttfamily
	
	\etapaInicial{40}
	\primeraAccion{40}{X40-A1}{Descartador\_Libre}
	%---

	\bifurcacion[100]{40}{41}{Act\_Descartador}{42}{SigMaqMala}

	\primeraAccion{41}{X41-A1}{QC4}
	\primeraAccion{42}{X42-A1}{QC4}
	
	%---
	
	%---
	\etapa[1.25\nodeDist]{41b}{41}{S4};
	\comentari[3]{X31b}{Esperamos máquina de calidad libre}

	\etapa[1.25\nodeDist]{41c}{41b}{SIG\_MAQ\_LIBRE};
	\primeraAccion{41c}{X41c-A1}{QC4}
	\otraAccion{X41c-A1}{X41c-A2}{ACT\_SIG\_MAQ}


	\etapa[1.25\nodeDist]{42b}{42}{S4};
	\primeraAccion{42b}{X42b-A1}{QTRANS}

	
	
	\retornoInicio[6em]{41c}{0}{\NOT{S4}}
	\retornoInicio[-16em]{42b}{0}{\NOT{S4}}











	




	

            
\end{tikzpicture}}
	\medskip
	\label{graf:fresadora}
	\caption{Grafcet de Transfer 2 Cajas}
\end{figure}


\subsection{Grafcet Posicionamiento Vacuna}
\begin{figure}[H]
	\centering
	\scalebox{0.75}{%\nodeDist = 2.5cm
%\retornoDist = 0.7\nodeDist % Distància vertical dels retorns
%\bifDistX = 10ex
%\bifDistY = 1.1\nodeDist
%\sincDistYabove = 0.85\nodeDist
%\sincDistYbelow = 0.6\sincDistYabove
%\sincDistYBlockbelow = 0.5\nodeDist


\begin{tikzpicture}[auto]
	
	\ttfamily
	
	\etapaInicial{70}
	\primeraAccion{70}{X70-A1}{MAQ\_CAL\_VAC\_LIBRE}
	%---
	\etapa[1.25\nodeDist]{71}{70}{Act\_Maq};
	\primeraAccion{71}{X71-A1}{QC5\_VAC}
	\otraAccion{X71-A1}{X71-A2}{QC6\_VAC}
	\otraAccion{X71-A2}{X71-A3}{QC7\_VAC}
	
	%--
	\etapa[1.25\nodeDist]{72}{71}{S5};
	\primeraAccion{72}{X72-A1}{QC5\_VAC}
	\otraAccion{X72-A1}{X72-A2}{QC6\_VAC}
	\otraAccion{X72-A2}{X72-A3}{QC7\_VAC}
	%---
	\etapa[1.25\nodeDist]{73}{72}{1s/X72};
	\primeraAccion{73}{X73-A1}{QCLAMP}

	\etapa[1.25\nodeDist]{74}{73}{1s/X73};
	\primeraAccion{74}{X74-A1}{HAYSoporte}

	\etapa[1.25\nodeDist]{75}{74}{Pieza\_Empaquetada};

	\etapa[1.25\nodeDist]{76}{75}{SIG\_MAQ\_LIBRE};
	\primeraAccion{76}{X76-A1}{QCLAMP\_SUB}
	\otraAccion{X76-A1}{X76-A2}{QC7\_VAC}
	\otraAccion{X76-A2}{X76-A3}{Act\_SIG\_MAQ}

	\retornoInicio[6em]{76}{0}{\underline{1}}

\end{tikzpicture}}
	\medskip
	\label{graf:fresadora}
	\caption{Grafcet en el que se posiciona la vacuna}
\end{figure}
\subsection{Grafcet Empaquetado}
\begin{figure}[H]
	\centering
	\scalebox{0.75}{%\nodeDist = 2.5cm
%\retornoDist = 0.7\nodeDist % Distància vertical dels retorns
%\bifDistX = 10ex
%\bifDistY = 1.1\nodeDist
%\sincDistYabove = 0.85\nodeDist
%\sincDistYbelow = 0.6\sincDistYabove
%\sincDistYBlockbelow = 0.5\nodeDist


\begin{tikzpicture}[auto]
	
	\ttfamily
	
	\etapaInicial{50}
	\primeraAccion{50}{X50-A1}{MAQ\_CAL\_VAC\_LIBRE}

	\etapa[1\nodeDist]{51}{50}{Act\_Maq};
	\primeraAccion{51}{X51-A1}{QC5\_VAC}
	\otraAccion{X51-A1}{X51-A2}{QC6\_VAC}
	\otraAccion{X51-A2}{X51-A3}{QC7\_VAC}

	%---

	%---
	\etapa[1\nodeDist]{52}{51}{S5};
	\primeraAccion{52}{X52-A1}{QC5\_VAC}
	\otraAccion{X52-A1}{X52-A2}{QC6\_VAC}
	\otraAccion{X52-A2}{X52-A3}{QC7\_VAC}
	

	%---
	\etapa[1\nodeDist]{53}{52}{1s/X52};
	\primeraAccion{53}{X53-A1}{QCLAMP}

	\etapa[1\nodeDist]{54}{53}{1s/X53};
	\primeraAccion{54}{X54-A1}{HAYVacuna}
	\otraAccion{X54-A1}{X54-A2}{QCLAMP}

	\etapa[1\nodeDist]{55}{54}{Hay\_Soporte};
	\primeraAccion{55}{X55-A1}{QBAJ}
	\otraAccion{X55-A1}{X55-A2}{QCLAMP}

	\etapa[1\nodeDist]{56}{55}{2s/X55};
	\primeraAccion{56}{X56-A1}{QBAJ}
	\otraAccion{X56-A1}{X56-A2}{GRIPPER}
	\otraAccion{X56-A2}{X56-A3}{QCLAMP}

	\etapa[1\nodeDist]{57}{56}{2s/X56};
	\primeraAccion{57}{X57-A1}{GRIPPER}
	\otraAccion{X57-A1}{X57-A2}{QCLAMP}

	\etapa[1\nodeDist]{58}{57}{2s/X57};
	\primeraAccion{58}{X58-A1}{GRIPPER}
	\otraAccion{X58-A1}{X58-A2}{QHOR}

	\etapa[1\nodeDist]{59}{58}{2s/X58};
	\primeraAccion{59}{X59-A1}{GRIPPER}
	\otraAccion{X59-A1}{X59-A2}{QBAJ\_MAQ}
	\otraAccion{X59-A2}{X56-A3}{QHOR}

	\etapa[1\nodeDist]{60}{59}{2s/X59};
	\primeraAccion{60}{X60-A1}{QHOR}
	\otraAccion{X60-A1}{X60-A2}{QBAJ\_MAQ}

	\etapa[1\nodeDist]{61}{60}{2s/X60};
	\primeraAccion{61}{X61-A1}{QHOR}

	\etapa[1\nodeDist]{62}{61}{2s/X61};
	\primeraAccion{62}{X62-A1}{Pieza\_Empaquetada}

	
	

	\retornoInicio[13em]{62}{0}{\underline{1}}
	













	




	

            
\end{tikzpicture}}
	\medskip
	\label{graf:Empaquetado}
	\caption{Grafcet de la fresadora}
\end{figure}



\subsection{Grafcet Máquina 5 embalaje}\label{sec:maq5_emb}
Esta máquina (\hyperref[graf:maq5_emb]{Máquina 5 Embalaje}) tiene un funcionamiento similar al indicado en la  \hyperref[sec:maq3]{Subsección de la  Máquina 3}, pero
con un único camino, ya que su única función es llevar una pieza buena al final de la cinta, 
para entregársela al transfer siguiente.
\begin{figure}[H]
	\centering
	\scalebox{0.75}{%\nodeDist = 2.5cm
%\retornoDist = 0.7\nodeDist % Distància vertical dels retorns
%\bifDistX = 10ex
%\bifDistY = 1.1\nodeDist
%\sincDistYabove = 0.85\nodeDist
%\sincDistYbelow = 0.6\sincDistYabove
%\sincDistYBlockbelow = 0.5\nodeDist


\begin{tikzpicture}[auto]
	
	\ttfamily
	
	\etapaInicial{80}
	\primeraAccion{80}{X80-A1}{Descartador\_Libre}
	%---
	\etapa[1.25\nodeDist]{81}{80}{Act\_Maq\_A\_Buena};
	\primeraAccion{81}{X81-A1}{QC5\_VAC}
	
	%--
	\etapa[1.25\nodeDist]{82}{81}{S5};
	\comentari[3]{X82}{Esperamos máquina de calidad libre}

	%---
	\etapa[1.25\nodeDist]{83}{82}{SIG\_MAQ\_LIBRE};
	\primeraAccion{83}{X83-A1}{QC3}
	\otraAccion{X83-A1}{X83-A2}{Act\_SIG\_MAQ}
	




	\retornoInicio[6em]{83}{0}{\NOT{S5}}

\end{tikzpicture}}
	\medskip
	\label{graf:maq5_emb}
	\caption{Grafcet de la Máquina 5 Embalaje}
\end{figure}

Este grafcet está descompuesto en las siguientes ecuaciones algebráicas:\\

\setlength{\sepsGrups}{2ex}

% PRODUCCIÓN

% Transiciones (sin FORCE_CURRENT)
$S_{80} = \mathtt{FirstScan} + X_{83}\cdot \Not{\mathtt{S5}}$\\
$S_{81} = X_{80}\cdot \mathtt{Act\_Proceso}$\\
$S_{82} = X_{81}\cdot \mathtt{S5}$\\
$S_{83} = X_{82}\cdot \mathtt{SigMaqLibre}$\\[\sepsGrups]

% Etapas (sin FORCE_INIT)
$X_{80} = S_{80} + X_{80}\cdot \Not{S_{81}}$\\
$X_{81} = S_{81} + X_{81}\cdot \Not{S_{82}}$\\
$X_{82} = S_{82} + X_{82}\cdot \Not{S_{83}}$\\
$X_{83} = S_{83} + X_{83}\cdot \Not{S_{80}}$\\[\sepsGrups]

% Marcas / salidas
$\mathtt{Maq\_Libre}    = X_{80}\cdot \Not{\mathtt{STOP\_ACT}}$\\
$\mathtt{QC3}          = (X_{81}+X_{83}+\mathtt{Mantenimiento\_Cinta}+\mathtt{vaciado\_cinta})\cdot \Not{\mathtt{STOP\_ACT}}$\\
$\mathtt{Act\_Sig\_Maq} = X_{83}\cdot \Not{\mathtt{STOP\_ACT}}$\\


\subsection{Grafcet Máquina 6 embalaje}
En esta máquina (\hyperref[graf:maq6_emb]{Máquina 6 Embalaje}) la función es desplazar el embalaje a la cinta final del proceso,
que se detalla más adelante.

\begin{figure}[H]
	\centering
	\includegraphics[width=0.3\linewidth]{./Figuras/maq6_EMB.png}
	\medskip
	\label{fig:maq6_emb}
	\caption{Máquina 6 de la línea de embalaje}
\end{figure}

\begin{figure}[H]
	\centering
	\scalebox{0.75}{%\nodeDist = 2.5cm
%\retornoDist = 0.7\nodeDist % Distància vertical dels retorns
%\bifDistX = 10ex
%\bifDistY = 1.1\nodeDist
%\sincDistYabove = 0.85\nodeDist
%\sincDistYbelow = 0.6\sincDistYabove
%\sincDistYBlockbelow = 0.5\nodeDist


\begin{tikzpicture}[auto]
	
	\ttfamily
	
% --------------------------------------------
	\ttfamily
	
	\etapaInicial{90}
	\primeraAccion{90}{X90-A1}{MAQ\_Libre}
	
	
	%---
	\etapa[1.25\nodeDist]{91}{90}{Act\_MAQ};
	\primeraAccion{91}{X91-A1}{QCINTA}
	
	%--
	\etapa[1.25\nodeDist]{92}{91}{SENSOR};

	%---
	\etapa[1.25\nodeDist]{93}{92}{SIG\_MAQ\_LIBRE};
	\primeraAccion{93}{X93-A1}{QTRANS}

	\etapa[1.25\nodeDist]{94}{93}{\NOT{SENSOR}};
	\primeraAccion{94}{X94-A1}{Act\_SIG\_MAQ}
	
	




	\retornoInicio[6em]{94}{0}{\underline{1}}
	
       
\end{tikzpicture}}
	\medskip
	\label{graf:maq6_emb}
	\caption{Grafcet de la Máquina 6 Embalaje}
\end{figure}

Este grafcet está descompuesto en las siguientes ecuaciones algebráicas:\\

\setlength{\sepsGrups}{2ex}

% PRODUCCIÓN

% Transiciones (sin FORCE_CURRENT)
$S_{90} = \mathtt{FirstScan} + X_{94}$\\
$S_{91} = X_{90}\cdot \mathtt{Act\_Proceso}$\\
$S_{92} = X_{91}\cdot \mathtt{SENSOR}$\\
$S_{93} = X_{92}\cdot \mathtt{SigMaqLibre}$\\
$S_{94} = X_{93}\cdot \Not{\mathtt{SENSOR}}$\\[\sepsGrups]

% Etapas (sin FORCE_INIT)
$X_{90} = S_{90} + X_{90}\cdot \Not{S_{91}}$\\
$X_{91} = S_{91} + X_{91}\cdot \Not{S_{92}}$\\
$X_{92} = S_{92} + X_{92}\cdot \Not{S_{93}}$\\
$X_{93} = S_{93} + X_{93}\cdot \Not{S_{94}}$\\
$X_{94} = S_{94} + X_{94}\cdot \Not{S_{90}}$\\[\sepsGrups]

% Marcas / salidas
$\mathtt{Maq\_Libre}    = X_{90}\cdot \Not{\mathtt{STOP\_ACT}}$\\
$\mathtt{QC3}          = (X_{91}+\mathtt{Mantenimiento\_Cinta}+\mathtt{vaicado\_cinta})\cdot \Not{\mathtt{STOP\_ACT}}$\\
$\mathtt{QTrans}       = (X_{93}+\mathtt{Mantenimiento\_Trans}+\mathtt{vaciado\_trnas})\cdot \Not{\mathtt{STOP\_ACT}}$\\
$\mathtt{Act\_Sig\_Maq} = X_{94}\cdot \Not{\mathtt{STOP\_ACT}}$\\

\subsection{Grafcet Máquina 7 embalaje}
Esta máquina (\hyperref[graf:maq7]{Máquina 7, embalaje}) es similar a la máquina 5 de la línea de soportes (\hyperref[sec:map5_sop]{Sección de la Máquina 5 de Soportes}). Se cuenta el número de piezas embaladas para, al alcanzar el cupo, levantar el clamper y mover la caja a otra zona.
 
\begin{figure}[H]
	\centering
	\includegraphics[width=0.8\linewidth]{./Figuras/maq7_EMB.png}
	\medskip
	\label{fig:maq7_emb}
	\caption{Máquina 7 de la línea de embalaje}
\end{figure}

\begin{figure}[H]
	\centering
	\scalebox{0.75}{%\nodeDist = 2.5cm
%\retornoDist = 0.7\nodeDist % Distància vertical dels retorns
%\bifDistX = 10ex
%\bifDistY = 1.1\nodeDist
%\sincDistYabove = 0.85\nodeDist
%\sincDistYbelow = 0.6\sincDistYabove
%\sincDistYBlockbelow = 0.5\nodeDist


\begin{tikzpicture}[auto]
	
	\ttfamily
	
% --------------------------------------------
	\ttfamily
	
	\etapaInicial{100}
	\primeraAccion{100}{X100-A1}{MAQ\_LIBRE}
    \otraAccion{X100-A1}{X100-A2}{$\mathtt{C:=0}$}
  \sobreActivacion{X100-A2}{X100-A3}{}
	%---

	\etapa[1.25\nodeDist]{101}{100}{Act\_Maq};
	\primeraAccion{101}{X101-A1}{QC7}
	
	%--
	\etapa[1.25\nodeDist]{102}{101}{S5};
	\primeraAccion{102}{X102-A1}{QC7}

	%---
	\etapa[1.25\nodeDist]{103}{102}{1s/X102};
	\primeraAccion{103}{X103-A1}{QCLAMP}

	\etapa[1.25\nodeDist]{104}{103}{1s/X103};
	\primeraAccion{104}{X104-A1}{HAYSPieza}

  \etapa[1.25\nodeDist]{105}{104}{Pieza\_Empaquetada};
  \primeraAccion{105}{X105-A1}{$\mathtt{C:=C+1}$}
  \sobreActivacion{X105-A1}{X105-A2}{}

  %\retornoInicio[-10em]{105}{104}{$\mathtt{C<2}$}
  %\etapa[1.25\nodeDist]{104}{105}{$\mathtt{C<2}$}
  \saltoHorizontal[5.5cm]{105}{104}{$\mathtt{C<2}$}
  
  \etapa[1.4\nodeDist]{106}{105}{$\mathtt{C==1}$};
	\primeraAccion{106}{X106-A1}{QCLAMP\_SUB}
	\otraAccion{X106-A1}{X106-A2}{QC7}

  \etapa[1.25\nodeDist]{107}{106}{3s/X106};
	\primeraAccion{107}{X107-A1}{Pieza\_Entregada}
	
	

	\retornoInicio[6em]{107}{0}{\underline{1}}
	
       
\end{tikzpicture}}
	\medskip
	\caption{Grafcet de la Máquina 7 embalaje}
	\label{graf:maq7}
\end{figure}

\subsection{Grafcet de Mantenimiento}
El \hyperref[graf:mantenimiento]{Grafcet de Mantenimiento}
 asegura que la máquina funciona correctamente antes de procesar
 lotes. Se ejecuta al arrancar la máquina y después de la ejecución
  del \hyperref[graf:principal_vaciado]{Grafcet Principal de Vaciado},
   garantizando que la puesta en marcha tras soltar la seta sea
    correcta.

	Dispone tamibién de un botón en la pantalla de configuración del HMI
	(\ref{fig:HMI_config}) para que un 
	usuario con permisos pueda activarlo
	en cualquier momento.

\begin{figure}[H]
	\centering
	\scalebox{0.75}{%\nodeDist = 2.5cm
%\retornoDist = 0.7\nodeDist % Distància vertical dels retorns
%\bifDistX = 10ex
%\bifDistY = 1.1\nodeDist
%\sincDistYabove = 0.85\nodeDist
%\sincDistYbelow = 0.6\sincDistYabove
%\sincDistYBlockbelow = 0.5\nodeDist


\begin{tikzpicture}[auto]
	
	\ttfamily
	
	\etapaInicial{110}
	
	%---
	\etapa[1.25\nodeDist]{111}{110}{Act\_Mantenimiento};

	
	%--
	\etapa[1.25\nodeDist]{112}{111}{\NOT{SENSORES}};
	\primeraAccion{112}{X112-A1}{Activar\_Cinta}
	
	%---
	\etapa[1.25\nodeDist]{113}{112}{2s/X112};
	\primeraAccion{113}{X113-A1}{Activar\_Transfers}

	\etapa[1.25\nodeDist]{114}{113}{2s/X113};
	\primeraAccion{114}{X114-A1}{Bajar\_Robot}
  \otraAccion{X114-A1}{X114-A2}{Subir\_CLAMPER}

	\etapa[1.25\nodeDist]{115}{114}{2s/X114};
	\primeraAccion{115}{X115-A1}{Grippers}
  \otraAccion{X115-A1}{X115-A2}{CLAMPER}

	\etapa[1.25\nodeDist]{116}{115}{2s/X115};
	\primeraAccion{116}{X116-A1}{Mover\_Horizontal\_Robots}

  \etapa[1.25\nodeDist]{117}{116}{2s/X116};
	\primeraAccion{117}{X117-A1}{Fin\_Mantenimiento}

	\retornoInicio[6em]{117}{0}{\underline{1}}

\end{tikzpicture}
}
	\medskip
	\caption{Grafcet de Mantenimiento}
	\label{graf:mantenimiento}
\end{figure}

\subsection{Grafcet Principal de Vaciado}
El \hyperref[graf:principal_vaciado]{Grafcet Principal de Vaciado}
 se activa tras desenclavar la seta de emergencia, 
 ya que no se aceptan las piezas que
  estuvieran en la máquina durante la emergencia.
   Las máquinas también se pueden vaciar 
   si un usuario con permisos pulsa el botón
    de vaciado del HMI (\ref{fig:HMI_config}).

\begin{figure}[H]
	\centering
	\scalebox{0.75}{\begin{tikzpicture}
  \ttfamily

  %---------------------------
  % Columna izquierda
  %---------------------------
  \etapaInicial{120}
  \etapa[1.25\nodeDist]{121}{120}{VACIAR\_MAQS};
  \primeraAccion{121}{X121-A1}{Activar\_Vaciado\_Maquinas}
  
 
  
  
  \etapa[1.25\nodeDist]{122}{121}{Fin\_VAC\_VAC\_1 $\cdot$ Fin\_VAC\_EMB\_1 $\cdot$ Fin\_VAC\_SOP\_1 $\cdot$ Fin\_VAC\_VAC\_2 $\cdot$ Fin\_VAC\_EMB\_2 $\cdot$ Fin\_VAC\_SOP\_2 };
 	\primeraAccion{122}{X122-A1}{Fin\_Vaciado}
  
    \etapa[1.25\nodeDist]{122V2}{122}{ $\cdot$ Fin\_VAC\_VAC\_3 $\cdot$ Fin\_VAC\_EMB\_3 $\cdot$ Fin\_VAC\_SOP\_3 $\cdot$ Fin\_VAC\_SOP\_4};
 	\primeraAccion{122V2}{X122V2-A1}{Fin\_Vaciado}

  \retornoInicio[6cm]{122V2}{120}{\underline{1}}






  


\end{tikzpicture}
}
	\medskip
	\caption{Grafcet Principal de Vaciado}
	\label{graf:principal_vaciado}
\end{figure}

\subsection{Grafcet vaciado tipo 1}
 \subsubsection{Grafcet Vaciado 1}
 
 El \hyperref[graf:vaciado_1]{Grafcet de Vaciado 1} se 
 activa por medio del \hyperref[graf:principal_vaciado]{Grafcet Principal de Vaciado}, 
 y este vacía las siguientes máquinas:
\begin{itemize}
	\item Máquina 1:
	\begin{itemize}
		\item \hyperref[fig:maq1]{Vacunas, Soportes y Embalajes}
	\end{itemize}
	\item Máquina 2:
	\begin{itemize}
		\item \hyperref[fig:maq2]{Vacunas, Soportes y Embalajes}
	\end{itemize}
	\item Máquina 3:
	\begin{itemize}
		\item \hyperref[fig:maq3]{Vacunas, Soportes y Embalajes}
	\end{itemize}
	\item Máquina 4:
	\begin{itemize}
		\item \hyperref[fig:maq4]{Vacunas, Soportes y Embalajes}
	\end{itemize}
\end{itemize}

\begin{figure}[H]
	\centering
	\scalebox{0.75}{%\nodeDist = 2.5cm
%\retornoDist = 0.7\nodeDist % Distància vertical dels retorns
%\bifDistX = 10ex
%\bifDistY = 1.1\nodeDist
%\sincDistYabove = 0.85\nodeDist
%\sincDistYbelow = 0.6\sincDistYabove
%\sincDistYBlockbelow = 0.5\nodeDist


\begin{tikzpicture}[auto]
	
	\ttfamily
	
	\etapaInicial{125}
	
	%---
	\etapa[1.25\nodeDist]{126}{125}{MAQ\_VAC\_VAC\_1};
	\primeraAccion{126}{X126-A1}{QC1\_VAC}
    \otraAccion{X126-A1}{X126-A2}{QC2\_VAC}

	\bifurcacion[80]{126}{127}{10s/X1}{126a}{\NOT{S2}}

	

	\primeraAccion{127}{X127-A1}{Fin\_VAC\_VAC\_1}
	
	%--
	
	\primeraAccion{126a}{X126a-A1}{QC2\_VAC}
	
	%---
	\etapa[1.25\nodeDist]{126b}{126a}{S3};
	\primeraAccion{126b}{X126b-A1}{QTransfer1\_VAC}
	
	\saltoHorizontal[10cm]{126}{126b}{S3}


	\retornoInicio[-19em]{126b}{0}{\NOT{S3}}
	\retornoInicio[6em]{127}{0}{Fin\_Vaciado}

\end{tikzpicture}}
	\medskip
	\caption{Grafcet de vaciado tipo 1}
	\label{graf:vaciado_1}
\end{figure}

\subsection{Grafcet vaciado tipo 2}
 \subsubsection{Grafcet Vaciado 2}
 El \hyperref[graf:vaciado_2]{Grafcet de Vaciado 2} se activa por medio del
  \hyperref[graf:principal_vaciado]{Grafcet Principal de Vaciado}, y este vacía
 las siguientes máquinas:
\begin{itemize}
	\item Máquina 5 de Soportes:
	\begin{itemize}
		\item \hyperref[fig:maq5]{Soporte}
	\end{itemize}
	\item Máquina 5 de vacunas y 6 de soportes:
	\begin{itemize}
		\item \hyperref[fig:maq5_vac_6_sop]{Vacunas y Soportes}
	\end{itemize}
	\item Máquina 6 y 7 de embalaje:
	\begin{itemize}
		\item \hyperref[graf:maq6]{Embalajes}
		\item \hyperref[graf:maq7]{Embalajes}
	\end{itemize}
\end{itemize}

\begin{figure}[H]
	\centering
	\scalebox{0.75}{\begin{tikzpicture}
  \ttfamily

  %---------------------------
  % Columna izquierda
  %---------------------------
  \etapaInicial{130}
  \etapa[1.25\nodeDist]{131}{130}{MAQ\_VAC\_VAC\_3};
  \primeraAccion{131}{X131-A1}{MAQ\_X}
  
 
  
  
    \etapa[1.25\nodeDist]{132}{131}{7s/X1};
 	\primeraAccion{132}{X132-A1}{Fin\_VAC\_VAC\_3}
  
    

  \retornoInicio[6cm]{132}{130}{Fin\_Vaciado}



\end{tikzpicture}}
	\medskip
	\caption{Grafcet de vaciado tipo 2}
	\label{graf:vaciado_2}
\end{figure}


\subsection{Gestión de las alaras}
El \hyperref[graf:emerg2]{Grafcet de alarmas en guía GEMMA} gestiona las alarmas de cada máquina,
asegurandose de que la máquina se detenga cuando haya una alarma activa y 
permitiendo resetear la alarma cuando se haya solucionado el problema, o reanudarlo 
desde donde se havia parado si el problema era leve.

Este grafcet se encuentra implementado en todas las máquinas de la línea de producción, 
siguiendo la guia GEMMA para la estructura del código.
\begin{figure}[H]
	\centering
	\scalebox{0.75}{\input{./grafcets/gestion_alarma.tex}}
	\medskip
	\caption{Grafcet de alarmas en guía GEMMA}
	\label{graf:emerg2}
\end{figure}

\subsection{Gestión de la seta de emergéncia}

El \hyperref[graf:emerg_gem]{Grafcet de la seta de emergéncia } ,gestiona la señal que produce la
seta de emergéncia, y todos los pasos que debe de seguir la máquina cuando se desenclava la seta 
para poder rearmar la máquina y seguir con el proceso de producción.

Lo primero que hacemos es activar la variable ''EMERGENCIA'' comunicando a todos 
las partes del programa que la seta se ha enclavado, y por lo tanto la máquina
 debe de detenerse. Lo siguiente es desenclavar la seta, y una vez desenclavada,
 pular ''reset'' para activar el vaciado de la máquina, ya que no se aceptan las piezas
  que estuvieran en la máquina durante la emergéncia. Una vez finalizado el vaciado,
   se activa el mantenimiento para comprobar que todo funciona correctamente, dejando 
   la máquina lista para volver a pulsar marcha o entrar a modo manual.


\begin{figure}[H]
	\centering
	\scalebox{0.75}{\begin{tikzpicture}
  \ttfamily

  %---------------------------
  % Columna izquierda
  %---------------------------
  \etapaInicial{60e}

  % 100 -> 101  (transición de alarmas)
  \etapa{61e}{60e}{SETA};
  \primeraAccion{61e}{X61e-A1}{EMERGENCIA}
  \etapa{62e}{61e}{\NOT{SETA}$*$Reset};
  \primeraAccion{62e}{X62e-A1}{Activar\_Vaciado}
  \etapa{63e}{62e}{Fin\_Vaciado};
  \primeraAccion{63e}{X63e-A1}{Activar\_Mantenimiento}
  \retornoInicio[4.2cm]{63e}{60e}{Mantenimiento\_fin}
  

  


\end{tikzpicture}
}
	\medskip
	\caption{Grafcet de la gestión de la seta}
	\label{graf:emerg_gem}
\end{figure}




El \hyperref[graf:emerg_seta]{Grafcet de la seta de emergéncia en guía GEMMA}, se encuentra implementado en todas las máquinas de la línea de producción, 
siguiendo la guia GEMMA para la estructura del código. Aunque como hemos hablado con anterioridad,
las señales que le llegan a cada bloque de la máquina son que se ha enclavado la seta, y cuando
ya se han pasado todas la verificaciones para reanudar el proceso.

\begin{figure}[H]
	\centering
	\scalebox{0.75}{\input{./grafcets/11_SETA.tex}}
	\medskip
	\caption{Grafcet de la seta de emergéncia en guía GEMMA}
	\label{graf:emerg_seta}
\end{figure}

\section{HMI}
\label{sec:HMI}

\subsection{Modelo utilizado y conexión con el PLC}
Para la realización del proyecto se ha escogido un HMI TP1500 Comfort pro de 15"
conectada vía Profinet al PLC, para poder tener acceso a las variables.

\subsection{Pantalla principal}
\label{subsec:pantalla_principal}
En la siguiente figura podemos observar la pantalla principal, donde podemos observar distintos elementos.
 
 \begin{figure}[H]
     \centering
     \includegraphics[width=0.8\linewidth]{./Figuras/HMI1.png}
	\medskip
	\caption{Pantalla principal del HMI}
	\label{fig:HMI_principal}
 \end{figure}

Respecto a los botones tenemos:
\begin{itemize}
    \item \underline{Marcha} : Este botón hace la misma función que el botón del panel
    físico del Factory IO, pone en marcha la máquina si esta no esta en alarmas
    o en emergencia.
    \item \underline{Paro}: Al igual que la marcha, es un botón que esta duplicado Respecto
    al cuadro físico que se encarga de parar la producción de piezas.
    \item \underline{Rearme} : Este botón solo se encuentra en el HMI, y lo que hace es
    que cuando está la máquina en alarma, si vemos que con mover un poco la pieza 
    la máquina puede seguir, apretaremos este botón sin necesidad de apretar el ''RESET''.
    \item \underline{Configuración}: Para pulsar este botón se pedirán credenciales; este apartado solo 
    está disponible para administradores y da acceso a la pantalla de configuración (véase \autoref{fig:HMI_config}).
    \item \underline{Alarma}: Este botón abre la pantalla del registro de alarmas (véase \autoref{fig:HMI_alarmas}).
\end{itemize}

Aparte de los botones, en la parte superior derecha vemos que hay un contador de embalajes completados
que ya han acabado el proceso de producción.

En el centro de la pantalla vemos una foto de todo el proceso con luces en cada máquina 
que forman el proceso completo. Las luces tienen los siguientes estados:
\begin{itemize}
    \item \underline{Verde} : Máquina libre y preparada para recibir piezas.
    \item \underline{Amarillo} : Máquina funcionando en modo automático.
    \item \underline{Rojo} : Máquina en alarma.
    \item \underline{Rojo parpadeante amarillo} : Máquina en emergencia.
\end{itemize}
Aparte de los colores, si pulsamos sobre cualquier parte de la máquina nos llevará a una ventana
donde podremos ver qué actuador está en funcionamiento en ese momento, cuál está parado,
o el contador de piezas descartadas.

\subsection{Pantalla máquinas}
\label{subsec:pantalla_maquinas}
En la siguiente figura (\ref{fig:HMI_maquinas}) podemos ver como es una pantalla cuando sobre la principal pulsamos sobre una
máquina, vemos que podemos ver dos máquinas, unas luces encima de los actuadores que nos indican si
están funcionando o no, y un contador donde se puede ver el numero de piezas descartadas en esa máquina.

Vemos que también hay unos botones con los nombres de los actuadores, esto es porque si tenemos el modo
manual activado, estos botones se harán visibles y podremos accionar los actuadores desde aquí.
 \begin{figure}[H]
     \centering
     \includegraphics[width=0.8\linewidth]{./Figuras/HMI2.png}
	\medskip
	\caption{Pantalla de detalle de una máquina}
	\label{fig:HMI_maquinas}
 \end{figure}

\subsection{Pantalla configuración}
\label{subsec:pantalla_configuracion}
En la siguiente figura (\ref{fig:HMI_config}) podemos ver la pantalla de configuración a la cual sólo se puede acceder a través
de credenciales de administrador. Vemos que hay una columna central donde
podemos ver:
\begin{itemize}
    \item Un campo de entrada, donde pondremos el número de vacunas por cajas que queremos
    \item Un botón que nos activa el mantenimiento, por si queremos comprobar que todo funciona
    \item Un botón que activa el vaciado de la máquina, por si en cualquier momento queremos vaciarla
    sin necesidad de la emergencia.
\end{itemize}

También disponemos de un botón de cerrar sesión, por si ya hemos configurado la máquina y probado, 
y ya el próximo que la toque es un trabajador normal.

 \begin{figure}[H]
     \centering
     \includegraphics[width=0.8\linewidth]{./Figuras/HMI3.png}
	\medskip
	\caption{Pantalla de configuración}
	\label{fig:HMI_config}
 \end{figure}

\subsection{Usuarios}
Respecto a los usuarios, el HMI dispone de dos grupos:
\begin{itemize}
    \item \underline{Grupo administrador} --> Acceso al modo manual, y al modo de configuración de la máquina
    \item \underline{Grupo Usuario} --> Solo tiene acceso a ver el estado de la máquina, ponerla en marcha, pararla, reanudarla ,o acceder
    a los valores de los contadores de piezas
\end{itemize}

\subsection{Registro de alarmas}
\label{subsec:registro_alarmas}
 
 \begin{figure}[H]
     \centering
     \includegraphics[width=0.8\linewidth]{./Figuras/HMI_AL.png}
	\medskip
	\caption{Registro de alarmas}
	\label{fig:HMI_alarmas}
 \end{figure}





\section{Resultados}\label{sec:errores}
Aqui adjutno el enlace al vídeo donde se puede ver el funcionamiento del sistema 
implementado en el PLC y el HMI funcionando en el entorno Factory IO, se enseñan las siguientes
funcionalidades:
\begin{itemize}
    \item Funcionamiento normal del sistema.
    \item Modo manual.  
    \item Modo mantenimiento.
    \item Gestión de alarmas.
    \item Gestión de emergencia.
    \item Vaciado del sistema.
\end{itemize}



\href{https://www.youtube.com/watch?v=CcwYSMDDBHw}{ENLACE VÍDEO}
%
\section{Errores y soluciones}\label{sec:errores}


\section{Conclusiones y Trabajo a futuro}\label{sec:conclusiones_trabajo_futuro}

\subsection{Conclusiones}
En el presente trabajo se ha abordado con éxito el diseño y la automatización de una línea de producción completa destinada a la clasificación, ensamblaje y embalaje de vacunas y soportes. La validación del sistema mediante el entorno virtual Factory IO y su programación en TIA Portal ha permitido verificar el funcionamiento coordinado de las tres líneas que componen la planta, demostrando que la lógica de control propuesta cumple con los requisitos funcionales y de seguridad establecidos.

La estructura modular adoptada, basada en la división por líneas y el uso de Grafcets independientes para cada máquina, ha resultado clave para lograr una programación ordenada y escalable. Esta organización no solo facilitó la detección de errores durante la fase de diseño, sino que también simplificó la gestión de los diferentes modos de operación. En este sentido, la implementación de la Guía GEMMA ha dotado al sistema de una robustez significativa, permitiendo una gestión eficaz de las paradas de emergencia, los reinicios y la alternancia entre los modos de control manual y automático.

Asimismo, la incorporación de una interfaz HMI (TP1500 Comfort) ha completado el proyecto proporcionando una herramienta esencial para la supervisión. El sistema SCADA desarrollado permite a los operarios interactuar de manera intuitiva con la planta, facilitando el monitoreo en tiempo real, la gestión de usuarios y el diagnóstico rápido de alarmas, elementos indispensables en cualquier entorno industrial moderno.

\subsection{Trabajo a futuro}
A pesar de los resultados satisfactorios obtenidos en la simulación, el proyecto presenta diversas oportunidades de mejora y expansión. Una de las líneas de trabajo más inmediatas sería el análisis detallado de los tiempos de ciclo de cada estación, con el objetivo de optimizar la lógica de los Grafcets y eliminar cuellos de botella para aumentar la cadencia de producción global.

Por otro lado, la integración del sistema de control con niveles superiores de gestión, como sistemas MES o ERP, representaría un avance significativo hacia la Industria 4.0. Esto permitiría automatizar la gestión de pedidos y almacenar datos históricos de producción en bases de datos externas para su posterior análisis.

Finalmente, el paso lógico siguiente sería trasladar la lógica validada en el gemelo digital a un entorno de hardware real. Esta implementación física requeriría un ajuste fino de la configuración de sensores y actuadores, así como la posible incorporación de algoritmos de mantenimiento predictivo que analicen el desgaste de los componentes para anticipar fallos antes de que detengan la producción.
\section{Anexos}\label{sec:anexos}

\subsection{Anexo 1: Grafcet\_0\_Principal.scl}\label{anexo01}
\lstinputlisting[language=Pascal, caption={Código del grafcet principal}]{./anexos/0_Principal.txt}

\subsection{Anexo 2: Grafcet\_10.scl}\label{anexo02}
\lstinputlisting[language=Pascal, caption={Código de la máquina 1: vacuna y soporte}]{./anexos/1_MAQ1_VAC_SOP.txt}

\subsection{Anexo 3: Grafcet\_10\_1.scl}\label{anexo03}
\lstinputlisting[language=Pascal, caption={Código de la máquina 1: embalaje}]{./anexos/1_MAQ1_EMB.txt}

\subsection{Anexo 4: Grafcet\_20.scl}\label{anexo04}
\lstinputlisting[language=Pascal, caption={Código de la máquina 2: vacuna y soporte}]{./anexos/2_MAQ2.txt}

\subsection{Anexo 5: Grafcet\_20\_1.scl}\label{anexo05}
\lstinputlisting[language=Pascal, caption={Código de la máquina 2: embalaje}]{./anexos/2_MAQ2_EMB.txt}

\subsection{Anexo 6: Grafcet\_30.scl}\label{anexo06}
\lstinputlisting[language=Pascal, caption={Código de la máquina 3}]{./anexos/3_MAQ3.txt}

\subsection{Anexo 7: Grafcet\_40.scl}\label{anexo07}
\lstinputlisting[language=Pascal, caption={Código de la máquina 4}]{./anexos/4_MAQ4.txt}

\subsection{Anexo 8: Grafcet\_50.scl}\label{anexo08}
\lstinputlisting[language=Pascal, caption={Código de la máquina 5: vacunas y 6 soporte}]{./anexos/5_MAQ5_VAC_6_SOP.txt}

\subsection{Anexo 9: Grafcet\_70.scl}\label{anexo09}
\lstinputlisting[language=Pascal, caption={Código de la máquina 5: soporte}]{./anexos/7_MAQ5_SOP.txt}

\subsection{Anexo 10: Grafcet\_80.scl}\label{anexo10}
\lstinputlisting[language=Pascal, caption={Código de la máquina 5: embalaje}]{./anexos/8_MAQ5_EMB.txt}

\subsection{Anexo 11: Grafcet\_90.scl}\label{anexo11}
\lstinputlisting[language=Pascal, caption={Código de la máquina 7: embalaje}]{./anexos/9_MAQ7_EMB.txt}

\subsection{Anexo 12: Grafcet\_100.scl}\label{anexo12}
\lstinputlisting[language=Pascal, caption={Código de la máquina 8: embalaje}]{./anexos/10_MAQ8_EMB.txt}

\subsection{Anexo 13: Grafcet\_110.scl}\label{anexo13}
\lstinputlisting[language=Pascal, caption={Funciones de mantenimiento}]{./anexos/11_Mant.txt}

\subsection{Anexo 14: Grafcet\_120.scl}\label{anexo14}
\lstinputlisting[language=Pascal, caption={Código de vaciado general}]{./anexos/12_VaciadoGen.txt}

\subsection{Anexo 15: Grafcet\_125.scl}\label{anexo15}
\lstinputlisting[language=Pascal, caption={Código de vaciado tipo 1}]{./anexos/13_VaciadoTipo1.txt}

\subsection{Anexo 16: Grafcet\_130.scl}\label{anexo16}
\lstinputlisting[language=Pascal, caption={Código de vaciado tipo 2}]{./anexos/14_VaciadoTipo2.txt}

\subsection{Anexo 17: Grafcet\_145.scl}\label{anexo17}
\lstinputlisting[language=Pascal, caption={Modo manual}]{./anexos/15_Manual.txt}






%------------------------------------------------------------------
%------------------------------------------------------------------

\end{document}

%------------------------------------------------------------------
%------------------------------------------------------------------
%------------------------------------------------------------------

