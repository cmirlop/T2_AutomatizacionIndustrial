\begin{tikzpicture}
  \ttfamily

  %---------------------------
  % Columna izquierda
  %---------------------------
  \etapaInicial{1000}

  % 100 -> 101  (transición de alarmas)
  \etapa{1001}{1000}{SETA};
  \primeraAccion{1001}{X1001-A1}{SETA\_ACTIVADA}
  %\forzado{X1101-A1}{X1101-A2}{G100\{$INIT$\}}
  \etapa{1002}{1001}{\NOT{SETA}};
  \primeraAccion{1002}{X1002-A1}{SETA\_DESACT}
  \etapa{1003}{1002}{RESET};
  \primeraAccion{1003}{X1003-A1}{Act\_Mantenimiento}
  \etapa{1004}{1003}{Mantenimiento\_Fin};
  \primeraAccion{1004}{X1004-A1}{Emergencia\_Acabada}
  \retornoInicio[4.2cm]{1004}{1000}{\underline{1}}
  

  


  % 101 -> 102 (bajada del lazo)
  %\etapa[1.35\nodeDist]{102}{101}{\underline{1}}
  %\forzado{102}{X102-A1}{G10\{INIT\}}
  %\otraAccion{X102-A1}{X102-A2}{Reiniciar\_Cinta}

  % 102 -> 104 (bajada final izquierda)
  %\concurrencia{102}{103}{TRANSICION}{104}

  

  %---------------------------
  % Columna derecha (etapa 103)
  %---------------------------
  % Colocamos 103 a la derecha de 102
  %\node[block, right of = X102, node distance = 9cm] (X103) {};
  %\node[below=0ex,anchor=south] at (X103) {103};
  %\forzado{103}{X103-A1}{G10\{\}}
  %\otraAccion{X103-A1}{X103-A2}{Reanudar\_cinta}

  %---------------------------
  % Lazo superior y retorno derecho
  %---------------------------
  % Ajuste del “rectángulo” superior: barra (101)→(103) con texto “Reset • Resume”
 %\setlength{\retornoDist}{0.75cm}
 % \setlength{\enlaceRetornoDist}{0.35\nodeDist}
 %\saltoHorizontal[9cm]{103}{104}{\makebox[0pt][c]{Reset $\bullet$ Resume}}
 %\transicion{103}{104}{a}
  % Retorno vertical derecho (de 103 hacia 102) para cerrar el lazo
  %\retornoHorizontal{103}{104}{}

  %---------------------------
  % Retorno grande inferior hacia el inicio (104 → 100)
  %---------------------------
  %\retornoInicio[4.2cm]{104}{100}{\underline{1}}

\end{tikzpicture}
