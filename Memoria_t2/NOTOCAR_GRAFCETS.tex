\documentclass[a4,10pt]{article}

% -------------------------------------------------------------------------

\usepackage[utf8]{inputenc}
\usepackage{caption}

\usepackage[
	vmargin={1cm,1cm},
	includefoot,includehead,
	]{geometry}

\input{./configuracions/preambul_GRAFCET}
\usepackage{listings}

\lstset{
	language=Pascal,
	basicstyle=\scriptsize\ttfamily,         
	commentstyle=\color[rgb]{.2,.5,.2},
	tabsize=4,
	classoffset=0,
	deletekeywords=[1]{reset},	
	morekeywords={
		END_IF, ELSIF, 
		},
	keywordstyle=\color{blue},
	morecomment=[l]{//}
%	numbers=left,                    
%	numbersep=4pt,                   
%	numberstyle=\sffamily\tiny\color[rgb]{.4,.4,.4}, 		
	}

% -------------------------------------------------------------------------

\parindent = 0cm
\parskip = 2ex

% -------------------------------------------------------------------------

\begin{document}

%\nodeDist = 2.5cm
%\retornoDist = 0.7\nodeDist % Distància vertical dels retorns
%\bifDistX = 10ex
%\bifDistY = 1.1\nodeDist
%\sincDistYabove = 0.85\nodeDist
%\sincDistYbelow = 0.6\sincDistYabove
%\sincDistYBlockbelow = 0.5\nodeDist

% -------------------------------------------------------------------------

\begin{figure}[htbp]
	\centering
	\scalebox{0.75}{\input{grafcet_01}}
	\medskip
	\caption{Grafcet de ejemplo}
\end{figure}


\bigskip

% -------------------------------------------------------------------------

Un ejemplo de \textbf{ecuaciones lógicas}:

\newlength{\sepsGrups}
\setlength{\sepsGrups}{2ex}

$S_{20} = X_{22} \cdot \Not{\mathtt{B3}} + \mathtt{FirstScan}$\\
$S_{21} = X_{20} \cdot \mathtt{ACTIVA\_REACTOR}$\\
$S_{22} = X_{21} \cdot (1800\;\mathrm{s}/X_{21})$\\[\sepsGrups]
%
$X_{20} = S_{20} + X_{20}\cdot\Not{S_{21}}$\\
$X_{21} = S_{21} + X_{21}\cdot\Not{S_{22}}$\\
$X_{22} = S_{22} + X_{22}\cdot\Not{S_{20}}$\\[\sepsGrups]
%
$\mathtt{REACTOR\_VACIO} = X_{20}$\\


% -------------------------------------------------------------------------

\begin{figure}[htbp]
	\centering
	\scalebox{0.75}{%\nodeDist = 2.5cm
%\retornoDist = 0.7\nodeDist % Distància vertical dels retorns
%\bifDistX = 10ex
%\bifDistY = 1.1\nodeDist
%\sincDistYabove = 0.85\nodeDist
%\sincDistYbelow = 0.6\sincDistYabove
%\sincDistYBlockbelow = 0.5\nodeDist


\begin{tikzpicture}[auto]
	
	\ttfamily
	
	\etapaInicial{125}
	
	%---
	\etapa[1.25\nodeDist]{126}{125}{MAQ\_VAC\_VAC\_1};
	\primeraAccion{126}{X126-A1}{QC1\_VAC}
    \otraAccion{X126-A1}{X126-A2}{QC2\_VAC}

	\bifurcacion[80]{126}{127}{10s/X1}{126a}{\NOT{S2}}

	

	\primeraAccion{127}{X127-A1}{Fin\_VAC\_VAC\_1}
	
	%--
	
	\primeraAccion{126a}{X126a-A1}{QC2\_VAC}
	
	%---
	\etapa[1.25\nodeDist]{126b}{126a}{S3};
	\primeraAccion{126b}{X126b-A1}{QTransfer1\_VAC}
	
	\saltoHorizontal[10cm]{126}{126b}{S3}


	\retornoInicio[-19em]{126b}{0}{\NOT{S3}}
	\retornoInicio[6em]{127}{0}{Fin\_Vaciado}

\end{tikzpicture}}
	\medskip
	\caption{Grafcet de ejemplo general. Se ha modificado la posición de la transición y la posición del retorno para que quede equilibrado}
\end{figure}

% -------------------------------------------------------------------------

\begin{figure}[htbp]
	\centering
	\scalebox{0.75}{\begin{tikzpicture}
  \ttfamily

  %---------------------------
  % Columna izquierda
  %---------------------------
  \etapaInicial{130}
  \etapa[1.25\nodeDist]{131}{130}{MAQ\_VAC\_VAC\_3};
  \primeraAccion{131}{X131-A1}{MAQ\_X}
  
 
  
  
    \etapa[1.25\nodeDist]{132}{131}{7s/X1};
 	\primeraAccion{132}{X132-A1}{Fin\_VAC\_VAC\_3}
  
    

  \retornoInicio[6cm]{132}{130}{Fin\_Vaciado}



\end{tikzpicture}}
	\medskip
	\caption{Grafcet de ejemplo}
\end{figure}


% -------------------------------------------------------------------------

\begin{figure}[htbp]
	\centering
	\scalebox{0.75}{\begin{tikzpicture}[auto]
	
	\ttfamily
	
	\setlength{\nodeDist}{1.2\nodeDist}
		
	% --------------------------------------------
	\etapaInicial{10}	
	\primeraAccion{10}{X10-A1}{CONT := 0}
	\sobreActivacion{X10-A1}

	% --------------------------------------------
	{\renewcommand{\transitionPos}{0.35}
	\etapa[1.35\nodeDist]{11}{10}{M}
	\primeraAccion{11}{X11-A1}{P1}
	\condicionada{X11-A1}{\NOT{T1}}
	\otraAccion{X11-A1}{X11-A2}{P2}
	\condicionada{X11-A2}{\NOT{T2}}
	\otraAccion{X11-A2}{X11-A3}{CONT := CONT + 1}
	\sobreActivacion{X11-A3}	
	}
	
	% --------------------------------------------
	\etapa[1.35\nodeDist]{12}{11}{T1$\,\cdot\,$T2$\,\cdot\,$REACTOR\_VACIO}
	\primeraAccion{12}{X12-A1}{V1}
	\condicionada{X12-A1}{B1}
	\otraAccion{X12-A1}{X12-A2}{V2}
	\condicionada{X12-A2}{B2}

	% --------------------------------------------
	\etapa{13}{12}{\NOT{B1}$\,\cdot\,$\NOT{B2}}
	\primeraAccion{13}{X13-A1}{ACTIVA\_REACTOR}

	% --------------------------------------------
	{
	\setlength{\retornoDist}{2.50cm}
	\renewcommand{\transitionPosRetorno}{0.40}
	\retornoInicio[3cm]{13}{10}{$[\mathtt{CONT} = 20]$}
	}
	
	% --------------------------------------------
	{
	\setlength{\retornoDist}{1.00cm}
	\setlength{\enlaceRetornoDist}{0.35\nodeDist}
	\retornoHorizontal[9.5cm]{13}{11}{$[\mathtt{CONT} < 20]$}
	}
            
\end{tikzpicture}}
	\medskip
	\caption{Grafcet de ejemplo en el que se han modificado las distancias para que quede equilibrado}
\end{figure}

% -------------------------------------------------------------------------

\clearpage

\begin{figure}[htbp]
	\centering
	\scalebox{0.7}{\begin{tikzpicture}[auto]
	
	\ttfamily
	
	\setlength{\nodeDist}{1.2\nodeDist}
		
	% --------------------------------------------
	
	\etapaInicial{0}
    \primeraAccion{0}{X0-A1}{LB\_INI}
    \otraAccion{X0-A1}{X0-A2}{Remover EME Almacén}

    % --------------------------------------------
	\etapa{1}{0}{B\_INI$\,\cdot\,$B\_STP$\,\cdot\,$B\_EME$\,\cdot\,$P\_SEG1$\,\cdot\,$P\_SEG2}
	\primeraAccion{1}{X1-A1}{LB\_STP}
    \otraAccion{X1-A1}{X1-A2}{LM\_NORM}
    \otraAccion{X1-A2}{X1-A3}{SEM\_G}

    % --------------------------------------------
	% Bifurcación

	\bifurcacion[35ex]{1}
    {2}{\NOT{B\_STP}$\,\cdot\,$B\_EME$\,\cdot\,$P\_SEG1$\,\cdot\,$P\_SEG2}
    {3}{\NOT{B\_EME}$\,+\,$\NOT{P\_SEG1}$\,+\,$\NOT{P\_SEG2}}
 	
    \primeraAccion{2}{X2-A1}{LB\_RST}
    \otraAccion{X2-A1}{X2-A2}{LM\_SG}
    \otraAccion{X2-A2}{X2-A3}{SEM\_Y}

    \primeraAccion{3}{X3-A1}{LB\_RST}
    \otraAccion{X3-A1}{X3-A2}{LM\_EME}
    
    
    %\etapa[3cm]{31}{3}
    
    % --------------------------------------------
    % Bifurcación

	\bifurcacionRetorno[27ex]{2}{4}{5}
    \retornoInicio[10ex]{4}{0}{B\_RST$\,\cdot\,$B\_STP$\,\cdot\,$\NOT{B\_INI}$\,\cdot\,$B\_EME$\,\cdot\,$P\_SEG1$\,\cdot\,$P\_SEG2}
    
	{
	\setlength{\enlaceRetornoDist}{0.1\nodeDist}
	\retorno[35ex]{5}{3}{\NOT{B\_EME}$\,+\,$\NOT{P\_SEG1}$\,+\,$\NOT{P\_SEG2}}
	}

    % --------------------------------------------

	{
	\setlength{\retornoDist}{3.25cm}
	\retornoInicioNF[16.2cm]{3}{0}{B\_RST$\,\cdot\,$B\_STP$\,\cdot\,$\NOT{B\_INI}$\,\cdot\,$B\_EME}
	}
            
\end{tikzpicture}

}
	\medskip
	\caption{Grafcet de ejemplo en el que se ha utilizado un retorno sin flecha y se ha modificado la distancia vertical de ese retorno}
\end{figure}

% -------------------------------------------------------------------------

\clearpage

\section*{Código SCL (IF -- THEN -- ELSE)}

\begin{lstlisting}
"T1".TON(IN := "X23" AND NOT #Alarma,
	PT := T#3s);

IF "T1".Q THEN // Disparo de la alarma
	#Alarma := TRUE;
END_IF;

// ---------------------------------------------------

IF EMERGENCIA OR (#Alarma AND #Reset) THEN
 
	#X20 := TRUE;
	#X21 := FALSE;
	#X22 := FALSE;
	...
 
	#Alarma := FALSE;
 
ELSIF (#Alarma AND #Resume) THEN
 
	#Alarma := FALSE;
 
END_IF;					

// ---------------------------------------------------

IF (NOT #Alarma AND NOT #EMERGENCIA) THEN
 
	#S20 := #X27 OR "FirstScan"
	#S21 := #X20 AND "F1";
	...
	
	#X20 := #S20 OR (#X20 AND NOT #S21);
	#X21 := #S21 OR (#X21 AND NOT #S22);
	...
	
	#QCIN_ADELA := #X22 OR #X23;
	#QCIN_ATRAS := #X25;
	#ACT_EST := #X24;
	#PIEZA_ESTAMPADA := #X27;
		 
ELSE // Alarma o emergencia

	#QCIN_ADELA := FALSE;
	#QCIN_ATRAS := FALSE;
	#ACT_EST := FALSE;
	#PIEZA_ESTAMPADA := FALSE;
 
END_IF;

// ---------------------------------------------------
// Fin
\end{lstlisting}


\clearpage

\section*{Código SCL (GRAFCET)}

\begin{lstlisting}
"T1".TON(IN := "X23" AND NOT #Alarma,
	PT := T#3s);

// ---------------------------------------------------

#S20e := (#X22e AND #CI) OR "FirstScan"
#S21e := #X20e AND "EMERGENCIA";
#S22e := #X21e;

#X20e := #S20e OR (#X20e AND NOT #S21e);
#X21e := #S21e OR (#X21e AND NOT #S22e);
#X22e := #S22e OR (#X22e AND NOT #S20e);

#FORCE_INIT_E := #X21e;
#FORCE_CURRENT_E := #X22e;

// ---------------------------------------------------

#S20a := (#X23a AND NOT #FORCE_CURRENT_E) OR "FirstScan"
#S21a := #X20a AND #Alarma AND NOT #FORCE_CURRENT_E;
#S22a := #X21a AND #Reset AND NOT #FORCE_CURRENT_E;
#S23a := #X22a OR #X21a AND #Resume AND NOT #FORCE_CURRENT_E;

#X20a := #S20a OR (#X20a AND NOT #S21a) OR #FORCE_INIT_E;
#X21a := #S20a OR (#X21a AND NOT #S22a AND NOT #S23a) AND NOT #FORCE_INIT_E;
#X22a := #S22a OR (#X22a AND NOT #S22a) AND NOT #FORCE_INIT_E;
#X23a := #S23a OR (#X23a AND NOT #S20a) AND NOT #FORCE_INIT_E;

#STOP_ACT := #X21a;
#FORCE_INIT := #FORCE_INIT_E OR #X23a;
#FORCE_CURRENT := #FORCE_CURRENT_E OR #X22a;

// ---------------------------------------------------
// Grafcet de produccion

#S20 := (#X27 AND NOT #FORCE_CURRENT) OR "FirstScan";
#S21 := (#X20 AND #F1) AND NOT #FORCE_CURRENT;
#S22 := (#X21 AND #INICIO AND #CIN_INICIO) AND NOT #FORCE_CURRENT;
...

// ---------------------------------------------------

#X20 := (#S20 OR (#X20 AND NOT #S21)) OR #FORCE_INIT:
#X21 := (#S21 OR (#X21 AND NOT #S22)) AND NOT #FORCE_INIT;
#X22 := (#S22 OR (#X22 AND NOT #S23)) AND NOT #FORCE_INIT;
...

// ---------------------------------------------------

#QCIN_ADELA := (#X22 OR #X23) AND NOT #STOP_ACT;
#QCIN_ATRAS := #X25 AND NOT #STOP_ACT;
#ACT_EST := #X24 AND NOT #STOP_ACT;
#PIEZA_ESTAMPADA := #X27 AND NOT #STOP_ACT;

#Alarma := "T1".Q AND NOT #FORCE_INIT;

// ---------------------------------------------------
// Fin
\end{lstlisting}

\end{document}

