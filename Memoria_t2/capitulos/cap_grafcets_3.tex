




\subsection{Grafcet Máquina 3}\label{sec:maq3}

Como vemos en la siguiente figura (\hyperref[fig:maq3]{Máquina 3}), esta máquina está compuesta por:
\begin{itemize}
	\item 1 cinta.
	\item 1 Sensor al final de la cinta.
\end{itemize}

\begin{figure}[H]
	\centering
	\includegraphics[width=0.8\linewidth]{./Figuras/maq3.png}
	\medskip
	\label{fig:maq3}
	\caption{Máquina 3 de la planta}
\end{figure}

Como vemos en el grafcet (\hyperref[graf:maq3]{Grafcet Máquina 3}), en esta máquina cuando nos entrega la pieza la máquina anterior, esta
solo será del tipo A, pero nos indicará si es buena o mala, para poder indicarle
a la máquina siguiente si la tiene que descartar o no.

El funcionamiento es recibir la pieza indicando si es buena o mala, 
llevarla hasta el final donde estará el sensor y si la siguiente máquina
está libre entregarla.

En está máquina solo se trata una alarma, que es que tardemos más de 10 segundos
en recorrer toda la cinta

\begin{figure}[H]
	\centering
	\scalebox{0.75}{%\nodeDist = 2.5cm
%\retornoDist = 0.7\nodeDist % Distància vertical dels retorns
%\bifDistX = 10ex
%\bifDistY = 1.1\nodeDist
%\sincDistYabove = 0.85\nodeDist
%\sincDistYbelow = 0.6\sincDistYabove
%\sincDistYBlockbelow = 0.5\nodeDist


\begin{tikzpicture}[auto]
	
	\ttfamily
	
% --------------------------------------------
	\etapaInicial{30}
	\primeraAccion{30}{X30-A1}{MAQ\_Libre}
	%---

	\bifurcacion[100]{30}{31}{Act\_Maq\_A\_Buena}{32}{Act\_Maq\_A\_Mala}

	\primeraAccion{31}{X31-A1}{QC3}
	\primeraAccion{32}{X32-A1}{QC3}
	
	%---
	
	%---
	\etapa[1.25\nodeDist]{31b}{31}{S3};

	\etapa[1.25\nodeDist]{31c}{31b}{MAQ\_SIG\_LIBRE};
	\primeraAccion{31c}{X31c-A1}{QC3}
	\otraAccion{X31c-A1}{X31c-A2}{SigMaqBuena}


	\etapa[1.25\nodeDist]{32b}{32}{S3};

	\etapa[1.25\nodeDist]{32c}{32b}{MAQ\_SIG\_LIBRE};
	\primeraAccion{32c}{X32c-A1}{QC3}
	\otraAccion{X32c-A1}{X32c-A2}{SigMaqMala}
	
	\retornoInicio[6em]{31c}{0}{\NOT{S3}}
	\retornoInicio[-16em]{32c}{0}{\NOT{S3}}

	








	




	

            
\end{tikzpicture}}
	\medskip
	\label{graf:maq3}
	\caption{Grafcet de la Máquina 3}
\end{figure}

Este grafcet está descompuesto en las siguientes ecuaciones algebráicas:\\

\setlength{\sepsGrups}{2ex}

% PRODUCCIÓN

% Transiciones (sin FORCE_CURRENT)
$S_{30} = \mathtt{FirstScan} + X_{31c}\cdot \Not{\mathtt{S3}} + X_{32c}\cdot \Not{\mathtt{S3}}$\\
$S_{31} = X_{30}\cdot \mathtt{Act\_Proceso\_Bueno}$\\
$S_{31b} = X_{31}\cdot \mathtt{S3}$\\
$S_{31c} = X_{31b}\cdot \mathtt{SigMaqLibre}$\\
$S_{32} = X_{30}\cdot \mathtt{Act\_Proceso\_malo}$\\
$S_{32b} = X_{32}\cdot \mathtt{S3}$\\
$S_{32c} = X_{32b}\cdot \mathtt{SigMaqLibre}$\\[\sepsGrups]

% Etapas (sin FORCE_INIT)
$X_{30}  = S_{30}  + X_{30}\cdot \Not{(S_{31}+S_{32})}$\\
$X_{31}  = S_{31}  + X_{31}\cdot \Not{S_{31b}}$\\
$X_{31b} = S_{31b} + X_{31b}\cdot \Not{S_{31c}}$\\
$X_{31c} = S_{31c} + X_{31c}\cdot \Not{S_{30}}$\\
$X_{32}  = S_{32}  + X_{32}\cdot \Not{S_{32b}}$\\
$X_{32b} = S_{32b} + X_{32b}\cdot \Not{S_{32c}}$\\
$X_{32c} = S_{32c} + X_{32c}\cdot \Not{S_{30}}$\\[\sepsGrups]

% Marcas / salidas
$\mathtt{Maq\_Libre}          = X_{30}\cdot \Not{\mathtt{STOP\_ACT}}$\\
$\mathtt{QC3}                = (X_{31}+X_{31c}+X_{32}+X_{32c}+\mathtt{Mantenimiento\_Cinta}+\mathtt{vaciado\_cinta}+\mathtt{man\_cinta})\cdot \Not{\mathtt{STOP\_ACT}}$\\
$\mathtt{Act\_SIG\_MAQ\_Buena}= X_{31c}\cdot \Not{\mathtt{STOP\_ACT}}$\\
$\mathtt{Act\_SIG\_MAQ\_Mala} = X_{32c}\cdot \Not{\mathtt{STOP\_ACT}}$\\




\subsection{Grafcet Máquina 4}
Como vemos en la siguiente figura (\hyperref[fig:maq4]{Máquina 4}), esta máquina esta compuesta por:
\begin{itemize}
	\item 1 transfer, que puede mover hacia un lado o ir en línea recta
	\item 1 Sensor al final de la cinta.
\end{itemize}

\begin{figure}[H]
	\centering
	\includegraphics[width=0.3\linewidth]{./Figuras/maq4.png}
	\medskip
	\label{fig:maq4}
	\caption{Máquina 4 de la planta}
\end{figure}

Como vemos en el (\hyperref[graf:maq4]{Grafcet Máquina 4}), en esta máquina cuando nos entrega la pieza la máquina anterior,esta
vendrá indicada si es buena o mala, por lo que al llegar al sensor si es buena se esperará
a que la siguiente máquina este libre, y si es mala, la descartaremos


En está máquina solo se trata una alarma, que es que tardemos más de 5 segundos
en llegar al sensor desde que nos indican que nos activemos


\begin{figure}[H]
	\centering
	\scalebox{0.75}{%\nodeDist = 2.5cm
%\retornoDist = 0.7\nodeDist % Distància vertical dels retorns
%\bifDistX = 10ex
%\bifDistY = 1.1\nodeDist
%\sincDistYabove = 0.85\nodeDist
%\sincDistYbelow = 0.6\sincDistYabove
%\sincDistYBlockbelow = 0.5\nodeDist


\begin{tikzpicture}[auto]
	
	\ttfamily
	
	\etapaInicial{40}
	\primeraAccion{40}{X40-A1}{Descartador\_Libre}
	%---

	\bifurcacion[100]{40}{41}{Act\_Descartador}{42}{SigMaqMala}

	\primeraAccion{41}{X41-A1}{QC4}
	\primeraAccion{42}{X42-A1}{QC4}
	
	%---
	
	%---
	\etapa[1.25\nodeDist]{41b}{41}{S4};
	\comentari[3]{X31b}{Esperamos máquina de calidad libre}

	\etapa[1.25\nodeDist]{41c}{41b}{SIG\_MAQ\_LIBRE};
	\primeraAccion{41c}{X41c-A1}{QC4}
	\otraAccion{X41c-A1}{X41c-A2}{ACT\_SIG\_MAQ}


	\etapa[1.25\nodeDist]{42b}{42}{S4};
	\primeraAccion{42b}{X42b-A1}{QTRANS}

	
	
	\retornoInicio[6em]{41c}{0}{\NOT{S4}}
	\retornoInicio[-16em]{42b}{0}{\NOT{S4}}











	




	

            
\end{tikzpicture}}
	\medskip
	\label{graf:maq4}
	\caption{Grafcet de la Máquina 4}
\end{figure}

Este grafcet está descompuesto en las siguientes ecuaciones algebráicas:\\

\setlength{\sepsGrups}{2ex}

% PRODUCCIÓN

% Transiciones (sin FORCE_CURRENT)
$S_{40} = \mathtt{FirstScan} + X_{41c}\cdot \Not{\mathtt{S4}} + X_{42b}\cdot \Not{\mathtt{S4}}$\\
$S_{41} = X_{40}\cdot \mathtt{Act\_Proceso\_Bueno}$\\
$S_{41b} = X_{41}\cdot \mathtt{S4}$\\
$S_{41c} = X_{41b}\cdot \mathtt{SigMaqLibre}$\\
$S_{42} = X_{40}\cdot \mathtt{Act\_Proceso\_Malo}$\\
$S_{42b} = X_{42}\cdot \mathtt{S4}$\\[\sepsGrups]

% Etapas (sin FORCE_INIT)
$X_{40}  = S_{40}  + X_{40}\cdot \Not{(S_{41}+S_{42})}$\\
$X_{41}  = S_{41}  + X_{41}\cdot \Not{S_{41b}}$\\
$X_{41b} = S_{41b} + X_{41b}\cdot \Not{S_{41c}}$\\
$X_{41c} = S_{41c} + X_{41c}\cdot \Not{S_{40}}$\\
$X_{42}  = S_{42}  + X_{42}\cdot \Not{S_{42b}}$\\
$X_{42b} = S_{42b} + X_{42b}\cdot \Not{S_{40}}$\\[\sepsGrups]

% Marcas / salidas
$\mathtt{Maq\_Libre}     = X_{40}\cdot \Not{\mathtt{STOP\_ACT}}$\\
$\mathtt{QC4}           = (X_{41}+X_{41c}+X_{42}+\mathtt{Mantenimiento\_Cinta}+\mathtt{vaciado\_cinta}+\mathtt{man\_cinta})\cdot \Not{\mathtt{STOP\_ACT}}$\\
$\mathtt{Act\_Sig\_Maq}  = X_{41c}\cdot \Not{\mathtt{STOP\_ACT}}$\\
$\mathtt{QTRANS}        = (X_{42b}+\mathtt{Mantenimiento\_Trans}+\mathtt{vaciado\_Trans}+\mathtt{man\_trans})\cdot \Not{\mathtt{STOP\_ACT}}$\\
