




\subsection{Grafcet máquina 3}

Como vemos en la siguiente figura (\ref{}), esta máquina esta compuesta por:
\begin{itemize}
	\item 1 cinta.
	\item 1 Sensor al final de la cinta.
\end{itemize}

\begin{figure}[H]
	\centering
	\includegraphics[width=0.8\linewidth]{./Figuras/maq3.png}
	\medskip
	\label{fig:maq2}
	\caption{Maquina 2 de la planta}
\end{figure}

Como vemos en el grafcet (REF), en esta máquina cuando nos entrega la pieza la máquina anterior,esta
solo será del tipo A, pero nos idicará si es buena o mala, para poder indicarle
a la máquina siguietne si la tiene que descartar o no.

El funcionamiento es recibir la pieza indicando si es buena o mala, 
llevarla hasta el final donde estará el sensor y si la siguiente máquina
esta libre entregarla.

En está máquina solo se trata una alarma, que es que tardemos más de 10 segundos
en recorrer toda la cinta

\begin{figure}[H]
	\centering
	\scalebox{0.75}{%\nodeDist = 2.5cm
%\retornoDist = 0.7\nodeDist % Distància vertical dels retorns
%\bifDistX = 10ex
%\bifDistY = 1.1\nodeDist
%\sincDistYabove = 0.85\nodeDist
%\sincDistYbelow = 0.6\sincDistYabove
%\sincDistYBlockbelow = 0.5\nodeDist


\begin{tikzpicture}[auto]
	
	\ttfamily
	
% --------------------------------------------
	\etapaInicial{30}
	\primeraAccion{30}{X30-A1}{MAQ\_Libre}
	%---

	\bifurcacion[100]{30}{31}{Act\_Maq\_A\_Buena}{32}{Act\_Maq\_A\_Mala}

	\primeraAccion{31}{X31-A1}{QC3}
	\primeraAccion{32}{X32-A1}{QC3}
	
	%---
	
	%---
	\etapa[1.25\nodeDist]{31b}{31}{S3};

	\etapa[1.25\nodeDist]{31c}{31b}{MAQ\_SIG\_LIBRE};
	\primeraAccion{31c}{X31c-A1}{QC3}
	\otraAccion{X31c-A1}{X31c-A2}{SigMaqBuena}


	\etapa[1.25\nodeDist]{32b}{32}{S3};

	\etapa[1.25\nodeDist]{32c}{32b}{MAQ\_SIG\_LIBRE};
	\primeraAccion{32c}{X32c-A1}{QC3}
	\otraAccion{X32c-A1}{X32c-A2}{SigMaqMala}
	
	\retornoInicio[6em]{31c}{0}{\NOT{S3}}
	\retornoInicio[-16em]{32c}{0}{\NOT{S3}}

	








	




	

            
\end{tikzpicture}}
	\medskip
	\label{graf:fresadora}
	\caption{Grafcet de la máquina 3}
\end{figure}


\subsection{Grafcet máquina 4}
Como vemos en la siguiente figura (\ref{}), esta máquina esta compuesta por:
\begin{itemize}
	\item 1 transfer, que puede mover hacia un lado o ir en linea recta
	\item 1 Sensor al final de la cinta.
\end{itemize}

\begin{figure}[H]
	\centering
	\includegraphics[width=0.8\linewidth]{./Figuras/maq4.png}
	\medskip
	\label{fig:maq2}
	\caption{Maquina 2 de la planta}
\end{figure}

Como vemos en el grafcet (REF), en esta máquina cuando nos entrega la pieza la máquina anterior,esta
vendrá indicada si es buena o mala, por lo que al llegar al sensor si es buena se esperará
a que la siguiente máquina este libre, y si es mala, la descartaremos


En está máquina solo se trata una alarma, que es que tardemos más de 5 segundos
en llegar al sensor desde que nos indican que nos activemos


\begin{figure}[H]
	\centering
	\scalebox{0.75}{%\nodeDist = 2.5cm
%\retornoDist = 0.7\nodeDist % Distància vertical dels retorns
%\bifDistX = 10ex
%\bifDistY = 1.1\nodeDist
%\sincDistYabove = 0.85\nodeDist
%\sincDistYbelow = 0.6\sincDistYabove
%\sincDistYBlockbelow = 0.5\nodeDist


\begin{tikzpicture}[auto]
	
	\ttfamily
	
	\etapaInicial{40}
	\primeraAccion{40}{X40-A1}{Descartador\_Libre}
	%---

	\bifurcacion[100]{40}{41}{Act\_Descartador}{42}{SigMaqMala}

	\primeraAccion{41}{X41-A1}{QC4}
	\primeraAccion{42}{X42-A1}{QC4}
	
	%---
	
	%---
	\etapa[1.25\nodeDist]{41b}{41}{S4};
	\comentari[3]{X31b}{Esperamos máquina de calidad libre}

	\etapa[1.25\nodeDist]{41c}{41b}{SIG\_MAQ\_LIBRE};
	\primeraAccion{41c}{X41c-A1}{QC4}
	\otraAccion{X41c-A1}{X41c-A2}{ACT\_SIG\_MAQ}


	\etapa[1.25\nodeDist]{42b}{42}{S4};
	\primeraAccion{42b}{X42b-A1}{QTRANS}

	
	
	\retornoInicio[6em]{41c}{0}{\NOT{S4}}
	\retornoInicio[-16em]{42b}{0}{\NOT{S4}}











	




	

            
\end{tikzpicture}}
	\medskip
	\label{graf:fresadora}
	\caption{Grafcet de Transfer 2 Cajas}
\end{figure}

