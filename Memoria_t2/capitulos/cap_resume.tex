\section*{Resumen}

En este proyecto se detalla el proceso completo desarrollado para resolver el problema propuesto de automatización 
y gestión de alarmas en una línea indexada. La solución se ha basado en el diseño de Grafcets que definen la secuencia 
de operaciones del sistema y su implementación en un PLC Siemens mediante el lenguaje SCL en el entorno TIA Portal. 
Además, el sistema integra una botonera externa cableada al PLC, desde la cual se generan las señales de control 
correspondientes a las órdenes de ejecución, parada y rearme del proceso.

Palabras clave: GRAFCET, SCL, TIA Portal, PLC.




%La segunda página de la memoria será un resumen del TFG/TFM tanto en la lengua de redacción del trabajo como en inglés. El resumen deberá reflejar el contenido de trabajo de forma precisa y descriptiva (entre 50 y 200 palabras).

%En esa misma página se incluirán, como máximo, 5 palabras clave. Las palabras clave asignadas al trabajo deberán reflejar la materia, método, lugar o cualquier otro aspecto relevante para la recuperación del trabajo en búsquedas bibliográficas en una base de datos. 