

\subsection{Grafcet Máquina 5 embalaje}\label{sec:maq5_emb}
Esta máquina (\hyperref[graf:maq5_emb]{Máquina 5 Embalaje}) tiene un funcionamiento similar al indicado en la  \hyperref[sec:maq3]{Subsección de la  Máquina 3}, pero
con un único camino, ya que su única función es llevar una pieza buena al final de la cinta, 
para entregársela al transfer siguiente.
\begin{figure}[H]
	\centering
	\scalebox{0.75}{%\nodeDist = 2.5cm
%\retornoDist = 0.7\nodeDist % Distància vertical dels retorns
%\bifDistX = 10ex
%\bifDistY = 1.1\nodeDist
%\sincDistYabove = 0.85\nodeDist
%\sincDistYbelow = 0.6\sincDistYabove
%\sincDistYBlockbelow = 0.5\nodeDist


\begin{tikzpicture}[auto]
	
	\ttfamily
	
	\etapaInicial{80}
	\primeraAccion{80}{X80-A1}{Descartador\_Libre}
	%---
	\etapa[1.25\nodeDist]{81}{80}{Act\_Maq\_A\_Buena};
	\primeraAccion{81}{X81-A1}{QC5\_VAC}
	
	%--
	\etapa[1.25\nodeDist]{82}{81}{S5};
	\comentari[3]{X82}{Esperamos máquina de calidad libre}

	%---
	\etapa[1.25\nodeDist]{83}{82}{SIG\_MAQ\_LIBRE};
	\primeraAccion{83}{X83-A1}{QC3}
	\otraAccion{X83-A1}{X83-A2}{Act\_SIG\_MAQ}
	




	\retornoInicio[6em]{83}{0}{\NOT{S5}}

\end{tikzpicture}}
	\medskip
	\label{graf:maq5_emb}
	\caption{Grafcet de la Máquina 5 Embalaje}
\end{figure}

Este grafcet está descompuesto en las siguientes ecuaciones algebráicas:\\

\setlength{\sepsGrups}{2ex}

% PRODUCCIÓN

% Transiciones (sin FORCE_CURRENT)
$S_{80} = \mathtt{FirstScan} + X_{83}\cdot \Not{\mathtt{S5}}$\\
$S_{81} = X_{80}\cdot \mathtt{Act\_Proceso}$\\
$S_{82} = X_{81}\cdot \mathtt{S5}$\\
$S_{83} = X_{82}\cdot \mathtt{SigMaqLibre}$\\[\sepsGrups]

% Etapas (sin FORCE_INIT)
$X_{80} = S_{80} + X_{80}\cdot \Not{S_{81}}$\\
$X_{81} = S_{81} + X_{81}\cdot \Not{S_{82}}$\\
$X_{82} = S_{82} + X_{82}\cdot \Not{S_{83}}$\\
$X_{83} = S_{83} + X_{83}\cdot \Not{S_{80}}$\\[\sepsGrups]

% Marcas / salidas
$\mathtt{Maq\_Libre}    = X_{80}\cdot \Not{\mathtt{STOP\_ACT}}$\\
$\mathtt{QC3}          = (X_{81}+X_{83}+\mathtt{Mantenimiento\_Cinta}+\mathtt{vaciado\_cinta})\cdot \Not{\mathtt{STOP\_ACT}}$\\
$\mathtt{Act\_Sig\_Maq} = X_{83}\cdot \Not{\mathtt{STOP\_ACT}}$\\


\subsection{Grafcet Máquina 6 embalaje}
En esta máquina (\hyperref[graf:maq6_emb]{Máquina 6 Embalaje}) la función es desplazar el embalaje a la cinta final del proceso,
que se detalla más adelante.

\begin{figure}[H]
	\centering
	\includegraphics[width=0.3\linewidth]{./Figuras/maq6_EMB.png}
	\medskip
	\label{fig:maq6_emb}
	\caption{Máquina 6 de la línea de embalaje}
\end{figure}

\begin{figure}[H]
	\centering
	\scalebox{0.75}{%\nodeDist = 2.5cm
%\retornoDist = 0.7\nodeDist % Distància vertical dels retorns
%\bifDistX = 10ex
%\bifDistY = 1.1\nodeDist
%\sincDistYabove = 0.85\nodeDist
%\sincDistYbelow = 0.6\sincDistYabove
%\sincDistYBlockbelow = 0.5\nodeDist


\begin{tikzpicture}[auto]
	
	\ttfamily
	
% --------------------------------------------
	\ttfamily
	
	\etapaInicial{90}
	\primeraAccion{90}{X90-A1}{MAQ\_Libre}
	
	
	%---
	\etapa[1.25\nodeDist]{91}{90}{Act\_MAQ};
	\primeraAccion{91}{X91-A1}{QCINTA}
	
	%--
	\etapa[1.25\nodeDist]{92}{91}{SENSOR};

	%---
	\etapa[1.25\nodeDist]{93}{92}{SIG\_MAQ\_LIBRE};
	\primeraAccion{93}{X93-A1}{QTRANS}

	\etapa[1.25\nodeDist]{94}{93}{\NOT{SENSOR}};
	\primeraAccion{94}{X94-A1}{Act\_SIG\_MAQ}
	
	




	\retornoInicio[6em]{94}{0}{\underline{1}}
	
       
\end{tikzpicture}}
	\medskip
	\label{graf:maq6_emb}
	\caption{Grafcet de la Máquina 6 Embalaje}
\end{figure}

Este grafcet está descompuesto en las siguientes ecuaciones algebráicas:\\

\setlength{\sepsGrups}{2ex}

% PRODUCCIÓN

% Transiciones (sin FORCE_CURRENT)
$S_{90} = \mathtt{FirstScan} + X_{94}$\\
$S_{91} = X_{90}\cdot \mathtt{Act\_Proceso}$\\
$S_{92} = X_{91}\cdot \mathtt{SENSOR}$\\
$S_{93} = X_{92}\cdot \mathtt{SigMaqLibre}$\\
$S_{94} = X_{93}\cdot \Not{\mathtt{SENSOR}}$\\[\sepsGrups]

% Etapas (sin FORCE_INIT)
$X_{90} = S_{90} + X_{90}\cdot \Not{S_{91}}$\\
$X_{91} = S_{91} + X_{91}\cdot \Not{S_{92}}$\\
$X_{92} = S_{92} + X_{92}\cdot \Not{S_{93}}$\\
$X_{93} = S_{93} + X_{93}\cdot \Not{S_{94}}$\\
$X_{94} = S_{94} + X_{94}\cdot \Not{S_{90}}$\\[\sepsGrups]

% Marcas / salidas
$\mathtt{Maq\_Libre}    = X_{90}\cdot \Not{\mathtt{STOP\_ACT}}$\\
$\mathtt{QC3}          = (X_{91}+\mathtt{Mantenimiento\_Cinta}+\mathtt{vaicado\_cinta})\cdot \Not{\mathtt{STOP\_ACT}}$\\
$\mathtt{QTrans}       = (X_{93}+\mathtt{Mantenimiento\_Trans}+\mathtt{vaciado\_trnas})\cdot \Not{\mathtt{STOP\_ACT}}$\\
$\mathtt{Act\_Sig\_Maq} = X_{94}\cdot \Not{\mathtt{STOP\_ACT}}$\\
