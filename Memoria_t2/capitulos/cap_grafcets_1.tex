\section{Grafcets}
\label{sec:introduccion}

\subsection{Grafcet Principal}
En la siguiente figura (\hyperref[graf:principal]{Grafcet Principal}), vemos el grafcet principal, el cual se encarga 
de poner la máquina en marcha, una vez no hay emergencia,funciona de manera que si modificamos 
el selector a modo manual, se va a activar el modo manual hasta que lo desactivemos, que entonces ahi,
si apretamos marcha, empezará la máquina con el  (\hyperref[graf:mantenimiento]{Grafcet de mantenimiento}), y poteriormente activará 
ya las primeras máquinas de cada objeto.

\begin{figure}[H]
	\centering
	\scalebox{0.75}{%\nodeDist = 2.5cm
%\retornoDist = 0.7\nodeDist % Distància vertical dels retorns
%\bifDistX = 10ex
%\bifDistY = 1.1\nodeDist
%\sincDistYabove = 0.85\nodeDist
%\sincDistYbelow = 0.6\sincDistYabove
%\sincDistYBlockbelow = 0.5\nodeDist


\begin{tikzpicture}[auto]
	
	\ttfamily
	
	% --------------------------------------------
	\etapaInicial{0}
	\primeraAccion{0}{X0-A1}{$\mathtt{C:=0}$}
	\sobreActivacion{X0-A1}{X0-A2}{}
	
	
	\bifurcacion[100]{0}{1}{Marcha $\cdot$ \NOT{MAN}}{4}{MAN}

	\retornoInicio[-6em]{4}{0}{\NOT{MAN}}

	% --------------------------------------------
  	{
   	\renewcommand{\transitionPos}{0.35}
   	%\etapa[1.25\nodeDist]{1}{0}{MARCHA};
	\primeraAccion{1}{X1-A1}{Act\_Mantenimiento}
	%\otraAccion{X1-A1}{X1-A2}{QCINALIM}
	}
	
	% --------------------------------------------
   	\etapa[1.5\nodeDist]{2}{1}{Mantenimiento\_Fin};
	\primeraAccion{2}{X2-A1}{Proceso\_Activo}
	%\condicionada{X2-A2}{CONDICION}
	%\comentari[1]{X2-A1}{Máquina fresadora en marcha}

	\etapa[1.25\nodeDist]{3}{2}{Pieza\_Empaquetada};
	\primeraAccion{3}{X3-A1}{Proceso\_Activo}
	\otraAccion{X3-A1}{X3-A2}{$\mathtt{C:=C+1}$}
	\sobreActivacion{X3-A2}{X3-A3}{}

	\bifurcacionRetorno[15ex]{3}{6}{7}
	\retornoInicio[6em]{6}{0}{Paro\_Ciclo}
	{
	\setlength{\enlaceRetornoDist}{0.30\nodeDist}
	\retorno[18em]{7}{2}{\NOT{Paro\_Ciclo}}
	}
	%\retornoInicio[7em]{4}{0}{\NOT{INICIO}}
	%\retorno[18em]{4}{3}{INICIO}


            
\end{tikzpicture}

}
	\medskip
	\label{graf:principal}
	\caption{Grafcet Principal}
\end{figure}

Este grafcet está descompuesto en las siguientes ecuaciones algebráicas:\\

\newlength{\sepsGrups}
\setlength{\sepsGrups}{2ex}

% Transiciones (sin FORCE_CURRENT_E)
$S_{0} = \mathtt{FirstScan} + X_{3}\cdot \mathtt{Paro\_Ciclo} + X_{4}\cdot \Not{\mathtt{Modo\_Manual}}$\\
$S_{1} = X_{0}\cdot (\mathtt{Marcha}+\mathtt{HMI\_Marcha})\cdot \Not{\mathtt{Modo\_Manual}}$\\
$S_{2} = X_{1}\cdot \mathtt{Mantenimiento\_Fin} + X_{3}\cdot \Not{\mathtt{Paro\_Ciclo}}$\\
$S_{3} = X_{2}\cdot \mathtt{Pieza\_Empaquetada}$\\
$S_{4} = X_{0}\cdot \mathtt{Modo\_Manual}$\\[\sepsGrups]

% Etapas (sin FORCE_INIT_E)
$X_{0} = S_{0} + X_{0}\cdot \Not{(S_{1}+S_{4})}$\\
$X_{1} = S_{1} + X_{1}\cdot \Not{S_{2}}$\\
$X_{2} = S_{2} + X_{2}\cdot \Not{S_{3}}$\\
$X_{3} = S_{3} + X_{3}\cdot \Not{(S_{0}+S_{2})}$\\
$X_{4} = S_{4} + X_{4}\cdot \Not{S_{0}}$\\[\sepsGrups]

% Marcas
$\mathtt{Act\_Mantenimiento} = X_{1}$\\
$\mathtt{Proceso\_Activo} = X_{2} + X_{3}$\\
$\mathtt{Modo\_Manual\_Activado} = X_{4}$\\[\sepsGrups]

% Contadores (mismo tipo/formato del ejemplo)
IF X0 THEN\\
\hspace*{2em}C = 0\\
END IF\\
IF X3 THEN\\
\hspace*{2em}C = C + 1\\
END IF\\[\sepsGrups]


\subsection{Grafcet Paro}
En la siguietne figura(\hyperref[graf:paro]{Grafcet Paro}) observamos el grafcet que trata el botón del paro, 
cuando apretamos este botón activamos una marca que recibe el grafcet principal para que se deje 
de emitir piezas y se queden las que están ya en la máquina.

\begin{figure}[H]
	\centering
	\scalebox{0.75}{%\nodeDist = 2.5cm
%\retornoDist = 0.7\nodeDist % Distància vertical dels retorns
%\bifDistX = 10ex
%\bifDistY = 1.1\nodeDist
%\sincDistYabove = 0.85\nodeDist
%\sincDistYbelow = 0.6\sincDistYabove
%\sincDistYBlockbelow = 0.5\nodeDist


\begin{tikzpicture}[auto]
	
	\ttfamily
	
	% --------------------------------------------
	\etapaInicial{0P}
		

	% --------------------------------------------
  	{
   	\renewcommand{\transitionPos}{0.35}
   	\etapa[1.25\nodeDist]{1}{0}{PARO};
	\primeraAccion{1}{X1-A1}{Paro\_de\_Ciclo}
	%\otraAccion{X1-A1}{X1-A2}{QCINALIM}
	}
	\retornoInicio[6em]{1}{0}{X0}
	%\retornoInicio[7em]{4}{0}{\NOT{INICIO}}
	%\retorno[18em]{4}{3}{INICIO}


            
\end{tikzpicture}

}
	\medskip
	\label{graf:paro}
	\caption{Grafcet del botón de Paro}
\end{figure}

Este grafcet está descompuesto en las siguientes ecuaciones algebráicas:\\


% Transiciones (sin FORCE_CURRENT_E)
$S_{0p} = \mathtt{FirstScan} + X_{0}$\\
$S_{1p} = X_{0}\cdot (\mathtt{STOP\_NEG} + \mathtt{HMI\_Paro})$\\[\sepsGrups]

% Etapas (sin FORCE_INIT_E)
$X_{0p} = S_{0p} + X_{0p}\cdot \Not{S_{1p}}$\\
$X_{1p} = S_{1p} + X_{1p}\cdot \Not{S_{0p}}$\\[\sepsGrups]

% Marcas
$\mathtt{Paro\_Ciclo} = X_{1p}$\\