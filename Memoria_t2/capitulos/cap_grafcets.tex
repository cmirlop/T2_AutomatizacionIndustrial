
\section{Grafcets}\label{sec:grafcets}

\subsection{Grafcet Principal}



\subsection{Grafcet de la cinta de entrada}\label{sec:graf_cin_ini}




\begin{figure}[H]
	\centering
	\scalebox{0.75}{%\nodeDist = 2.5cm
%\retornoDist = 0.7\nodeDist % Distància vertical dels retorns
%\bifDistX = 10ex
%\bifDistY = 1.1\nodeDist
%\sincDistYabove = 0.85\nodeDist
%\sincDistYbelow = 0.6\sincDistYabove
%\sincDistYBlockbelow = 0.5\nodeDist


\begin{tikzpicture}[auto]
	
	\ttfamily
	
	% --------------------------------------------
	\etapaInicial{10}
	\primeraAccion{10}{X10-A1}{Cinta\_Libre}	

	%-----
	\renewcommand{\transitionPos}{0.35}
   	\etapa[1.25\nodeDist]{11}{10}{Proceso\_Activo};
	\comentari[3]{X11}{Esperando a que pongan una pieza}
	

	%---
	\etapa[1.25\nodeDist]{12}{11}{CINAI$\,\cdot\,$\NOT{CINAF}};
	\primeraAccion{12}{X12-A1}{QCINA}
	\otraAccion{X12-A1}{X12-A2}{Alarma\_Cinta\_1}
	\condicionada{X12-A2}{3s/x12}

	%---
	\etapa[1.25\nodeDist]{12b}{12}{\NOT{CINAI}};
	\primeraAccion{12b}{X12b-A1}{QCINA}
	\condicionada{X12b-A1}{\NOT{CINAI}}
	\otraAccion{X12b-A1}{X12b-A2}{Alarma\_Cinta\_2}
	\condicionada{X12b-A2}{3s/x12b}

	%---
	\etapa[1.25\nodeDist]{13}{12b}{CINAF};
	\comentari[3]{X13}{Esperando a Empujadora 1 Libre}


	%---
	\etapa[1.25\nodeDist]{14}{13}{Empujadora1\_Libre$\,\cdot\,$\NOT{CINAI}};
	\primeraAccion{14}{X14-A1}{QCINA}
	\condicionada{X14-A1}{\NOT{CINAI}}
	\otraAccion{X14-A1}{X14-A2}{Alarma\_Cinta\_3}
	\condicionada{X14-A2}{3s/x14}


	%----
	\etapa[1.25\nodeDist]{14b}{14}{\NOT{CINAF}};
	\primeraAccion{14b}{X14b-A1}{QCINA}
	\condicionada{X14b-A1}{\NOT{CINAI}}
	\otraAccion{X14b-A1}{X14b-A2}{Alarma\_Cinta\_4}
	\condicionada{X14b-A2}{3s/x14b}
	\otraAccion{X14b-A2}{X14b-A3}{Act\_EMP}


	%----
	\etapa[1.25\nodeDist]{15}{14b}{Pieza\_REC\_EMP};
	\primeraAccion{15}{X15-A1}{Pieza\_Entregada\_Empujadora\_1}
	\retornoInicio[6em]{15}{0}{\underline{1}}



	

            
\end{tikzpicture}}
	\medskip
	\caption{Grafcet de la cinta de entrada}
\end{figure}

Este grafcet está descompuesto en las siguientes ecuaciones algebráicas:\\
% Transiciones
$S_{10} = \mathtt{FirstScan} + X_{15}$\\
$S_{11} = X_{10}\cdot \mathtt{Proceso\_Activo}$\\
$S_{12} = X_{11}\cdot \mathtt{CINAI\_M}$\\
$S_{12b} = X_{12}\cdot \Not{\mathtt{CINAI\_M}}$\\
$S_{13} = X_{12b}\cdot \mathtt{CINAF\_M}$\\
$S_{14} = X_{13}\cdot \mathtt{Empujadora\_Libre}\cdot \Not{\mathtt{CINAI\_M}}$\\
$S_{14b} = X_{14}\cdot \Not{\mathtt{CINAF\_M}}$\\
$S_{15} = X_{14b}\cdot \mathtt{Pieza\_rec\_emp}$\\[\sepsGrups]
%
% Etapas
$X_{10} = S_{10} + X_{10}\cdot \Not{S_{11}}$\\
$X_{11} = S_{11} + X_{11}\cdot \Not{S_{12}}$\\
$X_{12} = S_{12} + X_{12}\cdot \Not{S_{12b}}$\\
$X_{12b} = S_{12b} + X_{12b}\cdot \Not{S_{13}}$\\
$X_{13} = S_{13} + X_{13}\cdot \Not{S_{14}}$\\
$X_{14} = S_{14} + X_{14}\cdot \Not{S_{14b}}$\\
$X_{14b} = S_{14b} + X_{14b}\cdot \Not{S_{15}}$\\
$X_{15} = S_{15} + X_{15}\cdot \Not{S_{10}}$\\
$\mathtt{CINAI\_M} = S_{15} + \mathtt{CINAI\_M}\cdot \Not{X_{10}}$\\
$\mathtt{CINAF\_M} = X_{10} + \mathtt{CINAF\_M}\cdot \Not{S_{10}}$\\[\sepsGrups]
%
% Marcas
$\mathtt{CINTA\_LIBRE} = X_{10}$\\
$\mathtt{QCINA} = X_{12} + \big(X_{12b} + X_{14b} + X_{14}\cdot \Not{\mathtt{CINAI\_M}}\big) + \mathtt{Man\_Act\_QCIN}$\\
$\mathtt{Act\_EMP} = X_{14}$\\
$\mathtt{Pieza\_Entregada\_Empujadora} = X_{15}$\\


\subsection{Grafcet de Empujadores 1 y 2}
Este grafcet es el mismo tanto para el empujador 1 como el 2, ya que su función es la misma, el unico cambio es
que en el empujador 2, las varibables contienen un 2 en el nombre.
El funcionamiento es el siguiente:
\begin{itemize}
	\item Avisamos que estamos libres, y como hemos visto en el grafcet anterior(\ref{sec:graf_cin_ini}) 
	nos activan este grafcet,donde esperamos a que llegue la pieza durante 1 segundo, para una 
	vez tengamos la pieza, esperar a que la fresadora este disponible y poder enviarle la pieza
	\item Una vez nos avisa la fresadora que esta libre, moveremos el empujador hacia adelante hasta 
	alcanzar el final de carrera, que volveremos a la posicion de reposo
	\item Siempre tenemos en cuenta que si el empujador tarda más de 3 segundos en alcanzar cualquiera
	de los finales de carrera, saltará una alarma.
\end{itemize}


\begin{figure}[H]
	\centering
	\scalebox{0.75}{%\nodeDist = 2.5cm
%\retornoDist = 0.7\nodeDist % Distància vertical dels retorns
%\bifDistX = 10ex
%\bifDistY = 1.1\nodeDist
%\sincDistYabove = 0.85\nodeDist
%\sincDistYbelow = 0.6\sincDistYabove
%\sincDistYBlockbelow = 0.5\nodeDist


\begin{tikzpicture}[auto]
	
	\ttfamily
	
	% --------------------------------------------
	\etapaInicial{20}
	\primeraAccion{20}{X20-A1}{EMP1\_Libre}

	%-----
	\etapa[1.25\nodeDist]{21}{20}{Act\_EMP};
	\comentari[21]{X21}{Pieza LLegando}

	%---
	\etapa[1.25\nodeDist]{22}{21}{1s/x21};
	\primeraAccion{22}{X22-A1}{Pieza\_REC\_EMP}

	%---
	\etapa[1.25\nodeDist]{23}{22}{\underline{1}};
	\comentari[3]{X23}{Esperando Fresadora}

	%---
	\etapa[1.25\nodeDist]{24}{23}{Fresadora\_Libre};
	\primeraAccion{24}{X24-A1}{QE1BW}
	\otraAccion{X24-A1}{X24-A2}{Act\_Fresadora}
	\otraAccion{X24-A2}{X24-A3}{Alarma\_EMP1\_1}
	\condicionada{X24-A3}{3s/x24}

	%---
	\etapa[1.25\nodeDist]{25}{24}{EMP1F};
	\primeraAccion{25}{X25-A1}{QE1BW}
	\otraAccion{X25-A1}{X25-A2}{Alarma\_EMP1\_2}
	\condicionada{X25-A2}{3s/x25}
	\retornoInicio[6em]{25}{0}{EMP1I}




		




	

            
\end{tikzpicture}}
	\medskip
	\caption{Grafcet del Empujador 1 y 2}
	\label{graf:EMP}
\end{figure}
Este grafcet está descompuesto en las siguientes ecuaciones algebráicas:\\

% Transiciones
$S_{20} = \mathtt{FirstScan} + X_{25}\cdot \mathtt{EMPi}$\\
$S_{21} = X_{20}\cdot \mathtt{Act\_EMP}$\\
$S_{22} = X_{21}\cdot \mathtt{T1.Q}$\\
$S_{23} = X_{22}$\\
$S_{24} = X_{23}\cdot \mathtt{Fresadora\_Libre}$\\
$S_{25} = X_{24}\cdot \mathtt{EMPf}$\\[\sepsGrups]
%
% Etapas
$X_{20} = S_{20} + X_{20}\cdot \Not{S_{21}}$\\
$X_{21} = S_{21} + X_{21}\cdot \Not{S_{22}}$\\
$X_{22} = S_{22} + X_{22}\cdot \Not{S_{23}}$\\
$X_{23} = S_{23} + X_{23}\cdot \Not{S_{24}}$\\
$X_{24} = S_{24} + X_{24}\cdot \Not{S_{25}}$\\
$X_{25} = S_{25} + X_{25}\cdot \Not{S_{20}}$\\[\sepsGrups]
%
% Marcas
$\mathtt{EMP\_Libre} = X_{20}$\\
$\mathtt{Act\_Fresadora} = X_{24}$\\
$\mathtt{Pieza\_REC\_EMP} = X_{22}$\\[\sepsGrups]
%
% Salidas
$\mathtt{QEBW} = X_{25} + \mathtt{Man\_act\_BW}$\\
$\mathtt{QEFW} = X_{24} + \mathtt{Man\_act\_FW}$\\

\subsection{Grafcet de la fresadora}\label
Este grafcet realiza el control de la parte de la fresa del proyecto, donde siguie los siguientes pasos:

\begin{itemize}
	\item Avisamos que estamos libres para el empujador y poder recibir la pieza
	\item Una vez el empujador activa la fresadora porque esta le ha dicho que esta libre, activamos la cinta 
	de la fresadora, hasta que la pieza llegue al sensor ''FRESA'', si tarda más de 4 segundos salta una alarma que para la máquina
	\item Una vez llega la pieza al sensor, durante 3 segundos activamos la fresa, y pasamos a una etapa de espera, hasta que el taladro se encuentre libre
	\item Si el taladro nos indica que esta libre, activamos nuestra cinta, y la del taladro hasta que la pieza llegue al sensor del taladro,
	si tarda más de 5 segundos salará una alarma
\end{itemize}

\begin{figure}[H]
	\centering
	\scalebox{0.75}{%\nodeDist = 2.5cm
%\retornoDist = 0.7\nodeDist % Distància vertical dels retorns
%\bifDistX = 10ex
%\bifDistY = 1.1\nodeDist
%\sincDistYabove = 0.85\nodeDist
%\sincDistYbelow = 0.6\sincDistYabove
%\sincDistYBlockbelow = 0.5\nodeDist


\begin{tikzpicture}[auto]
	
	\ttfamily
	
% --------------------------------------------
	\etapaInicial{30}
	\primeraAccion{30}{X30-A1}{Fresadora\_Libre}
	%---
	\etapa[1.25\nodeDist]{31}{30}{Act\_Fresadora};
	\primeraAccion{31}{X31-A1}{QCINF}
	\otraAccion{X31-A1}{X31-A2}{Alarma\_Fresadora\_1}
	\condicionada{X31-A2}{3s/x31}
	

	%---
	\etapa[1.25\nodeDist]{32}{31}{FRESA};
	\primeraAccion{32}{X32-A1}{QFresadora}
	
	
	%---
	\etapa[1.25\nodeDist]{33}{32}{3s/x32};
	\comentari[3]{X33}{Esperando taladro libre}
	
	
	%---
	\etapa[1.25\nodeDist]{34}{33}{Taladro\_Libre};
	\primeraAccion{34}{X34-A1}{QCINF}
	\otraAccion{X34-A1}{X34-A2}{Act\_CIN\_Taladro}
	\otraAccion{X34-A2}{X34-A3}{Alarma\_Fresadora\_2}
	\condicionada{X34-A3}{3s/x34}
	\retornoInicio[6em]{34}{0}{Pieza\_REC\_Taladro}




	

            
\end{tikzpicture}}
	\medskip
	\label{graf:fresadora}
	\caption{Grafcet de la fresadora}
\end{figure}

Este grafcet está descompuesto en las siguientes ecuaciones algebráicas:\\
% Transiciones
$S_{30} = \mathtt{FirstScan} + X_{34}\cdot \mathtt{Pieza\_REC\_Taladro}$\\
$S_{31} = X_{30}\cdot \mathtt{Act\_Fresadora}$\\
$S_{32} = X_{31}\cdot \mathtt{FRESA\_M}$\\
$S_{33} = X_{32}\cdot \mathtt{T1.Q}$\\
$S_{34} = X_{33}\cdot \mathtt{Taladro\_Libre}$\\
%
% Etapas
$X_{30} = S_{30} + X_{30}\cdot \Not{S_{31}}$\\
$X_{31} = S_{31} + X_{31}\cdot \Not{S_{32}}$\\
$X_{32} = S_{32} + X_{32}\cdot \Not{S_{33}}$\\
$X_{33} = S_{33} + X_{33}\cdot \Not{S_{34}}$\\
$X_{34} = S_{34} + X_{34}\cdot \Not{S_{30}}$\\
%
% Marcas
$\mathtt{Fresadora\_Libre} = X_{30}$\\
$\mathtt{QCINF} = X_{31} + X_{34} + \mathtt{man\_act\_cin\_fres}$\\
$\mathtt{Act\_CIN\_Taladro} = X_{34}$\\
$\mathtt{QFresadora} = X_{32} + \mathtt{man\_Act\_fres}$\\



\subsection{Grafcet del taladro} \label{sec:graf_talad}

El siguiente grafcet controla la parte del proyecto del taladro, siguiendo los siguientes pasos:
\begin{itemize}
	\item Avismos que estamos libres, y esperamos a que nos active la fresadora como hemos dicho antes (\ref{sec:graf:fresa})
	\item Una vez nos activa la fresadora, activamos la cinta del taladro hasta que se alcance el sensor ''TALAD'', donde
	activamos el taladro, y avisamos a la fresadora que ya tenemos la cinta.
	\item Durante 2 segundos taladraremos la pieza, y nos quedaremos a la espera de que el empujador 2 se encuentre libre
	para poder enviarle la pieza
	\item Si nos indica que se encuentra libre, activaremos nuestra cinta, hasta que el taladro
	nos indique que se encuentra libre
\end{itemize}


\begin{figure}[H]
	\centering
	\scalebox{0.75}{%\nodeDist = 2.5cm
%\retornoDist = 0.7\nodeDist % Distància vertical dels retorns
%\bifDistX = 10ex
%\bifDistY = 1.1\nodeDist
%\sincDistYabove = 0.85\nodeDist
%\sincDistYbelow = 0.6\sincDistYabove
%\sincDistYBlockbelow = 0.5\nodeDist


\begin{tikzpicture}[auto]
	
	\ttfamily
	
% --------------------------------------------
	\etapaInicial{40}
	\primeraAccion{40}{X40-A1}{Taladro\_Libre}
	%---
	\etapa[1.25\nodeDist]{41}{40}{Act\_CIN\_Fresadora};
	\primeraAccion{41}{X41-A1}{QCINT}
	\otraAccion{X41-A1}{X41-A2}{Alarma\_Taladro\_1}
	\condicionada{X41-A2}{3s/x41}
	%---
	\etapa[1.25\nodeDist]{42}{41}{TALAD};
	\primeraAccion{42}{X42-A1}{QTALAD}
	\otraAccion{X42-A1}{X42-A2}{P\_en\_Talad}
	%---
	\etapa[1.25\nodeDist]{43}{42}{2s/x42};
	\comentari[2]{X43}{Esperando a empujador 2 libre}


	\etapa[1.25\nodeDist]{43b}{43}{EMP2\_Libre};
	\primeraAccion{43b}{X43b-A1}{QCINT}
	\otraAccion{X43b-A1}{X43b-A2}{Alarma\_Taladro\_2}
	\condicionada{X43b-A2}{3s/x43b}

	\etapa[1.25\nodeDist]{44}{43b}{\NOT{TALAD}};
	{%
  	\setlength{\actionMinimumWidth}{4.5cm}% <- tu ancho
  	\primeraAccion{44}{X44-A1}{QCINT}%
	}
	\condicionada{X44-A1}{\NOT{Alarma\_EMP2}$\,\cdot\,$\NOT{TALAD}}
	\otraAccion{X44-A1}{X44-A2}{Alarma\_Taladro\_3}
	\condicionada{X44-A2}{3s/x44}
	\otraAccion{X44-A2}{X44-A3}{Act\_EMP2}

	%---
	\etapa[1.25\nodeDist]{45}{44}{Pieza\_REC\_EMP2};
	\primeraAccion{45}{X45-A1}{P\_Entregada\_EMP2}
	\retornoInicio[6em]{45}{0}{\underline{1}}
	








	




	

            
\end{tikzpicture}}
	\medskip
	\label{graf:taladro}
	\caption{Grafcet del taladro}
\end{figure}

Este grafcet está descompuesto en las siguientes ecuaciones algebráicas:\\
% Transiciones
$S_{40} = \mathtt{FirstScan} + X_{45}$\\
$S_{41} = X_{40}\cdot \mathtt{Act\_CIN\_Taladro}$\\
$S_{42} = X_{41}\cdot \mathtt{TALAD\_M}$\\
$S_{42b} = X_{42}\cdot \mathtt{T1.Q}$\\
$S_{43} = X_{42b}\cdot \mathtt{EMP\_LIBRE}$\\
$S_{44} = X_{43}\cdot \Not{\mathtt{TALAD\_M}}$\\
$S_{45} = X_{44}\cdot \mathtt{Pieza\_Rec\_emp}$\\
%
% Etapas
$X_{40} = S_{40} + X_{40}\cdot \Not{S_{41}}$\\
$X_{41} = S_{41} + X_{41}\cdot \Not{S_{42}}$\\
$X_{42} = S_{42} + X_{42}\cdot \Not{S_{42b}}$\\
$X_{42b} = S_{42b} + X_{42b}\cdot \Not{S_{43}}$\\
$X_{43} = S_{43} + X_{43}\cdot \Not{S_{44}}$\\
$X_{44} = S_{44} + X_{44}\cdot \Not{S_{45}}$\\
$X_{45} = S_{45} + X_{45}\cdot \Not{S_{40}}$\\
%
% Marcas
$\mathtt{Taladro\_Libre} = X_{40}$\\
$\mathtt{Pieza\_rec\_taladro\_1} = X_{42}$\\
$\mathtt{QCINT} = X_{41} + X_{43} + X_{44} + \mathtt{Man\_Act\_cin}$\\
$\mathtt{QTALAD} = X_{42} + \mathtt{Man\_Act\_tal}$\\
$\mathtt{Act\_EMP} = X_{44}$\\
$\mathtt{P\_Entregada\_EMP} = X_{45}$\\

\subsection{Grafcet de la cinta final} \label{sec:graf_cin_fin}

El siguiente grafcet tiene la funcion de controlar la cinta del final, esta cinta esta compuesta solo 
por un sensor que nos indica que la pieza ha llegado al final de la cinta, y la salia de mover la cinta.
Los pasos que realiza este grafcet son los siguientes:
\begin{itemize}
	\item Avisamos que estamos libres, y esperamos a que el empujador 2 nos avise para activar la cinta hasta
	alcanzar el sensor ''CINF'', donde nos quedamos esperando hasta que nos retiren la pieza, donde avisaremos
	al grafcet principal  para que la cuente en el contador que hemos iniciado
	\item Si tardan más de 5 segundos en retirar la pieza saltará una alarma.
\end{itemize}

\begin{figure}[H]
	\centering
	\scalebox{0.75}{%\nodeDist = 2.5cm
%\retornoDist = 0.7\nodeDist % Distància vertical dels retorns
%\bifDistX = 10ex
%\bifDistY = 1.1\nodeDist
%\sincDistYabove = 0.85\nodeDist
%\sincDistYbelow = 0.6\sincDistYabove
%\sincDistYBlockbelow = 0.5\nodeDist


\begin{tikzpicture}[auto]
	
	\ttfamily
	
	\etapaInicial{60}
	\primeraAccion{60}{X60-A1}{Cinta\_FIN\_Libre}
	%---
	\etapa[1.25\nodeDist]{61}{60}{Act\_Cinta\_Fin};
	\primeraAccion{61}{X61-A1}{QCINS}
	\otraAccion{X61-A1}{X61-A2}{Alarma\_CIN\_FIN\_1}
	\condicionada{X61-A2}{5s/x61}
	%--
	\etapa[1.25\nodeDist]{62}{61}{CINF};
	\comentari[3]{X62}{Esperando a recoger pieza}
	%---
	\etapa[1.25\nodeDist]{63}{62}{\NOT{CINF}};
	\primeraAccion{63}{X63-A1}{Pieza\_Entregada}
	\retornoInicio[6em]{63}{0}{\underline{1}}











	




	

            
\end{tikzpicture}}
	\medskip
	\caption{Grafcet de la cinta final}
	\label{graf:cin_fin}
\end{figure}

Este grafcet está descompuesto en las siguientes ecuaciones algebráicas:\\
% Transiciones
$S_{60} = \mathtt{FirstScan} + X_{63}$\\
$S_{61} = X_{60}\cdot \mathtt{Activar\_Cinta\_FIN}$\\
$S_{62} = X_{61}\cdot \mathtt{CINF\_M}$\\
$S_{63} = X_{62}\cdot \Not{\mathtt{CINF\_M}}$\\
%
% Etapas
$X_{60} = S_{60} + X_{60}\cdot \Not{S_{61}}$\\
$X_{61} = S_{61} + X_{61}\cdot \Not{S_{62}}$\\
$X_{62} = S_{62} + X_{62}\cdot \Not{S_{63}}$\\
$X_{63} = S_{63} + X_{63}\cdot \Not{S_{60}}$\\
%
% Marcas
$\mathtt{Cinta\_FIN\_Libre} = X_{60}$\\
$\mathtt{Pieza\_Entregada} = X_{63}$\\
$\mathtt{salida\_Alarma\_Cinta\_fin} = \mathtt{Alarmas\_Cinta\_Fin\_1}$\\
%
% Salidas físicas
$\mathtt{QCINS} = X_{61} + \mathtt{man\_Act\_cinf}$\\




\subsection{Grafcet de mantenimiento}\label{sec:graf_mant}

Este grafcet se encarga de mover todos los componentes que componen el proyecto, 
dividido en dos ramas. Antes de inciar el mantenimiento se comprueba que no hayan piezas en cualquiera de los sensores.

\begin{itemize}
	\item Una rama mueve los empujadores hasta los dos finales de carrera 
	\item La otra rama, lo que hace primero es mover las 4 cintas que componen el proyecto durante 5 segundos, y despues activa el taladro y la 
	fresadora durante 5 segundos, para una vez finalizadas las dos ramas indicar que se ha acabado
\end{itemize} 

\begin{figure}[H]
	\centering
	\scalebox{0.75}{%\nodeDist = 2.5cm
%\retornoDist = 0.7\nodeDist % Distància vertical dels retorns
%\bifDistX = 10ex
%\bifDistY = 1.1\nodeDist
%\sincDistYabove = 0.85\nodeDist
%\sincDistYbelow = 0.6\sincDistYabove
%\sincDistYBlockbelow = 0.5\nodeDist


\begin{tikzpicture}[auto]
	
	\ttfamily
	
	\etapaInicial{70}
	\etapa[1.25\nodeDist]{70b}{70}{Act\_Mantenimiento};
	%---
	\sincronizacion[3cm]{70b}{\NOT{CINAI}$\,\cdot\,$\NOT{CINAF}$\,\cdot\,$\NOT{FRESA}$\,\cdot\,$\NOT{TALAD}$\,\cdot\,$\NOT{CINF}}{71a}{72a}
	%--71a
	\primeraAccion{71a}{X71a-A1}{QE1FW}
	\condicionada{X71a-A1}{\NOT{EMP1F}}
	\otraAccion{X71a-A1}{X71a-A2}{QE2FW}
	\condicionada{X71a-A2}{\NOT{EMP2F}}

	%---72a
	\primeraAccion{72a}{X72a-A1}{QCINA}
	\condicionada{X72a-A1}{\NOT{CINAI}$\,\cdot\,$\NOT{CINAF}}
	\otraAccion{X72a-A1}{X72a-A2}{QCINF}
	\condicionada{X72a-A2}{\NOT{FRESA}}
	\otraAccion{X72a-A2}{X72a-A3}{QCINT}
	\condicionada{X72a-A3}{\NOT{TALAD}}
	\otraAccion{X72a-A3}{X72a-A4}{QCINS}
	\condicionada{X72a-A4}{\NOT{CINF}}


	%---
	\etapa[1.25\nodeDist]{71b}{71a}{EMP1F$\,\cdot\,$EMP2F};
	\primeraAccion{71b}{X71b-A1}{QE1BW}
	\condicionada{X71b-A1}{\NOT{EMP1I}}
	\otraAccion{X71b-A1}{X71b-A2}{QE2BW}
	\condicionada{X71b-A2}{\NOT{EMP2I}}


	
	\etapa[1.25\nodeDist]{71c}{71b}{EMP1L$\,\cdot\,$EMP2L};

	%---
	\etapa[1.25\nodeDist]{72b}{72a}{5s/x72a};
	\primeraAccion{72b}{X72b-A1}{QFresadora}
	\condicionada{X72b-A1}{\NOT{FRESA}}
	\otraAccion{X72b-A1}{X72b-A2}{QTalad}
	\condicionada{X72b-A2}{\NOT{TALAD}}

	\etapa[1.25\nodeDist]{72c}{72b}{5s/x72b};

	%----
	\concurrencia{71c}{72c}{\underline{1}}{73}
	%\etapa[1.25\nodeDist]{63}{62}{\NOT{CINF}};
	\primeraAccion{73}{X73-A1}{Mantenimiento\_FIN}
	\retornoInicio[13em]{73}{0}{\underline{1}}
	













	




	

            
\end{tikzpicture}}
	\medskip
	\caption{Grafcet de Mantenimiento}
	\label{graf:mantenimiento}
\end{figure}




\subsection{Grafcet de emergencia}\label{sec:graf_emerg}

\subsubsection{Grafcet de emergencia general}
Este grafcet lo tenemos implementado en el PLC para las 6 máquinas que componen el proyecto en FBs dentro del código,
como hemos visto, cada grafcet activa unas alarmas, lo que hacemos en este grafcet es gestionarlas,
lo primero que hacemos es pausar la máquina, una vez pausada, y con los actuadores desactivados, esperamos
a que el operario nos aprete ''RESUME'' si queremos que siga el proceso, o ''RESET'' si queremos que
empieze de nuevo esa máquina.

\begin{figure}[H]
	\centering
	\scalebox{0.75}{\begin{tikzpicture}
  \ttfamily

  %---------------------------
  % Columna izquierda
  %---------------------------
  \etapaInicial{120}
  \etapa[1.25\nodeDist]{121}{120}{VACIAR\_MAQS};
  \primeraAccion{121}{X121-A1}{Activar\_Vaciado\_Maquinas}
  
 
  
  
  \etapa[1.25\nodeDist]{122}{121}{Fin\_VAC\_VAC\_1 $\cdot$ Fin\_VAC\_EMB\_1 $\cdot$ Fin\_VAC\_SOP\_1 $\cdot$ Fin\_VAC\_VAC\_2 $\cdot$ Fin\_VAC\_EMB\_2 $\cdot$ Fin\_VAC\_SOP\_2 };
 	\primeraAccion{122}{X122-A1}{Fin\_Vaciado}
  
    \etapa[1.25\nodeDist]{122V2}{122}{ $\cdot$ Fin\_VAC\_VAC\_3 $\cdot$ Fin\_VAC\_EMB\_3 $\cdot$ Fin\_VAC\_SOP\_3 $\cdot$ Fin\_VAC\_SOP\_4};
 	\primeraAccion{122V2}{X122V2-A1}{Fin\_Vaciado}

  \retornoInicio[6cm]{122V2}{120}{\underline{1}}






  


\end{tikzpicture}
}
	\medskip
	\caption{Grafcet de emergencia en guia GEMMA}
	\label{graf:emerg2}
\end{figure}


Este grafcet se ha implementado de tres maneras en nuestra solucion, los empujadores utilizan un bloque 
a parte donde siguen el grafcet de arriba mediante etapas, mientras que en el resto de máquinas se ha implementado el uso 
de IF-THEN-ELSE que implementa SCL.\\
En el Anexo del código de alarmas podemos ver las etapas implementadas 
con sus transiciones, y como activan las marcas que se comunican con el bloque de los empujadores para bloquear la máquina 
con el siguiente fragmento\\
Podemos ver en el siguiente fragmento como cuando recibimos un flanco de 
la variable Alarma que es la variable que leemos del bloque de alarma guardamos el estado actual
de la máquina en unas variables específicas, y que cuando hay un flanco de subida del botón ''RESUMEN'' se devuelven ese valor
, mientras que si hay un  ''RESET'' se inicializa la etapa principal solamente, las salidas no hace falta ponerlas a ''FALSE'' ya 
que no están realimentadas, solo se activan si la etapa esta activa, pero como las ponemos a ''FALSE'' se detiene todo.
\lstinputlisting[
	label=frag:1,
  firstline=89,
  lastline=122,
  caption={Tratamiento de la alarma en empujadores}
]

En las otras máquinas se ha implementado en el mismo bloque, pero siguiendo la estructura anterior de bloques ''IF-THEN'' como vemos en el
código de la cinta de inicio por ejemplo
\lstinputlisting[
	label=frag:2,
  firstline=155,
  lastline=199,
  caption={Tratamiento de la alarma en la cinta inicial}
]

La tercera solucion se ha implementado unicamene en el subapartado de la cinta final, y la diferencia respecto a 
las anteriores es que se eliminan los bloques ''IF-THEN-ELSE'' para el tratamiento de las alarmas,
y lo que se bloquea es la activacion de las etapas y actuadores como vemos en el siguiente fragmento de texto(\ref{frag:3}), en 
la cinta final solo se trata una alarma que como vemos en el grafcet (\ref{graf:cin_fin}) es que después del empujador la pieza
no alcance el sensor del final de la cinta
\lstinputlisting[
	label=frag:3,
  firstline=147,
  lastline=162,
  caption={Tratamiento de las alarmas en la cinta final}
] 
Como vemos, cuando surge una alarma, se bloquean los actuadores en la primera etapa como vemos en el fragmento (\ref{frag:4}), y tambien se activa la variable ''FORCE\_CURRENT'' 
la cual bloquea las transiciones de las etapas para que estas no se puedan activar, pero manteniendo el valor de las etapas a no ser que apretemos ''RESET'' que 
se activará la variable ''FORCE\_INIT'' que desactiva todas las etapas y activa solo la primera.
\lstinputlisting[
label=frag:4,
  firstline=100,
  lastline=110,
  caption={Bloqueo y reseteo de las etapas cuando hay una alarma}
]


%\subsubsection{Grafcet de emergencia guia GEMMA}

%En este grafcet nos hemos basado en la guia gemma para tratar las Alarmas
%de la cinta final, lo que hacemos es que cuando detectamos 
%una alarma, detenemos los actuadores,y bloqueamos el grafcet
%, y nos quedamos en espera de que pulsen el botón de ''Reset'' o ''Resume'',
% para empezar de nuevo el proceso de esa máquina, seguir el funcionamiento normal.





\subsection{Grafcet de la seta de emergencia}

\begin{comment}
	

\subsubsection{Tratamiento de la seta}
Cuando se apreta la ''SETA'' el primer grafcet que entra en ejecución es el siguiente(\ref(graf:seta2)), en el cual después de 
activar la ''SETA'' activamos una variable, que es compartida con todos los grafcets de seta de cada máquina(\ref{graf:seta1}), y 
se quedará activada hasta que la desactiven, una vez desactivada avisara que se ha desactivado, pero esta señal no la recibe nadie se
deja porque en un SCADA se podria visualizar, y esperamos a que apreten ''RESET'' donde activamos el grafcet de mantenimiento, para poder
comprobar que las máquinas vuelven a funcionar, y una vez se haya finalizado, avisamos al grafcet(\ref{graf:seta1}) que tiene a todos parados, para que devuelva el 
funcionamiento desde el inicio.


\begin{figure}[H]
	\centering
	\scalebox{0.75}{%\nodeDist = 2.5cm
%\retornoDist = 0.7\nodeDist % Distància vertical dels retorns
%\bifDistX = 10ex
%\bifDistY = 1.1\nodeDist
%\sincDistYabove = 0.85\nodeDist
%\sincDistYbelow = 0.6\sincDistYabove
%\sincDistYBlockbelow = 0.5\nodeDist


\begin{tikzpicture}[auto]
	
	\ttfamily
	
% --------------------------------------------
	\ttfamily
	
	\etapaInicial{100}
	\primeraAccion{100}{X100-A1}{MAQ\_LIBRE}
    \otraAccion{X100-A1}{X100-A2}{$\mathtt{C:=0}$}
  \sobreActivacion{X100-A2}{X100-A3}{}
	%---

	\etapa[1.25\nodeDist]{101}{100}{Act\_Maq};
	\primeraAccion{101}{X101-A1}{QC7}
	
	%--
	\etapa[1.25\nodeDist]{102}{101}{S5};
	\primeraAccion{102}{X102-A1}{QC7}

	%---
	\etapa[1.25\nodeDist]{103}{102}{1s/X102};
	\primeraAccion{103}{X103-A1}{QCLAMP}

	\etapa[1.25\nodeDist]{104}{103}{1s/X103};
	\primeraAccion{104}{X104-A1}{HAYSPieza}

  \etapa[1.25\nodeDist]{105}{104}{Pieza\_Empaquetada};
  \primeraAccion{105}{X105-A1}{$\mathtt{C:=C+1}$}
  \sobreActivacion{X105-A1}{X105-A2}{}

  %\retornoInicio[-10em]{105}{104}{$\mathtt{C<2}$}
  %\etapa[1.25\nodeDist]{104}{105}{$\mathtt{C<2}$}
  \saltoHorizontal[5.5cm]{105}{104}{$\mathtt{C<2}$}
  
  \etapa[1.4\nodeDist]{106}{105}{$\mathtt{C==1}$};
	\primeraAccion{106}{X106-A1}{QCLAMP\_SUB}
	\otraAccion{X106-A1}{X106-A2}{QC7}

  \etapa[1.25\nodeDist]{107}{106}{3s/X106};
	\primeraAccion{107}{X107-A1}{Pieza\_Entregada}
	
	

	\retornoInicio[6em]{107}{0}{\underline{1}}
	
       
\end{tikzpicture}}
	\medskip
	\caption{Grafcet de tratamiento de la seta}
	\label{graf:seta2}
\end{figure}

\subsubsection{Tratamiento de la seta en cada máquina}
Como hemos dicho antes, cuando nos pulsan la ''SETA'' nos avisan a través de una variable que tenemos que detener el proceso, lo que hacemos es devolver todos los 
grafcets a su etapa inicial y después los congelamos, hasta que nos vuelvan a avisar de que se ha desactivado la ''SETA'' y se ha acabado el mantenimiento

\begin{figure}[H]
	\centering
	\scalebox{0.75}{\input{./grafcets/9_Seta.tex}}
	\medskip
	\caption{Grafcet de la seta de emergencia}
	\label{graf:seta1}
\end{figure}
\end{comment}


En nuestra solución el tratamiento de la seta se ha realizado a través del siguiente grafcet(\ref{graf:seta3}) diseñado
a partir de la guia GEMMA, en el que al apretar la ''SETA'' hacemos que el proceso se vaya a la etapa inicial
seguidamente se bloque en esa etapa a la espera de que acabe mantenimiento, indicando que se ha acabado de tratar la alarma,
pero el mantenimiento solo sera activado con el botón ''Reset'' que con pulsarlo una unica vez sirve para que 
entre en funcionamiento el grafcet como hemos indiado anteriormente(\ref{sec:graf_mant}).

\begin{figure}[H]
	\centering
	\scalebox{0.75}{%\nodeDist = 2.5cm
%\retornoDist = 0.7\nodeDist % Distància vertical dels retorns
%\bifDistX = 10ex
%\bifDistY = 1.1\nodeDist
%\sincDistYabove = 0.85\nodeDist
%\sincDistYbelow = 0.6\sincDistYabove
%\sincDistYBlockbelow = 0.5\nodeDist


\begin{tikzpicture}[auto]
	
	\ttfamily
	
	\etapaInicial{110}
	
	%---
	\etapa[1.25\nodeDist]{111}{110}{Act\_Mantenimiento};

	
	%--
	\etapa[1.25\nodeDist]{112}{111}{\NOT{SENSORES}};
	\primeraAccion{112}{X112-A1}{Activar\_Cinta}
	
	%---
	\etapa[1.25\nodeDist]{113}{112}{2s/X112};
	\primeraAccion{113}{X113-A1}{Activar\_Transfers}

	\etapa[1.25\nodeDist]{114}{113}{2s/X113};
	\primeraAccion{114}{X114-A1}{Bajar\_Robot}
  \otraAccion{X114-A1}{X114-A2}{Subir\_CLAMPER}

	\etapa[1.25\nodeDist]{115}{114}{2s/X114};
	\primeraAccion{115}{X115-A1}{Grippers}
  \otraAccion{X115-A1}{X115-A2}{CLAMPER}

	\etapa[1.25\nodeDist]{116}{115}{2s/X115};
	\primeraAccion{116}{X116-A1}{Mover\_Horizontal\_Robots}

  \etapa[1.25\nodeDist]{117}{116}{2s/X116};
	\primeraAccion{117}{X117-A1}{Fin\_Mantenimiento}

	\retornoInicio[6em]{117}{0}{\underline{1}}

\end{tikzpicture}
}
	\medskip
	\caption{Grafcet de la seta de emergencia en guia GEMMA}
	\label{graf:seta3}
\end{figure}

Este grafcet se ha implementado de tres maneras distinas en nuestra solución, una
para los empujadores, otra para la fresadora, el taladro y la cinta inical, y una ultima 
para la cinta final sin el uso del bloque ''IF-THEN-ELSE'' similar al de las alarmas explicado anteriormente (\ref{sec:graf_emerg})

\subsubsection{Tratamiento de la seta en los empujadores}
En los empujadores lo hemos implementado en el bloque de las alarmas, y aprovechando
que ya tienen una variable donde se le avisa al empujador que detenga todo, lo que hacemos
es activarla cuando pulsamos la ''SETA'', por lo que bloqueamos el grafcet de los empujadores, y
dentro del bloque de las alarmas, bloqueamos las etapas de las alarmas, para solo poder 
trabajar con el botón ''RESET'' después de activar la ''SETA''.

\lstinputlisting[
label=frag:5,
  firstline=75,
  lastline=113,
  caption={Tratamiento de la SETA para los empujadores}
] 

\subsubsection{Tratamiento de la seta con bloque ''IF-THEN''}
En la cinta de inicio, el taladro y la fresadora se ha implementado dentro del 
propio bloque de control, como vemos en el fragmento de codigo , cuando se produce 
un flanco de subida de la ''SETA'', congelamos todo, ponemos todas las etapas a ''FALSE'', y activamos
una variable que nos indica que tenemos la seta activa, y bloquea todas las transiciones y etapas del proceso, y
hasta que no se active el botón de ''RESET'' no de deuvelve al estado normal de funcionamiento
\lstinputlisting[
label=frag:6,
  firstline=202,
  lastline=259,
  caption={Tratamiento de la SETA para la cinta inicial, fresadora y taladro}
] 

\subsubsection{Tratamiento de la seta en la cinta final}
En la cinta final hemos optado por tratar la seta de emergencia sin bloques ''IF-THEN'', similiar a como
lo hemos implementado en las alaramas , gestionamos la ''SETA'' con 3 etapas, donde en el codigo que vemos 
abajo , con dos variables forzamos seguir y el reinicio de la máquina, también activamos la variable
de mantenimiento, porque una vez se ha reiniciado la ''SETA'' activamos el mantenimiento para revisar que todo vuelve
a funcionar normal antes de empezar.
\lstinputlisting[
label=frag:7,
  firstline=134,
  lastline=145,
  caption={Tratamiento de la SETA para la cinta final}
]


