\section{Estructura TIA Portal}\label{sec:estructura_codigos_scl}
Respecto a la estructura del bloque main del proyecto en TIA Portal, se ha seguido una metodología de 
dividir el bloque main en varias ramas, donde cada rama corresponde a una línea o función del sistema.


\subsection{Línea 1}
En la línea 0 se encuentra el grafcet principal del sistema, que se encarga de coordinar el funcionamiento de todas las máquinas y líneas del sistema.
\begin{figure}[H]
	\centering
	\includegraphics[width=0.8\linewidth]{./Figuras/linea1.png}
	\caption{Estructura de la Línea 1}
	\label{fig:linea1}
\end{figure}



\subsection{Línea 2}
En la línea 2 vemos el bloque que corresponde con el modo de mantenimiento.

\begin{figure}[H]
	\centering
	\includegraphics[width=0.8\linewidth]{./Figuras/linea2.png}
	\caption{Estructura de la Línea 2}
	\label{fig:linea2}
\end{figure}


\subsection{Línea 3}
En esta línea podemos encontrar el modo manual del sistema.

\begin{figure}[H]
	\centering
	\includegraphics[width=0.8\linewidth]{./Figuras/linea3.png}
	\caption{Estructura de la Línea 3}
	\label{fig:linea3}
\end{figure}


\subsection{Línea 4}
En la siguiente línea encontramos el bloque que gestiona el vaciado general del sistema,
y más ramas donde cada una controla el vaciado de cada línea de producción.

\begin{figure}[H]
	\centering
	\includegraphics[width=0.6\linewidth]{./Figuras/linea4.png}
	\caption{Estructura de la Línea 4}
	\label{fig:linea4}
\end{figure}

\begin{figure}[H]
	\centering
	\includegraphics[width=0.8\linewidth]{./Figuras/linea4b.png}
	\caption{Estructura de la Línea 4}
	\label{fig:linea4b}
\end{figure}


\subsection{Línea 5}
Aquí encontramos todos los bloques que controlan la línea de las vacunas.
\begin{figure}[H]
	\centering
	\includegraphics[width=0.8\linewidth]{./Figuras/linea5.png}
	\caption{Estructura de la Línea 5}
	\label{fig:linea5}
\end{figure}


\subsection{Línea 6}
Aquí encontramos todos los bloques que controlan la línea de los soportes.
\begin{figure}[H]
	\centering
	\includegraphics[width=0.8\linewidth]{./Figuras/linea6.png}
	\caption{Estructura de la Línea 6}
	\label{fig:linea6}
\end{figure}


\subsection{Línea 7}
Aquí encontramos todos los bloques que controlan la línea de embalaje.
\begin{figure}[H]
	\centering
	\includegraphics[width=0.8\linewidth]{./Figuras/linea7.png}
	\caption{Estructura de la Línea 7}
	\label{fig:linea7}
\end{figure}





%Para realizar este proyecto, se ha seguido la siguiente metodología.
