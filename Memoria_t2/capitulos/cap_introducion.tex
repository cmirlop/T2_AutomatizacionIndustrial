\section{Introducción}
\label{sec:introduccion}

Se nos ha propuesto el problema de diseñar y programar un proceso de clasificación
de vacunas, soportes y su embalaje correspondiente, dentro de este proceso aparecen
diferentes tipos de cada uno, que deberemos de separar a 
su propia línea de producción. Tambien debemos de hacer un control de calidad de 
las piezas que se corresponden con el tipo elegido, y descartar las piezas que no
cumplen con los requisitos establecidos.

\subsection{Propuesta para el proceso}
Para el proceso de clasificación, proponemos la separción en 3 líneas de producción, Por
donde circualarán las vacunas, los soportes y los embalajes. Cada línea de producción 
esta compuesta por los procesos necesarios para poder separar los diferentes tipos, y realizar su control
de calidad. La idea es que una vez superado el control de calidad, las vacunas se dispondrán en su soporte,
y una vez esta en su soporte, se procederá a su embalaje.

Como se puede observar en la figura \ref{fig:proceso_general}, el proceso comienza con la llegada
de las piezas a la línea de producción, donde se realizará una primera separación de las piezas que són del 
tipo A, y las que son del tipo B, para despúes realizar un control de calidad de las piezas, y descartar
las que no cumplen con los requisitos establecidos. Una vez superado el control de calidad, las piezas
se dirigen a la siguiente estación, donde se realizará la unión de las vacunas con sus respectivos
soportes, y una vez en su soporte, se procederá a su embalaje.

\begin{figure}[H]
     \centering
     \includegraphics[width=0.8\linewidth]{./Figuras/General.png}
	\medskip
	\caption{Proceso general}
	\label{fig:proceso_general}
 \end{figure}




\subsection{Propuesta para la interacción del operario}
Para la interacción del operario con la máquina, se propone el uso de un HMI , y de un panel físico
para acciónes criticas como el paro de emergencia, y su rearme, a parte de la marcha y el paro de la máquina.
Como se puede observar en la figura \ref{fig:panel_fisico}, el panel físico cuenta con los botones necesarios
para:
\begin{itemize}
    \item La marcha de la máquina
    \item El paro de la máquina
    \item El paro de emergencia
    \item El reset del paro de emergencia o de las alarmas
    \item Selección del modo de operación (manual/automático)
\end{itemize}

\begin{figure}[H]
     \centering
     \includegraphics[width=0.5\linewidth]{./Figuras/PanelFisico.png}
    \medskip
    \caption{Panel físico de control}
    \label{fig:panel_fisico}
 \end{figure}

El HMI contará con diversas pantallas para el control y la monitorización de la máquina, que se explican
en la sección \ref{sec:HMI}. Desde el HMI se podrán realizar las siguientes acciónes:
\begin{itemize}
    \item Poner en marcha y parar la máquina
    \item Reanudar la máquina tras una alarma desde donde se ha producido la alarma
    \item Acceder a la pantalla de configuración (con credenciales de administrador)
    \item Acceder a la pantalla de registro de alarmas
    \item Controlar los actuadores en modo manual
    \item Visualizar los contadores de piezas producidas y descartadas
    \item Visualizar el estado de los actuadores
\end{itemize}

