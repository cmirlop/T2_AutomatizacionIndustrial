\subsection{Grafcet Máquina 5 soporte}\label{sec:map5_sop}

En la siguiente fígura (\hyperref[fig:maq5]{Máquina 5 soporte}) vemos los siguientes elementos:
\begin{itemize}
	\item 3 Cintas
	\item 1 Clamper
	\item 1 Sensor
\end{itemize}

\begin{figure}[H]
	\centering
	\includegraphics[width=0.8\linewidth]{./Figuras/maq5_SOP.png}
	\medskip
	\label{fig:maq5}
	\caption{Máquina 5 de la línea de soportes}
\end{figure}

El funcionamento de la máquina como vemos en el grafcet es el siguiente:

Activamos las 3 cintas, cuando se produce un flanco de bajada en el sensor que hay
antes del clamper ponemos un temporizador de 1 segundo para que la pieza entre dentro del clamper.
Activamos el clamper, esto lo que nos hace es colocar la pieza en el sitio, para que el 
robot al colocar la vacuna sobre el soporte, este no falle. Una vez tenemos la pieza
colocada avisamos al robot de las vacunas que hay un soporte en el sitio, este aviso lo recibimos en el 
grafcet que hablaremos más adelante (\hyperref[graf:maq5]{Grafcet Máquina 5 soporte}), y que nos colocará la pieza en su sitio. Una vez nos diga la 
otra máquina que tenemos la vacuna en el soporte, nos esperaremos a que la siguiente máquina nos indique 
que esté libre para poder levantar el clamper y activar las cintas.

Aquí la alarma que tratamos es que la pieza tarde más de 15 segundos en llegar al sensor.

\begin{figure}[H]
	\centering
	\scalebox{0.75}{%\nodeDist = 2.5cm
%\retornoDist = 0.7\nodeDist % Distància vertical dels retorns
%\bifDistX = 10ex
%\bifDistY = 1.1\nodeDist
%\sincDistYabove = 0.85\nodeDist
%\sincDistYbelow = 0.6\sincDistYabove
%\sincDistYBlockbelow = 0.5\nodeDist


\begin{tikzpicture}[auto]
	
	\ttfamily
	
	\etapaInicial{70}
	\primeraAccion{70}{X70-A1}{MAQ\_CAL\_VAC\_LIBRE}
	%---
	\etapa[1.25\nodeDist]{71}{70}{Act\_Maq};
	\primeraAccion{71}{X71-A1}{QC5\_VAC}
	\otraAccion{X71-A1}{X71-A2}{QC6\_VAC}
	\otraAccion{X71-A2}{X71-A3}{QC7\_VAC}
	
	%--
	\etapa[1.25\nodeDist]{72}{71}{S5};
	\primeraAccion{72}{X72-A1}{QC5\_VAC}
	\otraAccion{X72-A1}{X72-A2}{QC6\_VAC}
	\otraAccion{X72-A2}{X72-A3}{QC7\_VAC}
	%---
	\etapa[1.25\nodeDist]{73}{72}{1s/X72};
	\primeraAccion{73}{X73-A1}{QCLAMP}

	\etapa[1.25\nodeDist]{74}{73}{1s/X73};
	\primeraAccion{74}{X74-A1}{HAYSoporte}

	\etapa[1.25\nodeDist]{75}{74}{Pieza\_Empaquetada};

	\etapa[1.25\nodeDist]{76}{75}{SIG\_MAQ\_LIBRE};
	\primeraAccion{76}{X76-A1}{QCLAMP\_SUB}
	\otraAccion{X76-A1}{X76-A2}{QC7\_VAC}
	\otraAccion{X76-A2}{X76-A3}{Act\_SIG\_MAQ}

	\retornoInicio[6em]{76}{0}{\underline{1}}

\end{tikzpicture}}
	\medskip
	\label{graf:maq5}
	\caption{Grafcet de la Máquina 5 Soportes}
\end{figure}

Este grafcet está descompuesto en las siguientes ecuaciones algebráicas:\\

\setlength{\sepsGrups}{2ex}

% PRODUCCIÓN

% Transiciones (sin FORCE_CURRENT)
$S_{70} = \mathtt{FirstScan} + X_{76}\cdot \mathtt{T3.Q}$\\
$S_{71} = X_{70}\cdot \mathtt{Act\_Proceso}$\\
$S_{72} = X_{71}\cdot \mathtt{S5}$\\
$S_{73} = X_{72}\cdot \mathtt{T1.Q}$\\
$S_{74} = X_{73}\cdot \mathtt{T2.Q}$\\
$S_{75} = X_{74}\cdot \mathtt{Pieza\_Empaquetada}$\\
$S_{76} = X_{75}\cdot \mathtt{SigMaqLibre}$\\[\sepsGrups]

% Etapas (sin FORCE_INIT)
$X_{70} = S_{70} + X_{70}\cdot \Not{S_{71}}$\\
$X_{71} = S_{71} + X_{71}\cdot \Not{S_{72}}$\\
$X_{72} = S_{72} + X_{72}\cdot \Not{S_{73}}$\\
$X_{73} = S_{73} + X_{73}\cdot \Not{S_{74}}$\\
$X_{74} = S_{74} + X_{74}\cdot \Not{S_{75}}$\\
$X_{75} = S_{75} + X_{75}\cdot \Not{S_{76}}$\\
$X_{76} = S_{76} + X_{76}\cdot \Not{S_{70}}$\\[\sepsGrups]

% Marcas / salidas
$\mathtt{Maq\_Libre}     = X_{70}\cdot \Not{\mathtt{STOP\_ACT}}$\\
$\mathtt{QC5}           = (X_{71}+X_{72}+\mathtt{Mantenimiento\_Cinta}+\mathtt{vaciado\_cinta})\cdot \Not{\mathtt{STOP\_ACT}}$\\
$\mathtt{QC6}           = (X_{71}+X_{72}+\mathtt{Mantenimiento\_Cinta}+\mathtt{vaciado\_cinta})\cdot \Not{\mathtt{STOP\_ACT}}$\\
$\mathtt{QC7}           = (X_{71}+X_{72}+X_{73}+X_{76}+\mathtt{Mantenimiento\_Cinta}+\mathtt{vaciado\_cinta})\cdot \Not{\mathtt{STOP\_ACT}}$\\
$\mathtt{QCLAMP}        = (X_{73}+\mathtt{Mantenimiento\_Clamper})\cdot \Not{\mathtt{STOP\_ACT}}$\\
$\mathtt{QCLAMP\_SUB}   = (X_{76}+\mathtt{Mantenimiento\_SubirClamp}+\mathtt{vaciado\_subclamp})\cdot \Not{\mathtt{STOP\_ACT}}$\\
$\mathtt{Act\_SIG\_MAQ} = X_{76}\cdot \Not{\mathtt{STOP\_ACT}}$\\
$\mathtt{HayPieza}      = X_{74}\cdot \Not{\mathtt{STOP\_ACT}}$\\


\subsection{Grafcet máquina 5 vacuna y 6 soporte }
En la siguiente figura (\hyperref[fig:maq5_vac_6_sop]{Máquina 5 vacuna y 6 soporte}) observamos la máquina que forman la máquina número 5 de la línea de vacunas, 
y la número 6 de la línea de soportes, conformada por los siguietnes elementos:
\begin{itemize}
	\item 3 Cintas
	\item 1 Robot de 2 Ejes (Y-Z), con un gripper
\end{itemize}

\begin{figure}[H]
	\centering
	\includegraphics[width=0.8\linewidth]{./Figuras/maq5_VAC_6_SOP.png}
	\medskip
	\label{fig:maq5_vac_6_sop}
	\caption{Máquina 5 de la línea de soportes}
\end{figure}

El funcionamineto que describe el grafcet (\hyperref[graf:Empaquetado]{Grafcet Máquina 5 vacuna y 6 soporte}) es el siguiente, activamos 
las 3 cintas hasta que se produzca un flanco de bajada del sensor que hay antes del clamper
, cuando se produce el flanco nos esperamos 1 segundo, y activamos el clamper para que la pieza se quede 
colocada correctamente, y nos ponemos a la espera de que en la línea de vacunas nos avisen que tenemos un soporte
en el sitio, o en la línea de soportes, tengamos un embalaje en posición. A partir de ahí bajamos el robot durante 2
segundos, activamos el gripper, y con el gripper activado, subimos el robot para después desplazarlo en horizontal
y bajarlo, aquí ya soltaremos el gripper y devolveremos el robot a su posición de reposo.



\begin{figure}[H]
	\centering
	\scalebox{0.75}{%\nodeDist = 2.5cm
%\retornoDist = 0.7\nodeDist % Distància vertical dels retorns
%\bifDistX = 10ex
%\bifDistY = 1.1\nodeDist
%\sincDistYabove = 0.85\nodeDist
%\sincDistYbelow = 0.6\sincDistYabove
%\sincDistYBlockbelow = 0.5\nodeDist


\begin{tikzpicture}[auto]
	
	\ttfamily
	
	\etapaInicial{50}
	\primeraAccion{50}{X50-A1}{MAQ\_CAL\_VAC\_LIBRE}

	\etapa[1\nodeDist]{51}{50}{Act\_Maq};
	\primeraAccion{51}{X51-A1}{QC5\_VAC}
	\otraAccion{X51-A1}{X51-A2}{QC6\_VAC}
	\otraAccion{X51-A2}{X51-A3}{QC7\_VAC}

	%---

	%---
	\etapa[1\nodeDist]{52}{51}{S5};
	\primeraAccion{52}{X52-A1}{QC5\_VAC}
	\otraAccion{X52-A1}{X52-A2}{QC6\_VAC}
	\otraAccion{X52-A2}{X52-A3}{QC7\_VAC}
	

	%---
	\etapa[1\nodeDist]{53}{52}{1s/X52};
	\primeraAccion{53}{X53-A1}{QCLAMP}

	\etapa[1\nodeDist]{54}{53}{1s/X53};
	\primeraAccion{54}{X54-A1}{HAYVacuna}
	\otraAccion{X54-A1}{X54-A2}{QCLAMP}

	\etapa[1\nodeDist]{55}{54}{Hay\_Soporte};
	\primeraAccion{55}{X55-A1}{QBAJ}
	\otraAccion{X55-A1}{X55-A2}{QCLAMP}

	\etapa[1\nodeDist]{56}{55}{2s/X55};
	\primeraAccion{56}{X56-A1}{QBAJ}
	\otraAccion{X56-A1}{X56-A2}{GRIPPER}
	\otraAccion{X56-A2}{X56-A3}{QCLAMP}

	\etapa[1\nodeDist]{57}{56}{2s/X56};
	\primeraAccion{57}{X57-A1}{GRIPPER}
	\otraAccion{X57-A1}{X57-A2}{QCLAMP}

	\etapa[1\nodeDist]{58}{57}{2s/X57};
	\primeraAccion{58}{X58-A1}{GRIPPER}
	\otraAccion{X58-A1}{X58-A2}{QHOR}

	\etapa[1\nodeDist]{59}{58}{2s/X58};
	\primeraAccion{59}{X59-A1}{GRIPPER}
	\otraAccion{X59-A1}{X59-A2}{QBAJ\_MAQ}
	\otraAccion{X59-A2}{X56-A3}{QHOR}

	\etapa[1\nodeDist]{60}{59}{2s/X59};
	\primeraAccion{60}{X60-A1}{QHOR}
	\otraAccion{X60-A1}{X60-A2}{QBAJ\_MAQ}

	\etapa[1\nodeDist]{61}{60}{2s/X60};
	\primeraAccion{61}{X61-A1}{QHOR}

	\etapa[1\nodeDist]{62}{61}{2s/X61};
	\primeraAccion{62}{X62-A1}{Pieza\_Empaquetada}

	
	

	\retornoInicio[13em]{62}{0}{\underline{1}}
	













	




	

            
\end{tikzpicture}}
	\medskip
	\label{graf:Empaquetado}
	\caption{Grafcet de la Máquina 5 de vacunas y 6 de soportes}
\end{figure}

Este grafcet está descompuesto en las siguientes ecuaciones algebráicas:\\

\setlength{\sepsGrups}{2ex}

% Temporizadores (formato tipo ejemplo)
$T_{1}=2s/X_{52}$\\
$T_{2}=2s/X_{53}$\\
$T_{3}=2s/X_{55}$\\
$T_{4}=2s/X_{56}$\\
$T_{5}=2s/X_{57}$\\
$T_{6}=2s/X_{58}$\\
$T_{7}=2s/X_{59}$\\
$T_{8}=2s/X_{60}$\\
$T_{9}=2s/X_{61}$\\[\sepsGrups]

% PRODUCCIÓN

% Transiciones (sin FORCE_CURRENT)
$S_{50} = \mathtt{FirstScan} + X_{62}$\\
$S_{51} = X_{50}\cdot \mathtt{Act\_Proceso}$\\
$S_{52} = X_{51}\cdot \mathtt{S5}$\\
$S_{53} = X_{52}\cdot \mathtt{T1.Q}$\\
$S_{54} = X_{53}\cdot \mathtt{T2.Q}$\\
$S_{55} = X_{54}\cdot \mathtt{HaySoporte}$\\
$S_{56} = X_{55}\cdot \mathtt{T3.Q}$\\
$S_{57} = X_{56}\cdot \mathtt{T4.Q}$\\
$S_{58} = X_{57}\cdot \mathtt{T5.Q}$\\
$S_{59} = X_{58}\cdot \mathtt{T6.Q}$\\
$S_{60} = X_{59}\cdot \mathtt{T7.Q}$\\
$S_{61} = X_{60}\cdot \mathtt{T8.Q}$\\
$S_{62} = X_{61}\cdot \mathtt{T9.Q}$\\[\sepsGrups]

% Etapas (sin FORCE_INIT)
$X_{50} = S_{50} + X_{50}\cdot \Not{S_{51}}$\\
$X_{51} = S_{51} + X_{51}\cdot \Not{S_{52}}$\\
$X_{52} = S_{52} + X_{52}\cdot \Not{S_{53}}$\\
$X_{53} = S_{53} + X_{53}\cdot \Not{S_{54}}$\\
$X_{54} = S_{54} + X_{54}\cdot \Not{S_{55}}$\\
$X_{55} = S_{55} + X_{55}\cdot \Not{S_{56}}$\\
$X_{56} = S_{56} + X_{56}\cdot \Not{S_{57}}$\\
$X_{57} = S_{57} + X_{57}\cdot \Not{S_{58}}$\\
$X_{58} = S_{58} + X_{58}\cdot \Not{S_{59}}$\\
$X_{59} = S_{59} + X_{59}\cdot \Not{S_{60}}$\\
$X_{60} = S_{60} + X_{60}\cdot \Not{S_{61}}$\\
$X_{61} = S_{61} + X_{61}\cdot \Not{S_{62}}$\\
$X_{62} = S_{62} + X_{62}\cdot \Not{S_{50}}$\\[\sepsGrups]

% Marcas / salidas
$\mathtt{Maq\_Libre}        = X_{50}\cdot \Not{\mathtt{STOP\_ACT}}$\\
$\mathtt{QC5}              = (X_{51}+X_{52}+\mathtt{Mantenimiento\_Cinta}+\mathtt{vaciado\_cinta})\cdot \Not{\mathtt{STOP\_ACT}}$\\
$\mathtt{QC6}              = (X_{51}+X_{52}+\mathtt{Mantenimiento\_Cinta}+\mathtt{vaciado\_cinta})\cdot \Not{\mathtt{STOP\_ACT}}$\\
$\mathtt{QC7}              = (X_{51}+X_{52}+X_{53}+X_{54}+\mathtt{Mantenimiento\_Cinta}+\mathtt{vaciado\_cinta})\cdot \Not{\mathtt{STOP\_ACT}}$\\
$\mathtt{QCLAMP}           = (X_{53}+X_{54}+X_{55}+X_{56}+X_{57}+\mathtt{Mantenimiento\_Clamper})\cdot \Not{\mathtt{STOP\_ACT}}$\\
$\mathtt{QBAJ}             = (X_{55}+X_{56}+X_{59}+X_{60}+\mathtt{Mantenimiento\_BajarGripper})\cdot \Not{\mathtt{STOP\_ACT}}$\\
$\mathtt{QGRIPPER}         = (X_{56}+X_{57}+X_{58}+X_{59}+\mathtt{Mantenimiento\_Gripper})\cdot \Not{\mathtt{STOP\_ACT}}$\\
$\mathtt{QHOR}             = (X_{58}+X_{59}+X_{60}+X_{61}+\mathtt{Mantenimiento\_MOVHOR})\cdot \Not{\mathtt{STOP\_ACT}}$\\
$\mathtt{qsubclamp}        = \mathtt{vaciado\_subclamper}\cdot \Not{\mathtt{STOP\_ACT}}$\\
$\mathtt{HayPieza}         = X_{54}\cdot \Not{\mathtt{STOP\_ACT}}$\\
$\mathtt{Pieza\_Empaquetada}= X_{62}\cdot \Not{\mathtt{STOP\_ACT}}$\\
