
\usepackage{ifthen}

\usepackage{tikz}
%\usepackage{pgf}
\usepackage{pgfplots,xcolor}
%\usepgfmodule{plot}
\usetikzlibrary{
	shapes,
	arrows.meta,
	calc,
	patterns,
	decorations.pathmorphing,
	decorations.markings,
	decorations.shapes,
	backgrounds,	
	fit
	}


% -------------------------------------------------------------------------------------
% Longintuds

\newlength{\nodeDist}
\nodeDist = 2.5cm

\newlength{\retornoDist} % Distància vertical dels retorns
\retornoDist = 0.5\nodeDist

\newlength{\enlaceRetornoDist} % Distància vertical de la tornada (per dalt)
\enlaceRetornoDist = 0.125\nodeDist

\newcommand{\transitionPos}{0.5}
\newcommand{\transitionPosRetorno}{0.25}
\newcommand{\ArrowPos}{0.5}

\newlength{\bifDistX}
\bifDistX = 10ex

\newlength{\bifDistY}
\bifDistY = 1.1\nodeDist

\newlength{\sincDistYabove}
\sincDistYabove = 0.85\nodeDist

\newlength{\sincDistYbelow}
\sincDistYbelow = 0.6\sincDistYabove

\newlength{\sincDistYBlockbelow}
\sincDistYBlockbelow = 0.5\nodeDist

\newlength{\actionMinimumWidth}
\actionMinimumWidth = 5em


% --------------------------

\newdimen\lineWidth
\lineWidth = 1.15pt

\newlength{\blockLineWidth}
\blockLineWidth = 1pt

\newlength{\transitionsLineWidth}
\transitionsLineWidth = 1pt

% --------------------------

\newlength{\yTa} 
\yTa = 1.75em

\newlength{\yTb}
\yTb = 0.35em


% -------------------------------------------------------------------------------------
% Estils

\tikzset{
	big arrow/.style={
		decoration={markings,mark=at position 1 with {\arrow[scale=2.0,#1]{>}}},
		postaction={decorate},
		shorten >=0.4pt},
	big arrow/.default=black
	}

\tikzstyle{block} = [
	draw, 
	fill = white, %black!5, 
	rectangle, 
    minimum height = 3em, 
    minimum width = 3em,
    line width = \blockLineWidth,
    %thin
    ]

\tikzstyle{action} = [
	draw, 
	fill = white, %black!5, 
	rectangle, 
    minimum height = 3em, 
    minimum width = \actionMinimumWidth,
    line width = \blockLineWidth,
    %thin
    ]

\tikzstyle{signal} = [coordinate]
\tikzstyle{pinstyle} = [pin edge={<-, line width = 1.15pt, >=latex, black}]


% -------------------------------------------------------------------------------------
% La vida moderna

\newcommand{\RTrigg}{$\uparrow$}
\newcommand{\FTrigg}{$\downarrow$}

\newcommand{\NOT}[1]{$\overline{\mathtt{#1}}$}
\newcommand{\NOTs}[1]{$\overline{\mathsf{#1}}$}
\newcommand{\Not}[1]{\overline{#1}}

% -------------------------------------------------------------------------------------
% Comandaments

\newcommand{\dibujaTransicion}[1]{%
	\draw[line width = \transitionsLineWidth, -] ($(#1.west)-(\yTa,0cm)$) -- ($(#1.west)-(\yTb,0cm)$);
	}

\newcommand{\dibujaTransicionVertical}[1]{%
	\draw[line width = \transitionsLineWidth, -] ($(#1.north)+(0cm,0.9\yTa)$) -- ($(#1.north)+(0cm,0.5\yTb)$);
	}

\newcommand{\transicion}[3]{%
	\draw[line width = \transitionsLineWidth, -] (X#1) -- node[pos=\transitionPos, xshift=1em] (T#2) {#3} (X#2);
	\dibujaTransicion{T#2}
	}

\newcommand{\sinTransicion}[3]{%
	\draw[line width = \transitionsLineWidth, -] (X#1) -- (X#2);
	}

\newcommand{\etapaInicial}[1]{%
	\node[block] (X#1) {};
	\draw[line width = \blockLineWidth] ($(X#1.north west)+(0.3em,-0.3em)$) rectangle 
		($(X#1.south east)-(0.3em,-0.3em)$);
	\node[below=0.25ex,anchor=south] at (X#1) {#1};
	}
	
\newcommand{\etapa}[4][\nodeDist]{%
   	\node[block, node distance = #1, below of = X#3] (X#2) {};
	\node[below=0.25ex,anchor=south] at (X#2) {#2};
   	\transicion{#3}{#2}{#4}
	}

\newcommand{\etapaFantasma}[2]{%
   	\node[signal, below of = X#2, node distance = \nodeDist] (X#1) {#1};
   	\sinTransicion{#2}{#1}
	}
		
\newcommand{\primeraAccion}[3]{%
   	\node[action, right of = X#1, node distance = 3em, right] (#2) {#3};
   	\draw[line width = \transitionsLineWidth, -] (X#1) -- (#2);
}

\newcommand{\otraAccion}[3]{%
   	\node[action, right, xshift=-\blockLineWidth] (#2) at (#1.east) {#3};
}

\newcommand{\condicionada}[2]{%
   	\node[above right, xshift=1em] at (#1.north west) {#2};
   	\draw[line width = \transitionsLineWidth] ([xshift=1em]#1.north west) -- ([xshift=1em,yshift=1.25em]#1.north west);
}

\newcommand{\primeraAccionSobreDesactivacion}[3]{% OBSOLETA
   	\node[action, right of = X#1, node distance = 3em, right] (#2) {#3};
   	\draw[line width=\blockLineWidth = ->, arrows={-Triangle[angle=45:10pt]}] 
   		([xshift=0.5\blockLineWidth]#2.south west) -- ([yshift=-3ex]#2.south west);
   	\draw[line width = \transitionsLineWidth, -] (X#1) -- (#2);
}

\newcommand{\sobreDesactivacion}[1]{%
   	\draw[line width=\blockLineWidth = ->, arrows={-Triangle[angle=45:10pt]}] 
   		([xshift=0.5\blockLineWidth]#1.south west) -- ([yshift=-3ex]#1.south west);
}

\newcommand{\sobreActivacion}[1]{%
   	\draw[line width=\blockLineWidth = ->, arrows={-Triangle[angle=45:10pt]}] 
   		([xshift=0.5\blockLineWidth]#1.north west) -- ([yshift=+3ex]#1.north west);
}
	
%\bifurcacion[7ex]{14}{15}{Tansicion_15}{16}{Tansicion_16}	
\newcommand{\bifurcacion}[6][\bifDistX]{%
   	\node[block, below of = X#2, xshift=-#1, yshift=-\bifDistY] (X#3) {};
	\node[below=0.25ex, anchor=south] at (X#3) {#3};
   	\node[signal, yshift=4em] at (X#3.north) (cantoX#3) {};
	\draw[line width = \transitionsLineWidth, -] (X#2) |- (cantoX#3) -- node[xshift=1em] (T#3) {#4} (X#3);
	\dibujaTransicion{T#3};

   	\node[block, below of = X#2, xshift=#1, yshift=-\bifDistY] (X#5) {};
	\node[below=0.25ex, anchor=south] at (X#5) {#5};
   	\node[signal, yshift=4em] at (X#5.north) (cantoX#5) {};
	\draw[line width = \transitionsLineWidth, -] (X#2) |- (cantoX#5) -- node[xshift=1em] (T#5) {#6} (X#5);
	\dibujaTransicion{T#5};
	}

%\bifurcacionRetorno[7ex]{14}{15}{16}
\newcommand{\bifurcacionRetorno}[4][\bifDist]{%
   	\node[signal, below of = X#2, xshift=-#1, yshift=-0.15\bifDistY] (X#3) {};
   	\node[signal,yshift=0em] at (X#3.north) (cantoX#3) {};
	\draw[line width = \transitionsLineWidth, -] (X#2) |- (cantoX#3) -- (X#3);

   	\node[signal, below of = X#2, xshift=#1, yshift=-0.15\bifDistY] (X#4) {};
   	\node[signal,yshift=0em] at (X#4.north) (cantoX#4) {};
	\draw[line width = \transitionsLineWidth, -] (X#2) |- (cantoX#4) -- (X#4);
	}

%\sincronizacion[7ex]{14}{TRANSICION}{15}{16}
\newcommand{\sincronizacion}[5][\bifDist]{%
   	\node[signal, yshift=-\sincDistYabove] at (X#2) (SyncX#2) {}; % Punt central de la sincronització   		
	\draw[line width = \transitionsLineWidth, -] (X#2) -- node[xshift=1em] (T#2) {#3} (SyncX#2);
	\dibujaTransicion{T#2};
   	\node[signal, xshift=-#1] at (SyncX#2) (SyncEsq) {};
   	\node[signal, xshift=+#1] at (SyncX#2) (SyncDer) {};
   	\node[block, below of = SyncX#2, xshift=-#1, yshift=-0.1\nodeDist] (X#4) {};
	\node[below=0.25ex, anchor=south] at (X#4) {#4};
   	\node[block, below of = SyncX#2, xshift=#1, yshift=-0.1\nodeDist] (X#5) {};
	\node[below=0.25ex, anchor=south] at (X#5) {#5};
	\draw[line width = \transitionsLineWidth, -] (SyncEsq) -- (X#4);
	\draw[line width = \transitionsLineWidth, -] (SyncDer) -- (X#5);
	\draw[line width = \transitionsLineWidth, -, double, double distance = 0.2em] ($(SyncEsq)-(3em,0em)$) -- ($(SyncDer)-(-3em,0em)$);
	}

%\concurrencia{15}{16}{TRANSICION}{17}
\newcommand{\concurrencia}[4]{%
   	\node[signal,yshift=-\sincDistYbelow] at ($(X#1) !0.5! (X#2)$) (SyncX#4) {}; % Punt central de la sincronització
	\draw[line width = \transitionsLineWidth, -] (X#1) -- ($(X#1)-(0em,\sincDistYbelow)$);
	\draw[line width = \transitionsLineWidth, -] (X#2) -- ($(X#2)-(0em,\sincDistYbelow)$);
	\node[block, below of = SyncX#4, yshift=-\sincDistYBlockbelow] (X#4) {};
	\node[below=0.25ex,anchor=south] at (X#4) {#4};
	\draw[line width = \transitionsLineWidth, -] (SyncX#4) -- node[xshift=1em] (T#4) {#3} (X#4);
	\dibujaTransicion{T#4};
	\draw[line width = \transitionsLineWidth, -, double, double distance = 0.2em] ($(X#1)-(3em,\sincDistYbelow)$) -- ($(X#2)-(-3em,\sincDistYbelow)$);
}

%\concuSinco{15}{16}{TRANSICION}{17}{18}
\newcommand{\concuSinco}[5]{%
   	\node[signal,yshift=-\sincDistYbelow] at ($(X#1) !0.5! (X#2)$) (SyncA) {}; % Punt central de la sincronització
	\draw[line width = \transitionsLineWidth, -] (X#1) -- ($(X#1)-(0em,\sincDistYbelow)$);
	\draw[line width = \transitionsLineWidth, -] (X#2) -- ($(X#2)-(0em,\sincDistYbelow)$);
	
	\node[signal, below of = SyncA, yshift=-\sincDistYBlockbelow] (SyncB) {};
	\draw[line width = \transitionsLineWidth, -] (SyncA) -- node[xshift=1em] (T#4#5) {#3} (SyncB);
	\draw[line width = \transitionsLineWidth, -, double, double distance = 0.2em] ($(X#1)-(3em,\sincDistYbelow)$) -- ($(X#2)-(-3em,\sincDistYbelow)$);

	\dibujaTransicion{T#4#5};
   	\node[signal, xshift=-3cm] at (SyncB) (SyncEsq) {};
   	\node[signal, xshift=+3cm] at (SyncB) (SyncDer) {};
   	\node[block, below of = SyncB, xshift=-3cm, yshift=-0.1\nodeDist] (X#4) {};
	\node[below=0.25ex, anchor=south] at (X#4) {#4};
   	\node[block, below of = SyncB, xshift=3cm, yshift=-0.1\nodeDist] (X#5) {};
	\node[below=0.25ex, anchor=south] at (X#5) {#5};
	\draw[line width = \transitionsLineWidth, -] (SyncEsq) -- (X#4);
	\draw[line width = \transitionsLineWidth, -] (SyncDer) -- (X#5);
	\draw[line width = \transitionsLineWidth, -, double, double distance = 0.2em] ($(SyncEsq)-(3em,0em)$) -- ($(SyncDer)-(-3em,0em)$);
}

%\retornoInicio{13}{10}{JUAN}
\newcommand{\retornoInicio}[4][4em]{% Només per l'esquerra...
	\node[signal] at ($(X#2) !\ArrowPos! (X#3)$) (dif) {};
	\node[signal] at ($(X#2 |- dif)-(#1,0em)$) (flech) {};
	\draw[line width = \transitionsLineWidth, ->, arrows={-Triangle[angle=45:10pt]}] 
		(X#2.south) |- node[pos=\transitionPosRetorno, xshift=1em] (T#3) {#4}
		($(X#2.south -| flech)-(0em,\retornoDist)$) --
		(flech);
	\draw[line width = \transitionsLineWidth, -] 
		([yshift=-5pt]flech) --
		($(X#3.north -| flech)+(0em,2em)$) -|
		(X#3.north);
	\dibujaTransicion{T#3};
	}

%\Flecha{Etapa}{DistanciaHorizonal}{DistanciaVertical}
\newcommand{\Flecha}[3]{
	\node[signal] at ($(X#1) + (#2,#3)$) (flech) {};
	\draw[line width = \transitionsLineWidth, ->, arrows={-Triangle[angle=45:10pt]}] 
		([yshift=-0.25cm]flech) -- ([yshift=+0.25cm]flech);
	}


\newcommand{\retornoInicioNF}[4][4em]{% Només per l'esquerra NO FLETXA
	\node[signal] at ($(X#2) !0.5! (X#3)$) (dif) {};
	\node[signal] at ($(X#2 |- dif)-(#1,0em)$) (flech){};
	\draw[line width = \transitionsLineWidth, -] 
		(X#2.south) |- node[pos=0.25, xshift=1em] (T#3) {#4}
		($(X#2.south -| flech)-(0em,\retornoDist)$) --
		(flech);
	\draw[line width = \transitionsLineWidth, -] 
		([yshift=-5pt]flech) --
		($(X#3.north -| flech)+(0em,2em)$) -|
		(X#3.north);
	\dibujaTransicion{T#3};
	}

%\retorno[12ex]{17}{11}{ANTONIO}
\newcommand{\retorno}[4][12em]{% Només per la dreta...
	\node[signal] at ($(X#2) !0.5! (X#3)$) (dif) {};
	\node[signal] at ($(X#2 |- dif)-(-#1,0em)$) (flech) {};
	\draw[line width = \transitionsLineWidth, ->, arrows={-Triangle[angle=45:10pt]}] 
		(X#2.south) |- node[pos=0.25, xshift=1em] (T#3) {#4}
		($(X#2.south -| flech)-(0em,\retornoDist)$) --
		(flech);
	\draw[line width = \transitionsLineWidth, -] 
		([yshift=-6pt]flech) |-
		($(X#3.north)-(0em,-\enlaceRetornoDist)$);
	\dibujaTransicion{T#3};
	}

%\retornoHorizontal[12ex]{17}{11}{CondicioDeTransicio}
\newcommand{\retornoHorizontal}[4][12em]{% Només per la dreta...
	\node[signal] at ($(X#2) !0.5! (X#3)$) (dif) {};
	\node[signal] at ($(X#2 |- dif)-(-#1,0em)$) (flech) {};
	\draw[line width = \transitionsLineWidth, ->, arrows={-Triangle[angle=45:10pt]}] 
		(X#2.south) |-
		($(X#2.south -| flech)-(0em,\retornoDist)$) --
		node[below, pos=0, xshift=-0.5*#1, yshift=-2ex] (T#3) {#4} (flech);
	\draw[line width = \transitionsLineWidth, -] 
		([yshift=-6pt]flech) |-
		($(X#3.north)-(0em,-\enlaceRetornoDist)$);
	\dibujaTransicionVertical{T#3};
	}

%\retornoHorizontalX[12ex]{X17}{X11}{CondicioDeTransicio} 
\newcommand{\retornoHorizontalX}[4][12em]{% Només per la dreta...
	\node[signal] at ($(#2) !0.5! (#3)$) (dif) {};
	\node[signal] at ($(#2 |- dif)-(-#1,0em)$) (flech) {};
	\draw[line width = \transitionsLineWidth, ->, arrows={-Triangle[angle=45:10pt]}] 
		(#2.south) |-
		($(#2.south -| flech)-(0em,\retornoDist)$) --
		node[below, pos=0, xshift=-0.5*#1, yshift=-2ex] (T#3) {#4} (flech);
	\draw[line width = \transitionsLineWidth, -] 
		([yshift=-6pt]flech) |-
		($(#3.north)-(0em,-\enlaceRetornoDist)$);
	\dibujaTransicionVertical{T#3};
	}

%\retornoHorizontalXIzqZ[12ex]{X17}{CondicioDeTransicio} 
\newcommand{\retornoHorizontalXIzqZ}[3][12em]{% Només per l'esquerra. Va fins a una de pujada ja existent.
	\node[signal] at ($(#2) - (#1,\retornoDist)$) (fin) {};
	\draw[line width = \transitionsLineWidth] 
		(#2.south) |- node[below, pos=0.75, yshift=-2ex] (T#2) {#3} (fin);
	\dibujaTransicionVertical{T#2};
	}


%\comentari[1]{12}{Máquina fresadora en marcha}
\newcommand{\comentari}[3][0]{%
	\node [right] (C#2) at ([xshift=0.5em,yshift=#1]#2.east) {\sffamily\scriptsize``#3''};
}

% -------------------------------------------------------------------------------------
% Fi
